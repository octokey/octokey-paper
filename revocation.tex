\section{Key revocation}

What happens if one or more of the user's trusted devices are stolen? The passphrase with which the
key is encrypted can only delay an attacker from decrypting it, not reliably prevent decryption. So
it needs to be possible to revoke a device's ability to authenticate as a particular user.

The easiest way would be for each of a user's devices to have a different keypair and to register
all of a user's public keys with each of the services they want to log in to. Then revocation could
be implemented in two ways:

\begin{itemize}
\item On every authentication request, or periodically, the website checks a key revocation list at
a designated URL. This approach is problematic because it is not secure by default: a buggy or
missing implementation of revocation checking is likely to go unnoticed for a long time, and so an
attempted revocation may have no effect. It also raises questions about the server hosting the
revocation list: it must be trusted to always return the correct list of revoked keys, otherwise it
could be abused for denial of service (by marking too many keys as revoked) or unauthorized access
(by not omitting revoked keys from the list). Users also need some mechanism for securely adding
keys to the revocation list, which introduces an authentication bootstrapping problem.
\item Alternatively, when a user wishes to revoke a key, they use a different key on a different
device (or a backup key stored in a secure location) to authenticate with every service that it has
access to, and ask each service to remove the compromised public key from the list of authorized
keys for that user. This can be done programmatically if each website provides an API for
manipulating the authenticated user's public keys. However, it may still be slow (the user may have
accounts on hundreds of different websites) and error-prone (websites may be temporarily unavailable
or not implement the API correctly).
\end{itemize}

We propose a different approach to key revocation, which does not depend on the websites'
implementation of any protocol or API, and which allows revocation to take instant effect.

Instead of using a different keypair on each of a user's devices, we propose that a user should have
only one keypair. That keypair is never stored on any one device, even in encrypted form; instead,
the private key is split, each device only stores a portion of the private key, and the portions
from two devices must be combined in order to obtain the entire private key. One portion of the key
without the other portion is useless.

Boneh \emph{et al.}~\cite{Boneh01} describe the mRSA (mediated RSA) algorithm, which is well suited
in this application, because it allows an RSA signature to be incrementally generated on several
different devices, each holding a portion of the private key, without the complete private key ever
being present on any one of the devices. This reduces the risk of the private key being stolen.

A private key $p$ may be partitioned in several different ways, e.g. $p = a_1+a_2 = b_1+b_2 =
c_1+c_2$. If a device containing key portions $a_1$ and $b_1$ is lost or stolen, then to revoke that
device's access, the user must delete key portions $a_2$ and $b_2$ from any devices on which they
may be stored.

When two devices $D_1$ and $D_2$ store corresponding parts of the private key, e.g. $a_1$ and $a_2$
respectively, then we may say that $D_1$ and $D_2$ are \emph{paired}. Any two devices controlled by
the user may be paired. Of course, keys can be partitioned into three or more portions, but we
consider two portions to provide a good compromise between revokability and convenience.

For an authentication request to be signed, any two devices that have been paired must cooperate.
Typically, but not necessarily, one of the two devices is the device on which the user wants to log
in. The other device may be a device physically close to them (e.g. a smartphone), or a designated
remote server (that needs to be partially trusted, as discussed below).


