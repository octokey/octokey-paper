\section{Questions for Authentication Systems}

\subsection{Preventing Offline Attacks}\label{sec:offline}

A device storing the private key may fall into the hands of an attacker, or may be remotely
compromised. Thus, the device must protect the private key with a human-to-machine authentication
step (e.g.\ password, PIN number, biometrics) that checks whether the person trying to access the
private key is indeed the owner of the device.

A common approach, for example in password managers, is to encrypt sensitive material with a
symmetric key that is derived from a master password using a slow key derivation function (KDF) such
as PBKDF2, bcrypt or scrypt. The goal is that this makes a brute-force offline attack so
computationally expensive as to be unfeasible. However, we believe that merely using a slow KDF is
not sufficient to stop a determined attacker who can command significant computing resources.

In Octokey, we use a zero-knowledge approach: an attacker cannot determine whether a password guess
is correct without making a request to a server, and a separate request is required for every guess.
This gives the server the opportunity to rate-limit requests.

Alternatively, cryptographic hardware (e.g.\ smart cards) can be set up to allow only a small number
of guesses, and lock or even destroy itself after some number of failed attempts. This is secure,
under the assumption that the key cannot be extracted from the hardware using other techniques.

With any technique that limits the number of password guesses there is a risk of denial of service
(e.g.\ a child playing with the locked device could unwittingly exhaust the password attempts and
cause self-destruction). For this reason, self-destruction is undesirable in many use cases, and a
rate limit on password guesses is preferable.

\subsection{Authorization of Revocation}\label{sec:auth-revocation}

If the device storing the private key is lost or stolen, the user needs a mechanism for revoking it
(the key can be protected against offline attacks, see section~\ref{sec:offline}, but nevertheless
an attacker may be able to break the human-to-machine authentication, given sufficient time). This
raises the question: how can the system ensure that only the legitimate owner of the key may revoke
it (to prevent denial of service), in the absence of a key identifying the user (since it has been
lost)? Various approaches have been proposed:

\begin{itemize}
\item If the user's identity was originally established out-of-band by a CA, e.g.\ by checking a
passport, then the same process can be used to confirm that the revocation request is genuine, and
the CA can add the user's certificate to a revocation list (CRL).
\item The protocol may support revocation through a separate revocation key, perhaps stored offline
on paper. However, this key would also be prone to loss as it is only rarely needed.
\item A user may register multiple authenticator devices with each service. 
\end{itemize}

\subsection{Backup of Private Key Material}\label{sec:backup}

see section~\ref{sec:theftloss}
