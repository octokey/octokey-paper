\section{Delegated login by mobile device}
Browser and phone meet at rendezvous point (URL in QR code) / communicate over rendezvous channel

We must assume that the contents of the barcode can be read by an attacker: it may be
electromagnetically snooped~\cite{Kuhn05}, or an attacker may simply point a camera at the victim's
screen. In order to establish a secure channel between the smartphone and the desktop browser, it is
therefore not appropriate to embed a symmetric private key in the barcode.

The browser has a separate keypair, with the private key identifying the browser itself (not the
user). The QR code contains the URL of the rendezvous point, and the fingerprint of the browser's
public key. When the rendezvous channel is established, the browser sends a message to the client
signed with the browser's private key; and the client checks the signature, and checks that the
fingerprint of the public key that made the signature matches the fingerprint in the QR code. Thus
man-in-the-middling would require an attacker to actually modify the on-screen QR code, rather than
just eavesdrop it. If the attacker has enough access to the device that they are able to manipulate
what is displayed on the screen, then they can impersonate the user after logging in anyway (so all
bets are off for that user account on that site). The user still needs to make sure that they are
not being tricked into logging in to a different site from the one they thought they were logging in
to; this can be done by displaying the URL of the login page on the mobile phone when the user is
prompted to approve the authentication request. If the URL is not what the user was expecting, they
must decline the authentication request.
