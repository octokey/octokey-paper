\section{Related work}\label{sec:otherapproaches}

In this section we compare Octokey to other cryptographic authentication protocols. Since Octokey
can be used with various human-to-machine authentication methods (password, biometrics, etc.), we
concentrate on the machine-to-machine aspect, and refer the reader to the literature for a survey of
human-to-machine authentication methods~\cite{Bonneau12}.

\subsection{Certivox M-Pin}

Certivox M-Pin~\cite{Scott14} is a cryptographic authentication protocol based on elliptic curves.
It has some similarities to Octokey: it reconstructs a client secret from a stored token and a
4-digit PIN (similar to the human-to-machine authentication step in Octokey), and the client proves
ownership of the secret to the server without revealing the secret.

However, each M-Pin client and server requires a secret that is issued to it by a \emph{trusted
authority} (TA). This is reasonable in the context of employees authenticating with their company's
systems, because clients, servers and TA are all controlled by the same organization. In a consumer
internet context, each service would need to set up its own TA, using hardware security modules to
protect server secrets, or rely on a TA service hosted by Certivox. This makes server deployment
more complicated and costly than Octokey, which requires only a simple RSA signature verification,
and no secrets on the server.

A service that accepts M-Pin login can compute the user's PIN; thus, a rogue service that colludes
with attackers who steal devices can reconstruct client secrets.\footnote{Unless the user has a
different PIN for each service, which takes us back to square one.} Moreover, M-Pin does not support
instantaneous key revocation, only time-based key expiry (``time permits'').

\subsection{2D barcode authentication}

Several previous proposals use 2D barcodes as part of a user authentication flow: for example, tiqr
(based on OCRA~\cite{OCRA}), eKaay~\cite{Borchert11} and SQRL~\cite{Gibson13} all use 2D barcodes.
A mobile device obtains a challenge from a service by scanning a barcode, signs that challenge using
a secret stored on the device, and returns the signed challenge to the service. The secret may be a
symmetric secret that is shared between client and server~\cite{OCRA} or an asymetric private
key~\cite{Gibson13}. These proposals have similarities to Octokey's delegated authentication
(section~\ref{sec:delegation}).

However, these proposals do not have good answers to the issues of key loss, key revocation and
distribution of keys to new devices. tiqr and eKaay appear to not have any revocation mechanism;
SQRL's revocation requires the user to retrieve an ``identity unlock key'' that is stored on paper,
and use it to individually reset their key on every service they use (similar to the master key
discussed in section~\ref{sec:recovery}). To authenticate on multiple devices, the user must either
copy the entire secret key to every device, or they must go through the barcode-scanning procedure
every time they want to authenticate. There is no way of revoking only one device, no easy way of
enrolling new devices, and no provision for key rotation.

Finally, these proposals do not have a rate limiting mechanism like in section~\ref{sec:ratelimit},
and rely solely on a slow PBKDF such as Scrypt~\cite{Percival09} to slow down offline attacks on an
encrypted secret. Depending on the strength of the user's password, this may not be sufficient to
stop a determined attacker who has stolen an encrypted secret.

\subsection{Other approaches}

Windows CardSpace~\cite{Chappell06} was a user interface and abstraction layer on top of X.509
certificates and other authentication mechanisms, but it was not successful in the marketplace.
U-Prove~\cite{Paquin13} is an ambitious project to build a superset of the X.509 PKI using
zero-knowledge proofs. It significantly more complicated and broader in scope than Octokey, since it
deals with much more than just authentication, such as the selective disclosure of certificate
attributes for privacy reasons. We fear that this complexity will prove an obstacle to its practical
adoption.

OTTA~\cite{Thomas14} is similar in philosophy to Octokey. It uses a separate keypair per identity
per device, and a public log (similar to Certificate Transparency~\cite{CertTrans}) for revocation
and for linking keys belonging to the same user. However, revocation of a lost key is awkward for
users, because it relies on a revocation certificate that is stored ``somewhere safe'' (offline?).
We fear that if key revocation is too awkward, users will be too lazy to revoke a lost device, and
thus risk key theft.

\section{Conclusion}

In this paper, we have introduced the design of Octokey, an authentication system intended for
consumer internet use cases. We argue that it provides better security than all commonly-used
alternatives (passwords, password managers, and federated authentication methods), and provides a
better user experience and better flexibility than more secure alternatives (hardware-based
authenticators, smart cards).

Octokey builds on mature, well-understood components such as TLS and RSA, and has similarities to
the familiar SSH public key authentication protocol. By keeping the new algorithms to a minimum, and
by using familiar and mature components where possible, we hope that the security characteristics of
the system will be easily understood by users and implementers.

It is quite simple for service owners to start accepting Octokey as an authentication mechanism,
since libraries for verifying PKCS\#1 signatures are available for all common programming languages.
Octokey can also be used alongside passwords or other authentication mechanisms, allowing it to be
adopted gradually. It does not require any special hardware or any changes to the way online
services are deployed.

Most of the complexity of Octokey lies in the implementation of the client applications and the
mediator service. We plan to develop these using an entirely open source model, and to make them
available as free software for anyone to run and modify. The protocols should be open standards, and
anybody should be free to create alternative implementations if they wish.

Octokey strikes a pragmatic compromise between security and convenience, and strives for a very
simple user experience with minimal risk of user error. We hope that it can be the viable
alternative to passwords that has eluded us for so long. We invite others to join the project, give
feedback, spread the word, and start working towards a real implementation.
