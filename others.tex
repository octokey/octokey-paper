% \section{Related work}
%
% Certivox M-Pin~\cite{Scott14}
%
% Resources:
%
% https://www.schneier.com/blog/archives/2014/09/security_of_pas.html
% https://crypton.io/
% https://www.lightbluetouchpaper.org/2014/10/02/pico-part-iv-somethings-you-have/
% http://news.engineering.utoronto.ca/bionym-raises-14-million-wearable-password-replacing-tech/
% https://www.digits.com/
% http://www.cl.cam.ac.uk/techreports/UCAM-CL-TR-817.pdf (already in Papers)
% http://www.certivox.com/docs/m-pin-core
% Microsoft CardSpace http://msdn.microsoft.com/en-us/library/aa480189.aspx?ppud=4
% SQRL

\section{Conclusion}

Octokey does not claim to provide `perfect' security (if such a thing exists); for example, it is
still susceptible to attacks from malware. However, we argue that it provides better security than
all commonly-used alternatives (passwords, password managers, and federated authentication methods),
and provides a better user experience and better flexibility than more secure alternatives
(hardware-based authenticators, smart cards).

It is quite simple for service owners to start accepting Octokey as an authentication mechanism,
since libraries for verifying PKCS\#1 signatures are available for all common programming languages.
Octokey can also be used alongside passwords or other authentication mechanisms, allowing it to be
adopted gradually.

Most of the complexity of Octokey lies in the implementation of the client applications and the
mediator service. We plan to develop these using an entirely open source model, and to make them
available as free software for anyone to run and modify. The protocols should be open standards, and
anybody should be free to create alternative implementations if they wish.

Octokey strikes a pragmatic compromise between security and convenience, and strives for a very
simple user experience with minimal risk of user error. We hope that it can be the viable
alternative to passwords that has eluded us for so long, and we invite others to join the project,
give feedback, spread the word, and start working towards a real implementation.
