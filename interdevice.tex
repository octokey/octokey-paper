\section{Inter-device communication}\label{sec:interdevice}

Octokey assumes that the user has multiple internet-connected devices: two at minimum (one physical
device and one mediator), but likely several more (desktop computer, laptop, smartphone, tablet,
game console, television, automobile, etc).

We assume that users want to be able to access all of their services on any one of their physical
devices (devices with partial access to the user's services are discussed in
section~\ref{sec:delegation}). Thus, each physical device should be paired at minimum with the
mediator. Some physical devices may also be paired with each other, to provide redundancy in case of
a failure of the mediator (discussed in section~\ref{sec:theftloss}).

The system should make it easy to enroll new devices to the user's set of trusted devices, and to
revoke any devices which are lost or no longer in use. It may also provide convenience features,
such as synchronizing the list of usernames for all of a user's services, so that they can be
autofilled in login forms. In order to do this, Octokey needs a mechanism for the user's devices to
synchronize with each other.

\subsection{Communication protocol}\label{sec:channels}

First, we need to be able to establish a secure point-to-point communication channel between any two
of the user's devices (treating the mediator as one of the devices). The channel must provide
secrecy and integrity, so that an attacker cannot eavesdrop or modify communcation between devices.
It must also prevent spoofing, so that an attacker cannot impersonate a device or MITM a channel.

TLS~\cite{TLS} meets these requirements. Each device generates its own private key and self-signed
certificate, which are used only for inter-device communication and are independent from the mRSA
key fragments. One device then connects to another and performs a TLS handshake, using the devices'
certificates as client and server certificates for mutual authentication.

This approach does not require any PKI or certificate authorities: public key fingerprints are
exchanged out-of-band when devices are first discovered (see section~\ref{sec:newdevice}), so each
device can check whether its peer is the device that it claims to be. It is not important which
device acts as client or server in this channel, as the relationship between the devices is
symmetric.

However, there is a practical problem: it can be difficult to establish a peer-to-peer TCP
connection between two arbitrary devices, as they may be on different networks, behind firewalls,
and they may often be offline -- and rarely online at the same time. Thus, in addition to performing
TLS over TCP, we propose using an alternative transport layer in the form of a relay service.

The relay is a web service that can be reached from all devices (it may be co-located with the
mediator). It accepts messages from one device and relays them to another device: if the recipient
is online, the messages are forwarded immediately, otherwise they are stored until the recipient is
next online. The messages are records of the TLS Record Protocol~\cite{TLS}, so two devices can
perform a TLS handshake and establish a secure channel by sending each other messages via the relay.
The service needs to be highly available, but it does not need to be trusted for security purposes.

If devices are only occasionally online, the round-trip-times across this ``network connection'' may
be measured in days, so timeouts may need to be adjusted accordingly. However, once this channel has
been set up, it can remain open indefinitely (with occasional renegotiation), even across device
reboots.

Besides this relay, devices should support alternative transport mechanisms, such as direct TCP
connections on a local network, Bluetooth or NFC. These may only occasionally be used, but they are
good fallbacks so that the relay does not become a single point of failure. The same TLS record
protocol can be used over those transport layers.

\subsection{Data synchronization}\label{sec:devicesync}

These point-to-point channels are used for exchanging mRSA signing requests and responses between
devices. Moreover, we can use them for replicating information that we want stored on all devices:
each of the user's devices maintains a local database containing public key fingerprints of all of
the user's other devices (for mutual authentication of channels), configuration (e.g. URL of the
mediator), and information on which devices are paired (for key revocation purposes). In addition,
the user's physical devices maintain a database of usernames on all the user's services (for
username autofill), although this should not be replicated to the mediator, to protect the user's
privacy.

Each of the user's devices needs to be able to read and modify this database, and any changes should
be asynchronously propagated to the other devices. To this end, we can use the point-to-point
channels of section~\ref{sec:channels} to create a mesh of connections among the user's devices
(even between those which are not paired). It is not necessary to have a communication channel
between every pair of devices, because devices can forward incoming database changes to others.
However, a fairly densely connected graph of channels is desirable, so that any message from one
device reaches the other devices fairly quickly, and so that the communication is resilient to
device and channel failures.

Within this mesh network, we can use a \emph{gossip protocol} (also known as \emph{epidemic
protocol}) \cite{Demers89} for disseminating database changes. \emph{Version vectors}
\cite{ParkerJr83} can be used to efficiently detect data differences that need to be synchronized,
and \emph{CRDTs}~\cite{Shapiro11} can automatically resolve any conflicts between concurrent
database updates, without requiring user interaction. Every database change must be signed with the
device key of the device on which it originated, so that any changes made on a revoked device can be
rejected, and so that one device cannot make changes on another device's behalf.

\subsection{Bootstrapping}\label{sec:bootstrap}

When someone first starts using Octokey, they go through the following steps:

\begin{enumerate}
\item The user installs the Octokey application on a physical device, and selects ``create new
Octokey''.
\item The application generates the user's RSA keypair $(n, d, e)$, as well as the keypair and
self-signed certificate for that device.\footnote{It is possible, albeit rather slower, to perform
the key generation distributed across multiple devices, so that no device ever has the entire $d$.
\cite{Boneh01b} However, if we want to allow the user to print off the entire key as a backup, we
need to have the entire key in one place anyway. See section~\ref{sec:theftloss} for a discussion.}
\item The application contacts a mediator of the user's choice (a default may be provided) over TLS,
using its device certificate for client authentication, and verifying the server using well-known
certificate authorities (or a pinned public key fingerprint embedded in the application).
\item The application splits the private exponent $d = d_1 + d_2$ such that $d_1$ is a uniformly
distributed random integer with $0 < d_1 < d$. The device stores $d_1$ locally, and sends $d_2$ to
the mediator.
\item The user has the option of printing the private key on paper, as a last-resort backup in case
all of their devices are lost (see section~\ref{sec:theftloss}). Then the application deletes $d$.
\end{enumerate}

\subsection{Enrolling a new device}\label{sec:newdevice}

When the user has already set up Octokey on one or more devices, and wishes to enroll another
physical device (i.e. pair a new device with their existing devices), they proceed as follows:

\begin{enumerate}
\item The user installs the Octokey application on the new device, and selects ``set up this device
to use your existing Octokey''.
\item The application generates that device's keypair and self-signed certificate. It then registers
itself at the relay, using its device certificate for client authentication. The relay allocates a
URL at which this device can be reached, using the device's public key fingerprint as recipient
identifier.
\item The application on the device displays a 2D barcode on screen, containing an indicator that
this is an enrollment request, the relay URL at which it can be reached, the device's public key
fingerprint, and optionally information about other transport channels (e.g. Bluetooth) where it can
be reached. The barcode must not contain any sensitive data, since it may be visible to
eavesdroppers (see section~\ref{sec:barcode-intercept}).
\item The user scans this barcode using the camera of their existing Octokey device. The application
connects to the URL in the barcode, negotiates an end-to-end TLS connection and checks that the
remote key fingerprint matches the one in the barcode.
\item The existing Octokey device prompts the user whether they want to trust the device whose
barcode they just scanned. If the user is already using multiple devices, the applications on those
devices may additionally prompt the user.
\item When sufficient user approvals have been obtained, the device that scanned the barcode
registers the new device (including its public key fingerprint) in the database. This change is
replicated to other devices, which may connect to the new device (adding it to the mesh network).
\item The new device initiates the key re-pairing protocol (section~\ref{sec:pairing}) in order to
obtain key fragments.
\end{enumerate}

This interaction flow assumes that the first physical device on which the user sets up their private
key has a camera. This is plausible, since many laptops and smartphones come with built-in cameras
today. In the case where no camera is available, the user can type in the contents of the barcode in
textual form. Devices which are enrolled later do not require a camera, only a screen. For example,
a game console or an automobile on-board computer may not have a camera, but they can nevertheless
be enrolled.

If the user's existing device does not have internet access, and thus cannot reach the URL in the
barcode, the devices can set up the secure channel over a local wireless medium such as
Bluetooth, NFC or 802.11/WiFi instead. The necessary information to establish this connection can be
included in the barcode. We require that the enrollment is initiated using the barcode, and not
through the wireless medium, to ensure that an attacker cannot tamper with the public key
fingerprint (see also section~\ref{sec:barcode-phishing}).

\subsection{Key re-pairing}\label{sec:pairing}

Each mRSA pairing of devices requires the private exponent $d$ to be split in different ways:
$d = d_1 + d_2 = d_3 + d_4 = \dots$ with $d_1 \neq d_2 \neq d_3 \neq d_4$ (all values are drawn from
a uniform random distribution, so there is a very small probability that they might be equal).

Say there are two existing devices: device $a$ stores key fragment $d_1$ and device $b$ stores key
fragment $d_2$. When a new device $c$ is enrolled, and we want to pair it with device $a$, we need
to generate new key fragments $d_3$ (stored on device $c$) and $d_4$ (stored on device $a$). We can
do this as follows, without reassembling the entire private exponent $d$ on any single device (and
thus risking it being stolen):

\begin{enumerate}
\item Device $c$ chooses a random integer $d_3$ from the uniform range $0 < d_3 < n$ (where $n$ is
the RSA modulus).
\item Device $c$ chooses a second random integer $k$ from the uniform range $0 < k < d_3$.
\item Device $c$ sends $k$ to device $a$, and sends $(d_3 - k)$ to device $b$.
\item Device $b$, where $d_2$ is stored, calculates $(d_2 - (d_3 - k))$ and sends it to device $a$.
This number may be negative.
\item Device $a$, where $d_1$ is stored, has now received $k$ and $(d_2 - (d_3 - k))$. It uses these
to calculate $$d_4 = d_1 + (d_2 - (d_3 - k)) - k = d_1 + d_2 - d_3.$$
\item If $d_4$ is positive, the algorithm was successful: device $a$ stores $d_4$ and we're done. If
$d_4$ is negative, we start again from step 1, choosing new random numbers.
\end{enumerate}

The algorithm ensures that $d_3 + d_4 = d$, so $d_4$ is negative iff $d_3 > d$. However, since no
device knows $d$, we cannot simply generate $d_3$ to be less than $d$, but we must retry with new
random numbers until a suitable $d_3$ is found.

A malicious device could use this algorithm to determine $d$ by binary search. However, there are
also many other ways for a malicious device to reconstruct $d$, for example by pairing with a
colluding device. Thus, we must trust the software on the user's physical devices (see
section~\ref{sec:malware}), and use the fact that a device can only initiate the pairing process
after it has been authorized by the user. We also prohibit the mediator from initiating the
re-pairing process.

\subsection{Number of retries}\label{sec:retries}

The key re-pairing algorithm in section~\ref{sec:pairing} requires a potentially unbounded
number of retries. To avoid excessive waiting times for the user, we want to keep the number of
retries fairly small.

The number of retries $R$ is a geometrically distributed random variable with an expected value of
$\frac{n}{d}-1$. This means that if the private exponent $d$ is significantly smaller than the
modulus $n$, the number of retries can be high.

We therefore suggest that when the private key is first generated, it must be greater than some
threshold, e.g. $d \ge \frac{n}{16}$. With this restriction, a worst-case key would require 15
retries on average, and would have an approximately 0.1\% risk of requiring more than 100 retries.
Most keys would require significantly fewer retries.

Although this restriction would exclude 1/16th of the possible keyspace, with a 2048-bit key this is
equivalent to reducing the key size to $\log_2(2^{2048} - 2^{2044}) \approx 2047.9$ bits. It seems
unlikely that a loss of 0.1 bits would significantly harm the strength of the algorithm, but in any
case the cryptanalytic consequences of such a key restriction need to be studied carefully.

\subsection{Delegated login by mobile device}\label{sec:delegation}

A user should be able to authenticate with services on a semi-trusted device, for example a game
console at a friend's home, without having to generate key fragments for that device (and thus
giving the device full access to all of the user's accounts). To this end, we use the fact that a
signed Octokey mandate is like a one-time password: it can be given to another device in order to
authorize one-off access to one particular service.

By logging in, the semi-trusted device exchanges the mandate for a session identifier (e.g. a
cookie), which is destroyed when the user logs out, or after a timeout. In cases where the user
wants to give a device long-lived access to one service, the device can obtain an OAuth~\cite{OAuth}
token from the service. This is standard use of OAuth and does not require any special support from
Octokey.

The question is then how to get the mandate onto the device that needs it. We propose the following
flow:

\begin{enumerate}
\item Steps 1 to 4 are identical to the flow in section~\ref{sec:newdevice}, except that the user
selects ``log in with your Octokey on another device'' instead of ``set up this device to use your
existing Octokey''. The 2D barcode indicates that this is a signing request, not an enrollment
request.
\item The device requesting login requests a challenge $c$ from the service.
\item Using the secure channel that has been set up between the devices, the device requesting login
sends $c$ and the login URL $u$ to the device that scanned the barcode.
\item The latter device prompts the user whether they want to allow the device whose barcode they
just scanned to log in to their account at URL $u$. If yes, they select the username $r$ of the
desired account (which may be autofilled from their database of services).
\item If the user approves, the device goes through the usual mRSA signing flow to calculate a
signature $s$, and sends the mandate
$s \concat c \concat u \concat r \concat \mathit{ex} \concat n \concat e$
to the device requesting authentication, using the secure channel. That device sends the mandate to
the service to log in.
\end{enumerate}

It is important that the user checks whether the authentication request is for the URL they were
expecting, because a device could potentially request login for any service. As a usability
improvement, the user could be shown a screenshot of the website at the requested URL, not just the
URL itself. This would make it visually obvious if the user's banking login is being requested, when
they thought they were logging into an online game. (As before, creating a phishing website with
imitation design would not work --- in order to obtain a valid mandate for the user's bank, an
attacker must supply the real bank's URL.)

This process for delegated authentication is simple and fast, has fairly small risk of user error,
and can be implemented with any smartphone. It is significantly more pleasant to use than a password
manager, does not require typing anything, and is more secure.

\subsection{Key rotation}\label{sec:rotation}

A key that is considered strong today may be regarded as weak in future. Key fragments may be left
on old, rarely-used devices, so the user may forget to revoke them when they are lost or thrown
away. Security vulnerabilities may from time to time raise fears that private key material may have
been exposed. For all these reasons, we must assume that every key has a finite lifespan, and build
mechanisms for key rotation into the Octokey protocol from the start.

As described in section~\ref{sec:mandates}, every user keypair has an expiry date. We propose making
this four months after the key's creation date by default. Services must not accept public keys
without associated expiry date, or keys that have expired, or keys with an invalid signature on the
expiry date.

One month before the user's current key expires, one of the user's devices generates a new key
(similar to section~\ref{sec:bootstrap}), and distributes key fragments to the user's enrolled
devices (section~\ref{sec:pairing}). Each service should implement a HTTP API by which a device can
renew the user's public key: a mandate signed with the user's current key is sufficient
authorization to add a new public key to that user's account. The old and new key are both accepted
until the old key expires (see discussion in section~\ref{sec:recovery}).

Thus, a device that lies unused in a box in the user's attic will automatically lose access to the
user's accounts between one and four months after it was last used. A device whose enrollment has
lapsed can be re-enrolled. However, devices in active use will not require re-enrollment, because
they are updated with the latest key before the old key expires.

Our goal is to allow key rotation to happen automatically, with as little user interaction as
possible. The only main place where user interaction occurs is for printing a backup of the new key
on paper (see section~\ref{sec:theftloss}). It would be annoying for users to have to print
something, and take it to their safe storage place, every three months. Perhaps even this could be
automated: one could imagine banks or attorneys providing an API to which a user can securely submit
their private key; the bank promises to print it, to store it in a safe, and to return it to the
user if they authenticate themselves in person (by presenting a passport). Such an API would allow
key rotation to happen entirely without user interaction.
