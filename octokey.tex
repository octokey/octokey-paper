\documentclass[letterpaper,twocolumn,10pt]{article}
\usepackage{usenix,epsfig}
\usepackage{cite}
\usepackage{hyperref}
\usepackage{amsmath}
\newcommand*{\concat}{\mathbin{\|}}
\hyphenation{time-stamp}
\sloppy

\begin{document}
\date{} % No date
\title{\Large \bf Octokey: Public key authentication for the web}
\author{{\rm Martin Kleppmann} \and {\rm Conrad Irwin}}
\maketitle

\subsection*{Abstract}

We propose Octokey, a cryptographic protocol for authenticating users, designed as a viable
replacement for password authentication in consumer internet use cases. When a user signs up to a
service, they provide a public key. Clients subsequently authenticate by presenting a challenge
signed with the user's private key, similar to SSH public key authentication. We also describe
protocols which allow users to log in to their accounts on multiple devices, enroll new devices,
and revoke keys on devices that are lost or stolen.
\end{abstract}

\bigskip

\emph{Note: We, the authors, have a background in developing commercial web applications. In the
field of crypography, we are merely keen amateurs. We are aware that amateur cryptography is prone
to catastrophic flaws, and we fully expect such flaws to be found in the ideas we are proposing.
However, we have experience designing real-world products with good user experiences, a perspective
that we feel is often lacking in security research. We hope that this paper can stimulate a
discussion that brings together the perspectives from research and industry, and we would like to
solicit constructive feedback from all sides.}

\section{Introduction}

Despite their many well-known flaws, passwords are still by far the most commonly used
authentication system on the web.~\cite{Bonneau12} However, as users sign up for ever more services,
their use is becoming increasingly unsustainable: demanding that users choose passwords with
sufficient entropy, use a different password for every site, not write them down and even rotate
passwords periodically is unrealistic for most internet services.

With phishing attacks and leaks of password databases unfortunately commonplace, we seek a better
solution. In this paper we focus on authentication in a consumer internet context, such as social
media or e-commerce websites and mobile apps. For a majority of such use cases, an authentication
system with the following properties is desirable:

\begin{enumerate}
\item The system should minimize exposure to human error (such as falling for a phishing attack).
Even with education, human error cannot be avoided completely, so the system should be designed such
that it minimizes the impact in case of an error.
\item In case a device is lost or stolen, the user should be able to easily revoke any credentials
stored on the device, to minimize the chances that an attacker can use a stolen device to gain
access to other systems.
\item It is not necessary, and often undesirable, for the system to authenticate the user as a
physical person. Many services allow pseudonymous signup, which is desirable for reasons of privacy
or freedom of speech. The purpose of the authentication system is thus only to verify that the
current user is the same person as the one who originally signed up under a particular username; no
authentication is performed at signup.
\item The system should not depend on any trusted third party, to avoid risks arising from failure
of the third party (security compromise, going out of business, change of policies, etc).
\item The system should be easy for end-users to install and use, and easy for service owners to
deploy. Since user signup and user activity rates are important business metrics for many online
services, the system should make signup and login \emph{easier} than using a password.
\item Users should have the freedom to use the system on a variety of devices, including shared or
public computers that are only partially trusted, whilst minimizing their exposure to attackers.
For example, a user may choose log in to `unimportant' services (e.g. casual online games) on a
shared or public computer, knowing that their account on that service may be compromised; however,
they limit their use of `important' accounts (e.g. online banking) to trusted devices. Since there
is no generally agreed boundary between `important' and `unimportant' services, the same
authentication system should be able to support both use cases.
\end{enumerate}

In the next section we discuss various authentication systems that are used on the web, and explain
why none of them meet these requirements. We then introduce Octokey, an authentication protocol
which addresses these issues. We believe that Octokey could be a viable solution to online
authentication in the near future, and we seek the information security community's feedback to
refine this proposal.

\section{Existing systems}
\subsection{Passwords}

Passwords have many of the desirable properties mentioned above: no dependence on any third party,
pseudonymity, simplicity of implementation, familiarity and cross-device compatibility. However,
remembering a large number of passwords is a big burden on users, and they are highly susceptible to
human error: use of weak passwords, reuse of the same password across different services, phishing
attacks and leaks of unhashed or weakly hashed password databases are sadly common.

Web browsers' built-in password managers, or external password manager products such as 1Password
and LastPass, make it feasible for users to maintain a strong, unique password for each service they
use. However, even when a unique password is used, an attacker who succeeds in stealing a password
(e.g. by phishing, malware, keylogging, man-in-the-middle attack, eavesdropping when it is
accidentally sent over an unencrypted connection, or exploiting a vulnerability in the password
manager~\cite{Li14, Silver14}) has access to that user account until the password is changed, which
is often a long time.

If an attacker steals the encrypted password database (perhaps by stealing user's device, or a
backup of the filesystem), they can mount an offline dictionary or brute-force attack on the
encryption password. If the user fears the password database may have been stolen, they must
manually change the password for each of their services, which is laborious if they use many
services. If an attacker succeeds in breaking the encryption before the passwords are changed, they
may lock the legitimate user out of their accounts.

Password managers also have grave exposure to human error. For example, consider a user who wishes
to log in to an `unimportant' website on an untrusted computer. It is very tempting for the user to
type the master passphrase of their password manager into the untrusted computer, in order to
decrypt the password database. This has the effect of exposing the user's entire password database
to an attacker running malware on the untrusted computer.

\subsection{One-time passwords and two-factor authentication}

One-time passwords (OTPs) are a big security improvement over regular passwords. The user must
download a list of one-time passwords on a trusted device, or register a cryptographic device that
generates a pseudorandom sequence of OTPs, or register an email address or phone number to which
OTPs can be sent on demand. Some services allow use of OTPs as sole authentication mechanism, while
others use them in conjunction with a regular password (two-factor authentication).

The advantage of OTPs is that the exposure to attacks is limited to a single service, and ends when
the user clicks the logout button (assuming the session is correctly invalidated on the server, and
assuming that the attacker does not have a means of extending their privileges, e.g. by granting
themselves OAuth access to the account).

However, OTPs are so inconvenient to use that very few services are willing to adopt them as their
only or primary authentication mechanism, and only the most security-conscious users are willing to
use them. Moreover, the common ``remember me on this device'' feature weakens two-factor
authentication (attackers just need to steal the ``remember me'' cookie as well as keylogging the
password).

OTPs sent on demand via email or SMS also have the effect of `outsourcing' the authentication to the
email provider or the mobile network provider, respectively: any attacker who can access the email
provider or SMS messages can also gain access to any service using on-demand OTPs. This essentially
makes them federated login schemes (see next section).

\subsection{Federated identity}

OpenID~\cite{OpenID} is probably the best-known attempt to remove the need for a separate password
on every service. Some identity providers, such as Google Federated Login~\cite{GoogleOpenID} and
Facebook Connect, combine the ideas of OpenID and OAuth~\cite{OAuth}. Mozilla Persona
(BrowserID)~\cite{Persona, BrowserID} uses a different protocol, but is similar in the way it
delegates authentication to the user's email provider (or a fallback service provided by Mozilla).

Such federated authentication protocols do not solve the fundamental problem of authentication: they
only delegate it to the identity provider, who must then use some other authentication method (most
commonly a password, possibly in conjunction with a hardware token). This means that all the
problems discussed above apply to the identity provider, with an additional privilege escalation
problem: if an attacker gains access to the user's account with the identity provider, they can
access any account associated with that identity. Like with a password manager, a single user error
can lead to all of a user's accounts being compromised.

Moreover, the relying party (the service where the user is trying to log in) needs to trust the
identity provider to correctly authenticate the user.  If the identity provider is compromised, the
relying party has no way of detecting unauthorized logins from that identity provider. If the
identity provider experiences an outage or goes out of business, users of that provider lose their
ability to log in (unless they had previously set up delegation of their identity URL, which is
unrealistic to expect of non-technical users). When different services are in competition with each
other, it is typically not in one service's interest to accept a competing service as identity
provider.

For these reasons, OpenID login is typically only accepted by small services; major online services
rarely act as a relying party in OpenID. The competitive dynamic between services makes it unlikely
that a user will ever be able to use a single identity provider across all services.

\subsection{Client certificates}\label{sec:clientcerts}

TLS~\cite{TLS} provides a mechanism for a client to authenticate itself to the server using a X.509
certificate. The server may specify the certificate authority from which it is willing to accept
certificates. By calculating a signature over the TLS key exchange messages, client certificate
authentication also provides protection against man-in-the-middle attacks (a signature generated on
one TLS connection cannot be reused on another TLS connection).

Client certificates are a good solution in situations where the physical identity of the user is
important: certificates might be issued by a government to its citizens (for filing taxes online),
or by a company to its employees (for accessing internal systems). For example, the government of
Estonia issues certificates on smart cards to its citizens through the national ID card scheme, and
the certificates are widely used for authentication by banks, utility companies and other
organisations in Estonia.~\cite{Parsovs14}

Client certificate authentication can be performed by hardware tokens, e.g. using the PKCS \#11
\emph{Cryptoki} API. This provides some protection against private keys being stolen by malware, but
it is inconvenient for users (see section~\ref{sec:hardware}).

Despite their advantages, client certificates are not widely used for consumer internet services.
Problems include:
\begin{itemize}
\item The user interface for installing and managing certificates is unfriendly in most web
browsers and operating systems. Services are not able to customize the look and feel of the signup
and authentication process.
\item It is not possible to be logged in to several accounts on one service at the same time. Even
logging out and switching to another account is awkward.
\item There is no good solution for authenticating on multiple devices, or even on multiple
sandboxed applications on the same device. Either private keys must be copied to each device
(increasing the risk of key theft), or each device must have its own keys (in which case, every time
the user acquires or disposes of a device, they must must manually update each of their services to
reflect the certificates for their current set of devices -- very laborious if they use hundreds of
services).
\item Certificates issued by an external CA typically contain personally identifying information,
such as a real name or email address. In order to enable pseudonymous usage, and to remove the
third-party trust dependency, a service provider must itself act as a CA.
\item Revocation of certificates (which requires CRLs or OCSP~\cite{OCSP}) is often not implemented
correctly. If OCSP servers are slow or unavailable, a service provider must either fail all logins
(unacceptable in practice), or skip revocation checking (allowing stolen certificates to be used).
\end{itemize}

\subsection{Private keys in secure hardware}\label{sec:hardware}

The \emph{Fast Identity Online} (FIDO) Alliance, an industry consortium, has drafted a specification
for new user authentication protocol.~\cite{FIDOOverview, FIDOSpec} Its basic mechanism is that the
user registers a public key when signing up to an online service. Whenever the user wants to log in,
the service provides a challenge, and the user signs the challenge using their private key to prove
their identity.

Unlike TLS client certificates, FIDO is an application-layer protocol. We believe that the basic
components of FIDO (public key authentication at the application layer) are promising. However, much
of the FIDO specification is dedicated to issues of using hardware modules (\emph{authenticators})
to store private keys and perform signing operations. On the other hand, the specification largely
ignores the issue of revoking a key that has been lost or stolen.

Another hardware-based authentication project is Pico~\cite{Stajano11}, which also proves ownership
of a secret key by signing a challenge. Pico addresses the risk of key theft by using $k$-out-of-$n$
secret sharing across several hardware modules carried by the user, such as glasses, belt, wallet,
and jewellery. This approach is theoretically interesting, but the possibility of a large-scale
deployment seems remote.

Whilst the security characteristics of hardware modules are appealing, the user experience is quite
inconvenient at present. For example, a USB-based module cannot be used on a mobile device that has
no USB port. If the cryptographic module is built into the device, the authentication capability is
not portable across devices.

A small number of high-value services (such as online banking) can afford to significantly
inconvenience users in the name of security, but for the majority of online services, any
authentication method that is less convenient than a password is unlikely to gain adoption, even if
it is significantly more secure. We seek a solution that can be deployed today, which does not
inconvenience users, and which does not require new hardware to be developed.

\section{Octokey protocol overview}

Octokey is a user authentication protocol designed to meet the goals stated in the introduction. It
aims to provide good security in a software-only configuration, with cryptographic hardware modules
being an optional extra. It strives to be a \emph{trust-free} protocol: it is completely
decentralised, there is no dependency on any identity provider, and the extent to which PKI
certificate authorities need to be trusted is minimised as far as possible.

The basic operation of Octokey is inspired by SSH public key authentication, which is widely used
for shell access to remote servers. When using Octokey for the first time, a user must generate a
RSA keypair.\footnote{We hope that the approach could be extended to support other public-key
cryptosystems such as ECC.} When the user signs up to a service, or migrates from another
authentication method to Octokey, the user submits their public key to the service. This is
analogous to creating a password when a user first signs up to a service.

Whenever the user wishes to log in, they must prove ownership of the private key. A client does this
by requesting a challenge from the server, signing it using the private key, and submitting the
signature to the server. The details of this protocol are given below.

We can think of Octokey as a machine-to-machine authentication protocol, and it needs to be preceded
by a human-to-machine authentication step: for example, a password or biometric information can be
used by the client device to unlock or decrypt the private key. However, since this password or
biometric information is not sent over the network, this concern is orthogonal to the
machine-to-machine authentication. Any improvements in biometric sensors, for example, do not
require any changes to the Octokey protocol or any support by service providers.

If a password is used to decrypt the user's private key, they need only remember a single password
per device (or perhaps per keypair), but not a separate password for every service where they have
an account. We believe that a single password is an acceptable user experience.

\subsection{Authentication requests}

Each user has a RSA keypair $(n, d, e)$ where $n$ is the modulus, $d$ is the private exponent and
$e$ is the public exponent. When signing up to a service, the user submits their public key
$(n, e)$ to the service. The service stores the public key as part of the user record.

To log in, the client first requests a challenge $c$ from the service via a HTTP endpoint. It then
calculates $m = H(c, u, r)$ where $u$ is the URL of the service endpoint and $r$ is the user's
registered username. $H$ encodes its arguments as a binary string and then applies the
\textsc{EMSA-PSS-Encode} operation (hashing and padding) as defined in PKCS\#1.~\cite{PKCS1} From
that, the RSA signature can be calculated as $s = m^d \mod n$.

The client then constructs the authentication request, which is an encoding of the tuple
$(s, c, u, r, n, e)$. The authentication request is sent to the server as part of a HTTP request
over TLS, and is handled at the application layer. The server can verify the authentication request
by checking that all of the following are true:
\begin{itemize}
\item $m^d \mod n$ is a valid PKCS\#1 signature of the values $(c, u, r)$, checked against the
public key $(n, e)$.
\item $c$ is a valid challenge issued by this service, has not been used for login before, and is
not older than some maximum age. This prevents replay attacks.
\item $u$ is a valid URL for this service. A login request for an unknown URL must be rejected.
This prevents phishing-like attacks, whereby an attacker creates an imitation website under a
similar-looking URL and tricks the user into logging in.
\item $r$ is a registered username for this service, and $(n, e)$ is a public key for that user.
\end{itemize}
If the authentication request is successfully verified, the user is logged in by the server in the
usual manner, for example by setting an appropriate session cookie.

\subsection{Protection against key theft}

Many users have multiple devices (e.g. laptop, smartphone, tablet, game console) on which they need
to be able to log in to their accounts with online services. As described above, the private key
$(n, d, e)$ would need to be copied to each of those devices. If any of those devices is lost or
compromised, and an attacker can break the human-to-machine authentication step (perhaps due to a
weak password on the private key), the attacker could gain access to all of a user's accounts.

To mitigate this risk, we ensure that the private exponent $d$ is never stored on any device.
Instead, we split it into key fragments that are distributed amongst the user's devices. We use the
\emph{mediated RSA} (mRSA) scheme proposed by Boneh et al.~\cite{Boneh01} which uses the fact that
$$s = m^d = m^{d_a + d_b} = m^{d_a} m^{d_b} \mod n$$ provided that $d = d_a + d_b \mod \phi(n)$.

If two devices $a$ and $b$ each have a key fragment $d_a$ and $d_b$ respectively, and those
fragments sum to the private exponent $d$, then we call those devices \emph{paired}. In order to
generate a valid signature, any two paired devices need to collaborate. If device $a$ wants to
generate an authentication request, it can send $(c, u, r)$ to device $b$, which may first ask the
user to confirm whether they wish to log in to URL $u$ using username $r$. If yes, $b$ calculates
$H(c, u, r)^{d_b} = m^{d_b}$ and returns that value to $a$. Then $a$ can calculate
$s = m^{d_a} m^{d_b}$ by using its own key fragment $d_a$ and the value returned by $b$, and thus
log in.

This scheme ensures that the user can revoke a device's login capability if it is lost, stolen or
compromised: every device that is paired with the lost device can be instructed to delete the key
fragment corresponding to the lost device. When the paired fragments are deleted, the key fragments
on the lost device become useless. Thus, even if the human-to-machine authentication is weak, not
all is lost: the user only needs to revoke the lost device's key fragments faster than an attacker
can break the human-to-machine authentication.

A key could also be split into more than two fragments, which would protect against the case where
an attacker manages to steal two or more devices. However, this comes at the cost of a more
cumbersome user experience, so we concentrate on the two-fragment case for now.

We expect that users have at least three devices (e.g.\ a laptop, a smartphone and a remote store --
see below), that each device is paired with each of the others, and that each pairing uses
different, randomly selected key fragments. If the user has $k$ devices, they can lose up to $k-2$
devices and still be able to log in to services without resorting to a backup.

\subsection{Remote key fragment store}

Authenticating by using paired physical devices (e.g.\ a laptop and a smartphone) yields a similar
user experience to current 2-factor authentication solutions, whereby the user must fetch the phone
from their pocket, launch the appropriate app, and perform some kind of handshake. This is possible,
but distinctly less convenient for users than typing a password, so it is not the simple user
experience we are looking for.

However, there is a simple solution within the mRSA framework: one of the user's `devices' may be a
remote service on the internet. This service stores key fragments that are paired with each of the
user's physical devices, and responds to signing requests by performing the modular exponentiation
using its key fragments. This allows a user to authenticate with services using only one physical
device -- the coordination with the remote key fragment store happens automatically behind the
scenes.

The key fragment store need only be partially trusted. It cannot authenticate as the user without
the cooperation of one of the user's physical devices. From the point of view of a service verifying
the signature on an authentication request, the key fragment store does not even exist (unlike a
federated login system, where the relying party must trust the identity provider). The user only
needs to trust the key fragment store to not collude with attackers who steal devices, and to delete
key fragments when the user requires key revocation.

TODO Prevent denial of service due to rate limiting or due to an attacker revoking a user's active
devices.

\section{Key distribution}

To provision a new device for a user (i.e. to pair it with their existing devices), we need to
generate a new key split $(d_{i+1,a}, d_{i+1,b})$ such that
$d_{i+1,a} + d_{i+1,b} = d_{i,a} + d_{i,b} \mod \phi(n)$, without reassembling the entire private
exponent $d$ on any single device (and thus risking it being stolen), and without knowing $\phi(n)$.

Say the user has two existing devices $a$ and $b$, storing $d_{i,a}$ and $d_{i,b}$ respectively. We
now want to introduce a new device $c$ and pair it with $a$. To this end, $c$ generates a uniformly
distributed random number $d_{i+1,c}$ with $0 < d_{i+1,c} < 2^n$, and another uniformly distributed
random number $g$ with $0 < g < d_{i+1,c}$.

By exchanging a sequence of messages, the devices can arrive in the desired state. The algorithm
will be described in the full paper.

\subsection{Key rotation}
\subsection{Multiple keypairs per user}
\subsection{Security levels}
\subsection{Device-to-device communication}
\subsection{Threats}
Undetected malware on a compromised machine can secretly log in to the user's services. If it's not
noticed, it can't be revoked. But that is true of any software-based solution; only dedicated
cryptographic hardware can prevent this, and we don't want to rely on hardware, as discussed in the
introduction.

\subsection{Channel binding}

Question: how to bind the Octokey signature to the TLS channel in a mobile app, where the TLS
connection will probably get torn town when switching to the Octokey app? Set up OBC first (before
login)?

Origin-Bound Certificates (OBC)~\cite{Dietz12} should be used to protect the session cookie from
MITM attackers.

Note that channel binding does not protect against all MITM attack scenarios. For example, in an
e-commerce setting, an attacker who has stolen the service's private key, or who has fraudulently
obtained a certificate from a CA trusted by the user's browser, can still impersonate the service.
If the user does not notice the impersonation, and enters their credit card number to make a
purchase, the attacker is able to steal the credit card number. In situations like these -- where
the user is providing confidential information to the service -- the PKI still needs to be trusted.

However, in situations where the flow of confidential information is from the service to the
authenticated user, channel binding is an effective additional protection against MITM attacker
trying to steal that information.

Another attack vector is via JavaScript injected into the user's browser by a MITM attacker. Any
code that executes within the context of a document in the browser has full access to the contents
of that document. A page that is served to an authenticated user may download JavaScript code on
separate TLS connections, for example from a content delivery network (CDN). If a MITM attacker is
able to impersonate the CDN, they can inject arbitrary JavaScript into the browser.

To prevent this, the page should not directly execute JavaScript that is downloaded. Instead, it
should fetch JavaScript from the server using XMLHttpRequest over an authenticated TLS connection. 

\section{Inter-device communication}\label{sec:interdevice}

Octokey assumes that the user has multiple internet-connected devices: two at minimum (one physical
device and one remote key fragment store), but likely several more (desktop computer, laptop,
smartphone, tablet, game console, television, automobile, etc).

We assume that users want to be able to access all of their services on any one of their physical
devices (devices with partial access to the user's services are discussed in
section~\ref{sec:delegation}). Thus, each physical device should be paired at minimum with the
remote key fragment store. Some physical devices may also be paired with each other, to provide
redundancy in case of a failure of the remote store (discussed in section~\ref{sec:threat}).

The system should make it easy to enroll new devices to the user's set of trusted devices, and to
revoke any devices which are lost or no longer in use. It may also provide convenience features,
such as synchronizing the list of usernames for all of a user's services, so that they can be
autofilled in login forms. In order to do this, Octokey needs a mechanism for the user's devices to
synchronize with each other.

\subsection{Communication protocol}\label{sec:channels}

First, we need to be able to establish a secure point-to-point communication channel between any two
of the user's devices (treating the remote store as one of the devices). The channel must provide
secrecy and integrity, so that an attacker cannot eavesdrop or modify communcation between devices.
It must also prevent spoofing, so that an attacker cannot impersonate a device or MITM a channel.

TLS~\cite{TLS} meets these requirements. Each device generates its own private key and self-signed
certificate, which are used only for inter-device communication and are independent from the mRSA
key fragments. One device then connects to another and performs a TLS handshake, using the devices'
certificates as client and server certificates for mutual authentication.

This approach does not require any PKI or certificate authorities: public key fingerprints are
exchanged out-of-band when devices are first discovered (see section~\ref{sec:newdevice}), so each
device can check whether its peer is the device that it claims to be. It is not important which
device acts as client or server in this channel, as the relationship between the devices is
symmetric.

However, there is a practical problem: it can be difficult to establish a peer-to-peer TCP
connection between two arbitrary devices, as they may be on different networks, behind firewalls,
and they may often be offline -- and rarely online at the same time. Thus, rather than performing
TLS over TCP, we propose using an alternative transport layer in the form of a store-and-forward
service.

The store-and-forward service is a web service that can be reached from all devices (it may be
co-located with the remote key fragment store). It accepts messages from one device and relays them
to another device: if the recipient is online, the messages are forwarded immediately, otherwise
they are stored until the recipient is next online. The messages are records of the TLS Record
Protocol~\cite{TLS}, so two devices can perform a TLS handshake and establish a secure channel by
sending each other messages via the store-and-forward service. The service needs to be highly
available, but it does not need to be trusted for security purposes.

If devices are only occasionally online, the round-trip-times across this ``network connection'' may
be measured in days, so timeouts may need to be adjusted accordingly. However, once this channel has
been set up, it can remain open indefinitely (with occasional renegotiation), even across device
reboots.

Besides this store-and-forward service, devices should support alternative transport mechanisms,
such as direct TCP connections on a local network, Bluetooth or NFC. These may only occasionally be
used, but they are good fallbacks so that the store-and-forward service does not become a single
point of failure. The same TLS record protocol can be used over those transport layers.

\subsection{Data synchronization}\label{sec:devicesync}

These point-to-point channels are used for exchanging mRSA signing requests and responses between
devices. Moreover, we can use them for replicating information that we want stored on all devices:
each of the user's physical devices maintains a local database containing public key fingerprints of
all of the user's other devices (for mutual authentication of channels), configuration (e.g. URL of
the key fragment server), information on which devices are paired (for key revocation purposes), a
list of all the services they have logged in to (for username autofill), and so on.

Each of the user's physical devices should be able to read and modify this database, and any changes
should be asynchronously propagated to the other devices. To this end, we can use the point-to-point
channels of section~\ref{sec:channels} to create a mesh of connections among the user's devices
(even between those which are not paired). It is not necessary to have a communication channel
between every pair of devices, but a fairly densely connected graph of channels is desirable, so
that any message from one device reaches the other devices fairly quickly, and so that the
communication is resilient to device and channel failures.

Within this mesh network, we can use a \emph{gossip protocol} (also known as \emph{epidemic
protocol}) \cite{Demers89} for disseminating database changes. \emph{Version vectors}
\cite{ParkerJr83} can be used to efficiently detect data differences that need to be synchronized,
and \emph{CRDTs}~\cite{Shapiro11} can automatically resolve any conflicts between concurrent
database updates, without requiring user interaction. Every database change must be signed with the
device key of the device on which it originated, so that any changes made on a stolen device can be
rejected.

Although the key fragment store is one of the user's `devices', it does not need a copy of this
database, and it must not be allowed to make database updates. Only physical devices may
authenticate with services on the user's behalf, and only physical devices may enroll additional
trusted devices. This makes it safe for the key fragment store to be operated by a third party. To
protect the user's privacy, if a copy of the database is stored on the key fragment store for backup
purposes, it must be encrypted such that only the user's physical devices can restore it.

\subsection{Bootstrapping}\label{sec:bootstrap}

When someone first starts using Octokey, they go through the following steps:

\begin{enumerate}
\item The user installs the Octokey application on a physical device, and selects ``create new
Octokey''.
\item The application generates the user's RSA keypair $(n, d, e)$, as well as the keypair and
self-signed certificate for that device.
\item The application contacts a key fragment store of the user's choice (a default may be
provided) over TLS, using its device certificate for client authentication, and verifying the server
using well-known certificate authorities (or a pinned public key fingerprint embedded in the
application).
\item The application splits the private exponent $d = d_1 + d_2$ such that $d_1$ is a uniformly
distributed random integer with $0 < d_1 < d$. The device stores $d_1$ locally, sends $d_2$ to the
key fragment store.
\item The user has the option of printing off the private key, as a last-resort backup in case all
of their devices are lost (see section~\ref{sec:threat}). Then the application deletes $d$.
\end{enumerate}

\subsection{Enrolling a new device}\label{sec:newdevice}

When the user already has set up Octokey on one or more devices, and wishes to enroll another
physical device (i.e. pair a new device with their existing devices), they proceed as follows:

\begin{enumerate}
\item The user installs the Octokey application on the new device, and selects ``set up this device
to use your existing Octokey''.
\item The application generates that device's keypair and self-signed certificate. It then registers
itself at the store-and-forward service, using its device certificate for client authentication. The
store-and-forward service allocates a URL at which this device can be reached, using the device's
public key fingerprint as recipient identifier.
\item The application on the device displays a 2D barcode on screen, containing an indicator that
this is an enrollment request, the store-and-forward URL at which it can be reached, the device's
public key fingerprint, and optionally information about other transport channels (e.g. Bluetooth)
where it can be reached. The barcode must not contain any sensitive data, since it may be visible to
eavesdroppers (see section~\ref{sec:barcode-intercept}).
\item The user scans this barcode using the camera of their existing Octokey device. The application
connects to the URL in the barcode, negotiates an end-to-end TLS connection and checks that the
remote key fingerprint matches the one in the barcode.
\item The existing Octokey device prompts the user whether they want to trust the device whose
barcode they just scanned. If the user is already using multiple devices, the applications on those
devices may additionally prompt the user.
\item When sufficient user approvals have been obtained, the device that scanned the barcode
registers the new device (including its public key fingerprint) in the database. This change is
replicated to other devices, which may connect to the new device (adding it to the mesh network).
\item The new device initiates the key re-pairing protocol (section~\ref{sec:pairing}) in order to
obtain key fragments.
\end{enumerate}

This interaction flow assumes that the first physical device on which the user sets up their private
key has a camera. This is plausible, since many laptops and smartphones come with built-in cameras
today. In the case where no camera is available, the user can type in the contents of the barcode in
textual form.

Devices which are enrolled later do not require a camera, only a screen. For example, a game console
or an automobile on-board computer may not have a camera, but they can nevertheless be enrolled if
they have an internet connection.

Bluetooth or NFC are less suitable for initiating device enrollment, because they carry a higher
risk of accidentally enrolling the wrong device, including an attacker's device (see also
section~\ref{sec:barcode-phishing}). However, once a device is enrolled, they can be used safely,
because the devices know each others' fingerprints and can thus mutually authenticate.

\subsection{Key re-pairing}\label{sec:pairing}

Each mRSA pairing of devices requires the private exponent $d$ to be split in different ways:
$d = d_1 + d_2 = d_3 + d_4 = \dots$ with $d_1 \neq d_2 \neq d_3 \neq d_4$ (all values are drawn from
a uniform random distribution, so there is a very small probability that they might be equal).

Say there are two existing devices: device $a$ stores key fragment $d_1$ and device $b$ stores key
fragment $d_2$. When a new device $c$ is enrolled, and we want to pair it with device $a$, we need
to generate new key fragments $d_3$ (stored on device $c$) and $d_4$ (stored on device $a$). We can
do this as follows, without reassembling the entire private exponent $d$ on any single device (and
thus risking it being stolen):

\begin{enumerate}
\item Device $c$ chooses a random integer $d_3$ from the uniform range $0 < d_3 < n$ (where $n$ is
the RSA modulus).
\item Device $c$ chooses a second random integer $k$ from the uniform range $0 < k < d_3$.
\item Device $c$ sends $k$ to device $a$, and sends $(d_3 - k)$ to device $b$.
\item Device $b$, where $d_2$ is stored, calculates $(d_2 - (d_3 - k))$ and sends it to device $a$.
This number may be negative.
\item Device $a$, where $d_1$ is stored, has now received $k$ and $(d_2 - (d_3 - k))$. It uses these
to calculate $$d_4 = d_1 + (d_2 - (d_3 - k)) - k = d_1 + d_2 - d_3.$$
\item If $d_4$ is positive, the algorithm was successful: device $a$ stores $d_4$ and we're done. If
$d_4$ is negative, we start again from step 1, choosing new random numbers.
\end{enumerate}

The algorithm ensures that $d_3 + d_4 = d$, so $d_4$ is negative iff $d_3 > d$. However, since no
device knows $d$, we cannot simply generate $d_3$ to be less than $d$, but we must retry with new
random numbers until a suitable $d_3$ is found.

A malicious device could use this algorithm to determine $d$ by binary search. However, there are
also many other ways for a malicious device to reconstruct $d$, for example by pairing with a
colluding device. Thus, we must trust the software on the user's physical devices, and use the fact
that a device can only initiate the pairing process after it has been authorized by the user. We
also prohibit the remote key fragment store from initiating the re-pairing process.

\subsection{Number of retries}\label{sec:retries}

The key re-pairing algorithm in section~\ref{sec:pairing} requires a potentially unbounded
number of retries. To avoid excessive waiting times for the user, we want to keep the number of
retries fairly small.

The number of retries $R$ is a geometrically distributed random variable with an expected value of
$\frac{n}{d}-1$. This means that if the private exponent $d$ is significantly smaller than the
modulus $n$, the number of retries can be high.

We therefore suggest that when the private key is first generated, it must be greater than some
threshold, e.g. $d \ge \frac{n}{16}$. With this restriction, a worst-case key would require 15
retries on average, and would have an approximately 0.1\% risk of requiring more than 100 retries.
Most keys would require significantly fewer retries.

Although this restriction would exclude 1/16th of the possible keyspace, with a 2048-bit key this is
equivalent to reducing the key size to $\log_2(2^{2048} - 2^{2044}) \approx 2047.9$ bits. It seems
unlikely that a loss of 0.1 bits would significantly harm the strength of the algorithm, but in any
case the cryptanalytic consequences of such a key restriction need to be studied carefully.

\subsection{Delegated login by mobile device}\label{sec:delegation}

A user should be able to authenticate with services on a semi-trusted device, for example a game
console at a friend's home, without having to generate key fragments for that device (and thus
giving the device full access to all of the user's accounts). To this end, we use the fact that a
signed Octokey authentication request is like a one-time password: it can be given to another device
in order to authorize one-off access to one particular service.

By logging in, the semi-trusted device exchanges the signed authentication request for a session
identifier (e.g. a cookie), which is destroyed when the user logs out, or after a timeout. In cases
where the user wants to give a device long-lived access to one service, the device can obtain an
OAuth~\cite{OAuth} token from the service. This is standard use of OAuth and does not require any
special support from Octokey.

The question is then how to get the signed authentication request onto the device that needs it. We
propose the following flow:

\begin{enumerate}
\item Steps 1 to 4 are identical to the flow in section~\ref{sec:newdevice}, except that the user
selects ``log in with your Octokey on another device'' instead of ``set up this device to use your
existing Octokey''. The 2D barcode indicates that this is a signing request, not an enrollment
request.
\item The device requesting login requests a challenge $c$ from the service.
\item Using the secure channel that has been set up between the devices, the device requesting login
sends $c$ and the login URL $u$ to the device that scanned the barcode.
\item The latter device prompts the user whether they want to allow the device whose barcode they
just scanned to log in to their account at URL $u$. If yes, they choose the username $r$ to use
(which may be autofilled from their database of services).
\item If the user approves, the device goes through the usual mRSA signing flow to calculate a
signature $s$, and sends the completed authentication request $(s, c, u, r, n, e)$ to the device
requesting authentication, using the secure channel. That device sends the authentication request to
the service to log in.
\end{enumerate}

It is important that the user checks whether the authentication request is for the URL they were
expecting, because a device could potentially request login for any service. As a usability
improvement, user could be shown a screenshot of the website at the requested URL, not just the URL
itself; this would make it visually obvious if the user's banking login is being requested when they
thought they were logging into an online game. (As before, creating a phishing website with
imitation design would not work --- in order to obtain a valid authentication request for the user's
bank, they must supply the real bank's URL.)

This process for delegated authentication is simple and fast, has fairly small risk of user error,
and can be implemented with any smartphone. It is significantly more pleasant to use than a password
manager, does not require typing anything, and is more secure.

\section{Threat model and trade-offs}\label{sec:threat}

In this section, we discuss the assumptions we have made with regard to attackers' capabilities,
some aspects which require careful implementation, and some areas where different concerns need to
be traded off against each other.

\subsection{Malware and untrusted software}

Most authentication schemes in a consumer context are based on the principle that once a user has
authenticated, they have unlimited access to their account: using different credentials for
different actions is considered too inconvenient.\footnote{An exception are some online banking
websites, which require explicit authorization of each payment action, using a hardware token.}

Consequently, once a user has authenticated with a service on a device, the software on that device
can in principle do anything it wants to the user account. That is true no matter which
authentication method is used: if there is malware which can read files, log keystrokes and steal
session cookies, or if an attacker can obtain remote root access to a device, any services in use on
that device can be compromised.

With Octokey, there is the additional risk of the device's private key (for inter-device
communication) and mRSA key fragment being stolen. If the device has a built-in hardware security
module, the actual keys are protected, but malware can nevertheless sign arbitrary authentication
requests, because it is almost indistinguishable from legitimate user software. Thus, if malware is
detected on a device, its keys must immediately be revoked.

In sandboxed environments (e.g. websites in web browsers, apps on some mobile OSes), mutually
untrusting applications are somewhat protected from each other. In this case, there needs to be an
explicit way of passing authentication requests between the Octokey application (which manages the
keys) and the websites and apps that require authentication. This mechanism must be implemented
carefully, ensuring that a website or application can only obtain a signed authentication request
for its own URL, and not some other one.

\subsection{Physical device theft and loss}

We assume that the private key material on a device is protected by a human-to-machine
authentication step, e.g. encrypted with a password. This is not much use against malware, but it
does help against an attacker who physically steals a device (or a backup of a device's filesystem).
In section~\ref{sec:ratelimit} we discussed a technique for strengthening this step against brute
force and dictionary attacks.

If the user has enrolled another device, they can use that device to revoke the stolen device. If
the revocation happens before the human-to-machine authentication is broken, no harm is done. If the
device is not revoked in time, the attacker can sign arbitrary authentication requests, and steal
the key by enrolling another device and pairing it with the stolen device. The risk of key theft can
be mitigated by prompting the user on several devices before enrolling another device (provided that
the user has previously enrolled several devices).

If an attacker steals two devices that are paired with each other, the risk of key theft is much
greater, because the human-to-machine authentication is the only remaining protection: deleting key
fragments from the remote store is not sufficient to prevent the attacker from recovering the key.
This means there is a trade-off between security and reliability: pairing physical devices with each
other provides redundancy in case the remote key fragment store is unavailable, but it increases the
risk of the private key being stolen.

Another risk is that the user may lose access to their services: perhaps an attacker who gains
control of a device uses it to revoke the user's legitimate devices (thus preventing the user from
stopping the attacker); perhaps the user has only enrolled one or two physical devices, and they are
stolen simultaneously; perhaps the user's house burns down and all of their devices are lost. To
allow recovery from such situations, we recommend that users print off their private key and store
it in a safe place.

\subsection{Key fragment store}

The key fragment store is likely to be an attack target. If it is compromised, all users need to
re-pair with a new key fragment store (otherwise the ability to revoke keys is lost). For this
reason, the protocol should allow forced re-pairing within a deadline.

Denial of service attacks on the key fragment store are also likely.

Pair all physical devices with each other? Advantage: redundancy in case remote store has an outage.
Disadvantage: if an attacker steals two paired devices (e.g. taking both your laptop and your
smartphone in a robbery), mRSA revocation is not possible, so the human-to-machine authentication
step is the only remaining protection of the private key.

Does revocation still work if the user has only one device (besides the remote service)?
Need to assume everyone has two physical devices?

Does the remote key fragment store need to authenticate signing requests?

The remote key fragment store sees plaintext URLs and challenges, so it could track a user's
activity. Is this ok? Advantage: allows more granular limiting of abuse, e.g. rate limiting login
attempts on one website without affecting other websites.

The user should print off their entire private key, and store it in a safe place (e.g. bank vault).
This gives them a recovery method in case all of their devices are simultaneously destroyed (house
burns down).

Question: how does Octokey know that a certain website URL is the same service as a certain iOS
bundle ID, which is the same service as an Android app signed with a certain public key?

\subsection{2D barcode interception}\label{sec:barcode-intercept}

In the flow for enrolling a new device (section~\ref{sec:newdevice}), and with delegated login
(section~\ref{sec:delegation}), we proposed displaying a 2D barcode on the screen of one device, and
scanning it with the camera of another. We cannot assume confidentiality of these barcodes: an
attacker may snoop it electromagnetically~\cite{Kuhn05}, or simply point a camera at the victim's
screen.

An attacker can thus connect to the URL in the barcode, and enroll the new device to the attacker's
account, instead of the user's own account as intended. With delegated login, the attacker can get
the user to log in with the attacker's key rather than the user's own key. This does not compromise
the user's key, but if the user logs in to the wrong account and doesn't realize it, they may
inadvertently disclose sensitive information to the attacker.

It is not clear whether this would be a problem in practice. If it is, a mutual authentication step
could be added to the flows for enrolling a new device and for delegated authentication (for
example, verifying a 2D barcode in the other direction). However, this makes the process more
complicated for users, especially when trying to delegate authentication to a device which has no
camera, so the additional verification step should probably be optional.

\subsection{2D barcode phishing}\label{sec:barcode-phishing}

When enrolling a new device or performing delegated authentication, we assume that an attacker
cannot trick a user into scanning a different barcode from the one they intended --- i.e. we assume
that the visual channel between the barcode-displaying and the barcode-scanning device provides
integrity (but not confidentiality). If the attacker is able to manipulate what is displayed on
screen, or somehow insert themselves into that channel, the user has bigger problems.

However, a real risk is that a malicious website or app displays a barcode that originates from an
attacker-controlled device, and tricks the user into scanning that barcode and granting the attacker
unwanted access. In this scenario we must rely on well-written warning messages and user education
to ensure the user really understands what they are doing.

For users who have already set up multiple physical devices, as an additional safeguard against
accidentally pairing with an attacker-controlled device, we can require that enrolling a new device
requires user approval on a quorum (e.g. majority) of existing devices. This does not require
additional cryptographic algorithms, but can simply be implemented as a policy in the Octokey
software.

\subsection{Key rotation}
\subsection{Multiple keypairs per user}
\subsection{Multi-way key splitting}
\subsection{Security levels}

% Sync login history across devices, and compare to login history reported by server, to detect
% use of a stolen key


\subsection{Denial of service}

Prevent denial of service due to rate limiting or due to an attacker revoking a user's active
devices.

In the worst case, if an attacker manages to get the entire private key (e.g. by stealing or
compromising two devices that are paired, and breaking the human-to-machine authentication step),
the user's last resort is to generate a new key and update their accounts on all services, adding
the new public key and removing the old one. Unfortunately, the same key-swapping can be performed
by the attacker to lock out the legitimate user.

Therefore, perhaps changing a user's public key on a service should require an additional hurdle,
e.g. using a recovery key that is only stored on paper but not electronically? Or a key that is
split 3 ways? But that would make key rotation difficult.

\section{Related work}\label{sec:otherapproaches}

In this section we compare Octokey to other cryptographic authentication protocols. Since Octokey
can be used with various human-to-machine authentication methods (password, biometrics, etc.), we
concentrate on the machine-to-machine aspect, and refer the reader to the literature for a survey of
human-to-machine authentication methods~\cite{Bonneau12}.

\subsection{Certivox M-Pin}

Certivox M-Pin~\cite{Scott14} is a cryptographic authentication protocol based on elliptic curves.
It has some similarities to Octokey: it reconstructs a client secret from a stored token and a
4-digit PIN (similar to the human-to-machine authentication step in Octokey), and the client proves
ownership of the secret to the server without revealing the secret.

However, each M-Pin client and server requires a secret that is issued to it by a \emph{trusted
authority} (TA). This is reasonable in the context of employees authenticating with their company's
systems, because clients, servers and TA are all controlled by the same organization. In a consumer
internet context, each service would need to set up its own TA, using hardware security modules to
protect server secrets, or rely on a TA service hosted by Certivox. This makes server deployment
more complicated and costly than Octokey, which requires only a simple RSA signature verification,
and no secrets on the server.

A service that accepts M-Pin login can compute the user's PIN; thus, a rogue service that colludes
with attackers who steal devices can reconstruct client secrets.\footnote{Unless the user has a
different PIN for each service, which takes us back to square one.} Moreover, M-Pin does not support
instantaneous key revocation, only time-based key expiry (``time permits'').

\subsection{2D barcode authentication}

Several previous proposals use 2D barcodes as part of a user authentication flow: for example, tiqr
(based on OCRA~\cite{OCRA}), eKaay~\cite{Borchert11} and SQRL~\cite{Gibson13} all use 2D barcodes.
A mobile device obtains a challenge from a service by scanning a barcode, signs that challenge using
a secret stored on the device, and returns the signed challenge to the service. The secret may be a
symmetric secret that is shared between client and server~\cite{OCRA} or an asymetric private
key~\cite{Gibson13}. These proposals have similarities to Octokey's delegated authentication
(section~\ref{sec:delegation}).

However, these proposals do not have good answers to the issues of key loss, key revocation and
distribution of keys to new devices. tiqr and eKaay appear to not have any revocation mechanism;
SQRL's revocation requires the user to retrieve an ``identity unlock key'' that is stored on paper,
and use it to individually reset their key on every service they use (similar to the master key
discussed in section~\ref{sec:recovery}). To authenticate on multiple devices, the user must either
copy the entire secret key to every device, or they must go through the barcode-scanning procedure
every time they want to authenticate. There is no way of revoking only one device, no easy way of
enrolling new devices, and no provision for key rotation.

Finally, these proposals do not have a rate limiting mechanism like in section~\ref{sec:ratelimit},
and rely solely on a slow PBKDF such as Scrypt~\cite{Percival09} to slow down offline attacks on an
encrypted secret. Depending on the strength of the user's password, this may not be sufficient to
stop a determined attacker who has stolen an encrypted secret.

\subsection{Other approaches}

Windows CardSpace~\cite{Chappell06} was a user interface and abstraction layer on top of X.509
certificates and other authentication mechanisms, but it was not successful in the marketplace.
U-Prove~\cite{Paquin13} is an ambitious project to build a superset of the X.509 PKI using
zero-knowledge proofs. It significantly more complicated and broader in scope than Octokey, since it
deals with much more than just authentication, such as the selective disclosure of certificate
attributes for privacy reasons. We fear that this complexity will prove an obstacle to its practical
adoption.

\section{Conclusion}

In this paper, we have introduced the design of Octokey, an authentication system intended for
consumer internet use cases. We argue that it provides better security than all commonly-used
alternatives (passwords, password managers, and federated authentication methods), and provides a
better user experience and better flexibility than more secure alternatives (hardware-based
authenticators, smart cards).

Octokey builds on mature, well-understood components such as TLS and RSA, and has similarities to
the familiar SSH public key authentication protocol. By keeping the new algorithms to a minimum, and
by using familiar and mature components where possible, we hope that the security characteristics of
the system will be easily understood by users and implementers.

It is quite simple for service owners to start accepting Octokey as an authentication mechanism,
since libraries for verifying PKCS\#1 signatures are available for all common programming languages.
Octokey can also be used alongside passwords or other authentication mechanisms, allowing it to be
adopted gradually. It does not require any special hardware or any changes to the way online
services are deployed.

Most of the complexity of Octokey lies in the implementation of the client applications and the
mediator service. We plan to develop these using an entirely open source model, and to make them
available as free software for anyone to run and modify. The protocols should be open standards, and
anybody should be free to create alternative implementations if they wish.

Octokey strikes a pragmatic compromise between security and convenience, and strives for a very
simple user experience with minimal risk of user error. We hope that it can be the viable
alternative to passwords that has eluded us for so long. We invite others to join the project, give
feedback, spread the word, and start working towards a real implementation.


{\footnotesize
    \bibliographystyle{plain}
    \bibliography{references}{}
}

\end{document}
