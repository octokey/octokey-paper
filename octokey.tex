\documentclass[letterpaper,twocolumn,10pt]{article}
\usepackage{usenix,epsfig}
\usepackage{cite}
\usepackage{hyperref}
\usepackage{amsmath}
\newcommand*{\concat}{\mathbin{\|}}
\hyphenation{time-stamp}
\sloppy

\begin{document}
\date{} % No date
\title{\Large \bf Octokey: Public key authentication for the web}
\author{{\rm Martin Kleppmann} \and {\rm Conrad Irwin}}
\maketitle

\subsection*{Abstract}

We propose Octokey, a cryptographic protocol for authenticating users, designed as a viable
replacement for password authentication in consumer internet use cases. When a user signs up to a
service, they provide a public key. Clients subsequently authenticate by presenting a challenge
signed with the user's private key, similar to SSH public key authentication. We also describe
protocols which allow users to log in to their accounts on multiple devices, enroll new devices,
and revoke keys on devices that are lost or stolen.
\end{abstract}

\bigskip

\emph{Note: We, the authors, have a background in developing commercial web applications. In the
field of crypography, we are merely keen amateurs. We are aware that amateur cryptography is prone
to catastrophic flaws, and we fully expect such flaws to be found in the ideas we are proposing.
However, we have experience designing real-world products with good user experiences, a perspective
that we feel is often lacking in security research. We hope that this paper can stimulate a
discussion that brings together the perspectives from research and industry, and we would like to
solicit constructive feedback from all sides.}

\section{Introduction}

Despite their many well-known flaws, passwords are still by far the most commonly used
authentication system on the web.~\cite{Bonneau12} However, as users sign up for ever more services,
their use is becoming increasingly unsustainable: demanding that users choose passwords with
sufficient entropy, use a different password for every site, not write them down and even rotate
passwords periodically is unrealistic for most internet services.

With phishing attacks and leaks of password databases unfortunately commonplace, we seek a better
solution. In this paper we focus on authentication in a consumer internet context, such as social
media or e-commerce websites and mobile apps. For a majority of such use cases, an authentication
system with the following properties is desirable:

\begin{enumerate}
\item The system should minimize exposure to human error (such as falling for a phishing attack).
Whilst human error is unavoidable, especially amongst non-technical users, the system should be
designed so as to minimize the impact in the case of error.
\item In case a device is lost or stolen, the user should be able to easily revoke any credentials
stored on the device, to minimise the chances that an attacker can use a stolen device to gain
access to other systems.
\item It is not necessary, and often undesirable, for the system to authenticate the user as a
physical person. Many websites allow pseudonymous signup, which is desirable for reasons of privacy
or freedom of speech. The purpose of the authentication system is thus only to verify that the
current user is the same person as the one who originally signed up under a particular username; no
authentication is performed at signup.
\item The system should not depend on any trusted third party, to avoid risks arising from failure
of the third party (security compromise, going out of business, change of policies, etc).
\item The system should be easy for end-users to install and use, and easy for website owners to
deploy. Since user signup and user activity rates are important business metrics for many online
services, the system should make signup and login extremely easy and enjoyable for users.
\item Users should have the freedom to use the system on a variety of devices, including shared or
public computers that are only partially trusted, whilst minimizing their exposure to attackers.
For example, a user may choose log in to `unimportant' services (e.g. casual online games) on a
shared or public computer, knowing that their account on that service may be compromised; however,
they limit their use of `important' accounts (e.g. online banking) to trusted devices. Since there
is no generally agreed boundary between `important' and `unimportant' websites, the same
authentication system should be able to support both use cases.
\end{enumerate}

In the next section we discuss various authentication systems that are used on the web, and explain
why none of them meet these requirements. We then introduce Octokey, an authentication protocol
which addresses these issues. We believe that Octokey could be a viable solution to online
authentication in the near future, and we seek the information security community's feedback to
refine this proposal.

\section{Existing web authentication systems}
\subsection{Passwords}

Passwords have many of the desirable properties mentioned above: no dependence on any third party,
pseudonymity, simplicity of implementation, familiarity and cross-device compatibility. However,
remembering a large number of passwords is a big burden on users, and they are highly susceptible to
human error: use of weak passwords, reuse of the same password across different services, phishing
attacks and leaks of unhashed or weakly hashed password databases are sadly common.

Password managers built into web browsers, or external password manager products such as 1Password
and LastPass, make it feasible for users to maintain a strong, unique password for each service they
use. However, even when a unique password is used, an attacker who succeeds in stealing a password
(e.g. by phishing, malware, keylogging, man-in-the-middle attack or eavesdropping when it is
accidentally sent over an unencrypted connection) has access to that user account until the password
is changed, which is often a long time.

Password managers also have grave exposure to human error. For example, consider a user who wishes
to log in to an `unimportant' website on an untrusted computer. It is very tempting for the user to
type the master passphrase of their password manager into the untrusted computer, in order to
decrypt the password database. This has the effect of exposing the user's entire password database
to an attacker running malware on the untrusted computer.

\subsection{One-time passwords and two-factor authentication}

One-time passwords (OTPs) offer a solution for users who want to log in to their accounts on
untrusted or partially trusted devices. The user must download a list of one-time passwords on a
trusted device, or register a cryptographic device that generates a pseudorandom sequence of OTPs,
or register an email address or phone number to which OTPs can be sent on demand. Some services
allow use of OTPs as sole authentication mechanism, while others use them in conjunction with a
regular password (two-factor authentication).

The advantage of OTPs is that the exposure to attacks is limited to a single service, and ends when
the user clicks the logout button (assuming the session is correctly invalidated on the server, and
assuming that the attacker does not have a means of extending their privileges, e.g. by granting
themselves OAuth access to the account).

However, OTPs are so inconvenient to use that very few websites are willing to adopt them as their
only or primary authentication mechanism, and only the most security-conscious users are willing to
use them. Moreover, the common ``remember me on this device'' feature weakens two-factor
authentication (attackers just need to steal the ``remember me'' cookie as well as keylogging the
password).

OTPs sent on demand via email or SMS also have the effect of `outsourcing' the authentication to the
email provider or the mobile network provider, respectively: any attacker who can access the email
provider or SMS messages can also gain access to any service using on-demand OTPs. This essentially
makes them federated login schemes (see next section).

\subsection{Federated identity}

OpenID~\cite{OpenID} is probably the best-known attempt to remove the need for a separate password
on every service. Some identity providers, such as Google Federated Login~\cite{GoogleOpenID} and
Facebook Connect, combine the ideas of OpenID and OAuth~\cite{OAuth}. Mozilla Persona
(BrowserID)~\cite{Persona, BrowserID} uses a different protocol, but is similar in the way it
delegates authentication to the user's email provider (or a fallback service provided by Mozilla).

Such federated authentication protocols do not solve the fundamental problem of authentication: they
only delegate it to the identity provider, who must then use some other authentication method (most
commonly a password, possibly in conjunction with a hardware token). This means that all the
problems discussed above apply to the identity provider, with an additional privilege escalation
problem: if an attacker gains access to the user's account with the identity provider, they can
access any account associated with that identity. Like with a password manager, a single user error
can lead to all of a user's accounts being compromised.

Moreover, the relying party needs to trust the identity provider to correctly authenticate the user.
If the identity provider is compromised, the relying party has no way of detecting unauthorized
logins from that identity provider. If the identity provider experiences an outage or goes out of
business, users of that provider lose their ability to log in (unless they had previously set up
delegation of their identity URL, which is unrealistic to expect of non-technical users). When
different services are in competition with each other, it is typically not in one service's interest
to accept a competing service as identity provider.

For these reasons, OpenID login is typically only accepted by small services; major online services
rarely act as a relying party in OpenID. The competitive dynamic between services makes it unlikely
that a user will ever be able to use a single identity provider across all services.

\subsection{Client certificates}

TLS~\cite{TLS} provides a mechanism for a client to authenticate itself to the server using a X.509
certificate. The server may specify the certificate authority from which it is willing to accept
certificates. By calculating a signature over the TLS key exchange messages, client certificate
authentication also provides protection against man-in-the-middle attacks (a signature generated on
one TLS connection cannot be reused on another TLS connection).

Client certificates are a good solution in situations where the physical identity of the user is
important: certificates might be issued by a government to its citizens (for filing taxes online),
or by a company to its employees (for accessing internal systems). For example, the government of
Estonia issues certificates on smart cards to its citizens through the national ID card scheme, and
the certificates are widely used for authentication by banks, utility companies and other
organisations in Estonia.~\cite{Parsovs14}

Client certificate authentication can be performed by hardware tokens, e.g. using the PKCS \#11
\emph{Cryptoki} API. This provides some protection against private keys being stolen by malware.

Despite their advantages, client certificates are not widely used for consumer internet services.
Problems include:

\begin{itemize}
\item The user experience of installing and managing certificates is unfriendly in most web
browsers.
\item Web browsers do not offer a consistent mechanism for logging out or switching to another
account after logging in with a client certificate.
\item A website provider must decide which CAs to trust, or it must itself act as a CA in order to
avoid a trust dependency on a third party.
\item Pseudonymous usage of services is difficult if the certificate is issued by an external CA.
\item The certificate (including identifying personal details such as the user's real name) is sent
unencrypted during the TLS handshake, making it visible to passive network attackers.\footnote{This
can be mitigated by performing the authentication handshake during a TLS renegotiation, but that has
a negative performance impact, and makes the server configuration more complicated.}
\item Certificate validation can be difficult to deploy on the server, as TLS termination is
typically performed on a different server from the application server running business logic.
\item Certificate validation can be computationally expensive, creating a risk of denial of
service.~\cite{Parsovs14}
\item Revocation of certificates (which requires CRLs or OCSP~\cite{OCSP}) is easy to get wrong, and
a slowdown or outage of OCSP servers can impact a service's availability (creating another
third-party dependency).
\end{itemize}

\subsection{Private keys in secure hardware}

The \emph{Fast Identity Online} (FIDO) Alliance, an industry consortium, has drafted a specification
for new user authentication protocol.~\cite{FIDOOverview, FIDOSpec} Its basic mechanism is that the
user registers a public key when signing up to an online service. Whenever the user wants to log in,
the service provides a challenge, and the user signs the challenge using their private key to prove
their identity.

Unlike TLS client certificates, FIDO is an application-layer protocol. We believe that the basic
components of FIDO (public key authentication at the application layer) are promising. However, much
of the FIDO specification is dedicated to issues of using hardware modules (\emph{authenticators})
to store private keys and perform signing operations. On the other hand, the specification largely
ignores the issue of revoking a key that has been lost or stolen.

Another hardware-based authentication project is Pico~\cite{Stajano11}, which also proves ownership
of a secret key by signing a challenge. Pico addresses the risk of key theft by using $k$-out-of-$n$
secret sharing across several hardware modules carried by the user, such as glasses, belt, wallet,
and jewellery. This approach is theoretically interesting, but the prospect of a practical
deployment seems remote.

Whilst the security characteristics of hardware modules are appealing, the user experience is quite
inconvenient at present. For example, a USB-based module cannot be used on a mobile device that has
no USB port. A small number of high-value services (such as online banking) can afford to
significantly inconvenience users in the name of security, but for the majority of online services,
any authentication method that is less convenient than a password is unlikely to gain adoption, even
if it is significantly more secure.

It is conceivable that in future, authentication modules will be ubiquitously built into devices, so
that users can to use them without the friction of a special-purpose authenticator device. However,
we seek a solution that can be deployed today, without waiting for a hardware upgrade cycle.

% https://www.schneier.com/blog/archives/2014/09/security_of_pas.html
% https://crypton.io/
% https://www.lightbluetouchpaper.org/2014/10/02/pico-part-iv-somethings-you-have/
% http://news.engineering.utoronto.ca/bionym-raises-14-million-wearable-password-replacing-tech/
% https://www.digits.com/
% http://www.cl.cam.ac.uk/techreports/UCAM-CL-TR-817.pdf (already in Papers)

\section{Octokey protocol overview}\label{sec:protocol}

Octokey is a user authentication protocol designed to meet the goals stated in the introduction. It
aims to provide good security in a software-only configuration, with cryptographic hardware modules
being an optional enhancement. It strives to be a \emph{trust-free} protocol: it is completely
decentralized, there is no dependency on any identity provider, and the extent to which PKI
certificate authorities need to be trusted is minimized as far as possible.

The basic operation of Octokey is inspired by SSH public key authentication, which is widely used
for shell access to remote servers. When using Octokey for the first time, a user generates a RSA
keypair.\footnote{We hope that the approach could be extended to support other public-key
cryptosystems such as ECC.} When the user signs up to a service, or migrates from another
authentication method to Octokey, the user submits their public key to the service. This is
analogous to creating a password when a user first signs up to a service.

Whenever the user wishes to log in, they must prove ownership of the private key. A client does this
by requesting a challenge from the server, signing it using the private key, and submitting the
signature to the server. The details of this protocol are given below.

We can think of Octokey as a machine-to-machine authentication protocol, and it needs to be preceded
by a human-to-machine authentication step: for example, a password or biometric information can be
used by the client device to unlock or decrypt the private key. However, since this password or
biometric information is not sent over the network, this concern is orthogonal to the
machine-to-machine authentication. Any improvements in biometric sensors, for example, do not
require any changes to the Octokey protocol or any support by service providers.

If a password is used to decrypt the user's private key, they need only remember a single password
per device, but not a separate password for every service where they have an account. We believe
that a single password (like in a password manager) is an acceptable user experience.

\subsection{Challenges}\label{sec:challenges}

A service that accepts Octokey login must generate challenges, which are then signed by clients (see
section~\ref{sec:mandates}). The client does not interpret challenges, but treats them as opaque
byte strings. We propose that a service constructs a challenge from a wall-clock timestamp $t$, a
nonce $x$, and a secret key $K$ that is known only to the service:
$$c = t \concat x \concat \mathrm{HMAC}(K, t \concat x)$$
The symbol $\concat$ denotes encoding and concatenating the values into a byte string. When a client
submits a signed challenge to the service, the service can verify its validity by checking that all
of the following are true:
\begin{itemize}
\item $\mathrm{HMAC}(K, t \concat x)$ is a valid HMAC using the secret key K. This ensures that the
challenge was issued by this service.
\item $t$ is a recent timestamp (e.g. no more than 5 minutes old). This is a guard against offline
brute-force attacks.
\item $x$ has not been used within the time interval $t$. This prevents replay attacks.
\end{itemize}

\subsection{Authentication mandates}\label{sec:mandates}

Each user has a RSA keypair $(n, d, e)$ where $n$ is the modulus, $d$ is the private exponent and
$e$ is the public exponent. A key also has an expiry date $\mathit{ex}$, which is chosen when the
key is generated, and which is used to rotate the user's keypair periodically (see
section~\ref{sec:rotation}). The service may have multiple public keys on record for a user, and
should allow any one of those keys to authenticate as the user.

To log in or sign up, the user's client first requests a challenge $c$ from the service via a HTTP
endpoint. It then calculates $m = H(c \concat u \concat r \concat \mathit{ex})$ where $u$ is the URL
of the service endpoint, $r$ is the user's registered username, and $\mathit{ex}$ is the expiry date
of the key. $H$ is shorthand for the \textsc{EMSA-PSS-Encode} operation (hashing and padding) as
defined in PKCS\#1.~\cite{PKCS1} The RSA signature can then be calculated as $s = m^d \mod n$.

The client then constructs the \emph{mandate}, which is an encoding of the RSA-signed message and
the user's public key: $$\mathit{mandate} = s \concat c \concat u \concat r \concat \mathit{ex} \concat n \concat e.$$
The mandate is sent to the server as part of a HTTP request over TLS, and is handled at the
application layer. The server can verify the mandate by checking that all of the following are true:
\begin{itemize}
\item $s$ is a valid PKCS\#1 signature of the message $c \concat u \concat r \concat \mathit{ex}$,
checked against the public key $(n, e)$.
\item $c$ is a valid challenge, as defined in section~\ref{sec:challenges}.
\item $u$ is a valid URL for this service. A mandate for an unknown URL must be rejected. This
prevents phishing-like attacks, whereby an attacker creates an imitation website under a
similar-looking URL and tricks the user into logging in.
\item For a signup request, the username $r$ is not yet taken. For a login request, $r$ is a
registered username for this service, and $(n, e, \mathit{ex})$ is a public key for that user.
\item $\mathit{ex}$ is in the future.
\end{itemize}

If a login mandate is successfully verified, the user is logged in by the server in the usual
manner, for example by setting an appropriate session cookie. If a signup mandate is successfully
verified, the signup flow continues as usual, e.g. asking for the user's name and verifying their
email address.

\subsection{Protection against key theft}\label{sec:revocation}

Many users have multiple devices (e.g. laptop, smartphone, tablet, game console) on which they need
to be able to log in to their accounts with online services. As described above, the private key
$(n, d, e)$ would need to be copied to each of those devices. If any of those devices is lost or
compromised, and an attacker can break the human-to-machine authentication step (perhaps due to a
weak password on the private key), the attacker could gain access to all of a user's accounts.

To mitigate this risk, we ensure that the private exponent $d$ is never stored on any one device.
Instead, we split it into key fragments that are distributed among the user's devices. We use the
\emph{mediated RSA} (mRSA) scheme~\cite{Boneh01, Kutyiowski12} which is based on the fact that
$$s = m^d = m^{d_a + d_b} = m^{d_a} m^{d_b} \mod n$$ provided that $d = d_a + d_b \mod \phi(n)$.

If two devices $a$ and $b$ each have a key fragment $d_a$ and $d_b$ respectively, and those
fragments sum to the private exponent $d$, then we call those devices \emph{paired}. In order to
generate a valid signature, any two paired devices need to collaborate. If device $a$ wants to
generate a mandate, it can send a signing request $\mathit{req}$ to device $b$:
$$\mathit{req} = H(c \concat u \concat r \concat \mathit{ex}) \concat n \concat e$$
where the public key $(n, e)$ indicates which key should be used, in case device $b$ stores multiple
keys. Device $b$ then uses its key fragment $d_b$ to calculate a response:
$$\mathit{resp} = H(c \concat u \concat r \concat \mathit{ex})^{d_b} = m^{d_b}$$
and returns $\mathit{resp}$ to $a$. Now, $a$ can calculate $s = m^{d_a} m^{d_b}$ by using its own
key fragment $d_a$ and $\mathit{resp}$, construct a mandate with a valid signature, and thus log in.

If a device is lost, stolen or compromised, this scheme allows the user to revoke that device's
login capability: every device that is paired with the lost device must be instructed to delete the
key fragment from the pairing with the lost device. When all the paired fragments have been deleted,
the key fragments on the lost device become useless. Thus, even if the human-to-machine
authentication is weak, not all is lost: the user only needs to revoke the lost device's key
fragments faster than an attacker can break the human-to-machine authentication.

A key could also be split into more than two fragments; we discuss this in section~\ref{sec:multiway}.

\subsection{The mediator service}\label{sec:mediator}

Authenticating by using paired physical devices (e.g.\ a laptop and a smartphone) yields a similar
user experience to current 2-factor authentication solutions, whereby the user must fetch the phone
from their pocket, launch the appropriate app, and perform some kind of handshake. This is
possible, but distinctly less convenient for users than typing a password, so it is not the simple
user experience we are looking for.

However, there is a simple solution within the mRSA framework: one of the user's `devices' may be a
remote service on the internet, which we call the \emph{mediator}. This service stores key fragments
that are paired with each of the user's physical devices, and responds to signing requests by
performing the modular exponentiation using its key fragments. This allows a user to authenticate
with services using only one physical device -- the coordination with the mediator happens
automatically behind the scenes.

The mediator need only be partially trusted. It cannot authenticate as the user without the
cooperation of one of the user's physical devices. The user only needs to trust the mediator to not
collude with attackers who steal devices, and to correctly delete key fragments when the user
requires key revocation. The user's privacy is protected by hashing the message
$c \concat u \concat r \concat \mathit{ex}$ before sending it to the mediator, so it does not learn
which services the user is logging in to, or which usernames they are using.

From the point of view of a service that accepts Octokey login, the mediator does not even exist: a
service simply verifies the RSA signature on a mandate, and does not care how that signature was
constructed. This is in contrast to federated login systems, where the relying party must trust the
identity provider.

\subsection{Rate limiting password guesses}\label{sec:ratelimit}

Besides enabling key revocation, mRSA can also be used to strengthen the human-to-machine
authentication step against offline attacks.

For example, say the key fragment on a device is encrypted with a symmetric key derived from a
password.\footnote{This discussion also applies to other human-to-machine authentication methods,
for example an encryption key that is derived from biometric measurements.} Consider an attacker who
has stolen this encrypted fragment. In order to brute-force the password, the attacker needs a way
of determining whether a password guess is correct. However, a key fragment is just a uniformly
distributed random number; by itself, the correctly decrypted key fragment is indistinguishable from
the garbage that results from trying to decrypt with the wrong password.\footnote{If the encryption
uses a block cipher, the key fragment must be padded with random bits up to the block size. Padding
with a predictable bit pattern would leak information on whether a password guess was correct.}

Assuming the attacker has no other key fragments, they can only determine whether the password guess
was correct by communicating with the mediator and testing whether they are able to construct a
valid PKCS\#1 signature. This gives us an opportunity to rate-limit password guessing attempts: if
the mediator receives too many requests based on an incorrect password, it can block further
attempts and advise the user to revoke the device pairing.

In order to achieve this, we must design the protocol such that an attacker must communicate with
the mediator for every password attempt, but without revealing the password or the decrypted key
fragment to the mediator. We can do this as follows:

Say the key fragment $d_a$ has been encrypted with password $\mathit{pass}$, and the attacker has
stolen the encrypted fragment
$$\mathit{efrag} = \mathrm{encrypt}(\mathrm{PBKDF2}(\mathit{pass}), d_a).$$
The attacker now guesses $\mathit{pass}^\prime$ and computes a guess $d_a^\prime$ of the plaintext:
$$d_a^\prime = \mathrm{decrypt}(\mathrm{PBKDF2}(\mathit{pass}^\prime), \mathit{efrag})$$
To check whether $d_a^\prime = d_a$ the attacker needs to contact the mediator where $d_b$ is held.

We modify the mediator's request processing as follows:
\begin{enumerate}
\item In addition to the signing request $\mathit{req}$, the client is required to submit a
signature $s_\mathit{req}$:
\begin{align*}
    \mathit{req} &= H(c \concat u \concat r \concat \mathit{ex}) \concat n \concat e \\
    s_\mathit{req} &= H(\mathit{req} \concat \mathit{cb})^{d_a^\prime}
\end{align*}
where $\mathit{cb}$ is the \texttt{tls-unique} channel binding~\cite{ChannelBinding}
of the TLS connection between the client and the mediator. Channel binding is further discussed in
section~\ref{sec:channelbinding}.
\item The mediator queries its database for a key fragment $d_b$ belonging to the user with public
key $(n, e)$. If the device uses a TLS client certificate when connecting to the mediator (see
section~\ref{sec:channels}), it can retrieve the key fragment for the authenticated client device.
\item Using the channel binding $\mathit{cb}^\prime$ of the TLS connection's server side, the
mediator computes
$$s_\mathit{req} H(\mathit{req} \concat \mathit{cb}^\prime)^{d_b} =
  H(\mathit{req} \concat \mathit{cb})^{d_a^\prime} H(\mathit{req} \concat \mathit{cb}^\prime)^{d_b}$$
and checks whether the result is a valid PKCS\#1 signature of
$\mathit{req} \concat \mathit{cb}^\prime$ for the user's public key $(n, e)$. This check succeeds if
$d_a^\prime = d_a$ (i.e. the user's password was correct), and if $\mathit{cb}^\prime = \mathit{cb}$
(preventing MITM and replay attacks).
\item If the signature is valid, the mediator computes
$\mathit{resp} = H(c \concat u \concat r \concat \mathit{ex})^{d_b}$ as before, and returns it to
the client. If the signature is not valid, the mediator returns ``bad signature''. A
password-guessing attacker learns that the password guess $\mathit{pass}^\prime$ was incorrect, but
otherwise nothing is revealed that would help them guess the password.
\end{enumerate}

Note that although the mediator computes an RSA signature using the user's private key, the value
being signed ($\mathit{req} \concat \mathit{cb}$) cannot be used to construct a mandate, so the
mediator cannot log in to services on the user's behalf.

This protection against password guessing only works if the attacker does not have any knowledge of
previous requests to the mediator. If the attacker knows $x^{d_a}$ (a request) or $x^{d_b}$ (a
response) for any $x$, they can brute-force the password without contacting the mediator, and thus
circumvent the rate-limiting.  It is therefore important that communication with the mediator is
protected from eavesdropping (using TLS) and is not logged on the device.

\subsection{Channel binding and preventing MITM}\label{sec:channelbinding}

TLS connections are susceptible to man-in-the-middle attacks -- using forged TLS certificates, due
to users ignoring warnings about invalid certificates, or due to malware. Such incidents have been
observed in practice.~\cite{Huang14, Adkins11} An attacker who succeeds in establishing a MITM
position can steal signed mandates, and use them to impersonate the user. Can we secure the
connection between the device and the service against MITM?

Some authentication methods such as SCRAM-SHA-1-PLUS~\cite{SCRAM} use \emph{channel binding} to
prevent MITM attacks. For example, the \texttt{tls-unique} channel binding type works by hashing the
handshake messages that established the TLS connection: in a direct connection, server and client
obtain the same hash value, but if the connection was terminated and restarted by a MITM, the server
and client's values differ. If the client incorporates this hash value into the mandate (such that
it cannot be changed by the MITM), and the server checks that it equals the server-side view of the
connection, then a mandate is rendered invalid by the presence of a MITM.

Origin-Bound Certificates~\cite{Dietz12} generalize this idea to the web, creating a channel
binding that is not tied to a particular TCP connection. When a client first connects to a
particular domain name, it automatically creates a self-signed TLS certificate (without any user
interaction) and presents it to the server. This certificate does not have any authentication
purpose, but the fingerprint of the certificate could be incorporated into a mandate as a channel
binding.

APIs for accessing TLS channel bindings and creating TLS certificates are currently not readily
available in web browsers and in HTTP server implementations. Thus, while channel binding for
Octokey mandates would yield a significant improvement in security, it would also make the protocol
much harder to deploy, both on the client and on the server side. We therefore propose making it an
optional extension.\footnote{For the communication between Octokey clients and mediator, as
described in section~\ref{sec:ratelimit}, we do require channel binding. As this protocol only needs
to be implemented in the Octokey software, not by every service that accepts Octokey as
authentication mechanism, the burden of deployment is much smaller.}

An alternative effort to prevent MITM is the Certificate Transparency project~\cite{CertTrans},
which aims to strengthen trust in the PKI by providing a public audit log of issued certificates,
and rejecting certificates that do not appear in the log. Certificate Transparency does not protect
against attackers who have stolen a website's private key (or governments, which can obtain the
private key with a court order), but it does prevent more casual kinds of MITM attacks, so it may be
sufficient.

\section{Inter-device communication}\label{sec:interdevice}

Octokey assumes that the user has multiple internet-connected devices: two at minimum (one physical
device and one mediator), but likely several more (desktop computer, laptop, smartphone, tablet,
game console, television, automobile, etc).

We assume that users want to be able to access all of their services on any one of their physical
devices (devices with partial access to the user's services are discussed in
section~\ref{sec:delegation}). Thus, each physical device should be paired at minimum with the
mediator. Some physical devices may also be paired with each other, to provide redundancy in case of
a failure of the mediator (discussed in section~\ref{sec:theftloss}).

The system should make it easy to enroll new devices to the user's set of trusted devices, and to
revoke any devices which are lost or no longer in use. It may also provide convenience features,
such as synchronizing the list of usernames for all of a user's services, so that they can be
autofilled in login forms. In order to do this, Octokey needs a mechanism for the user's devices to
synchronize with each other.

\subsection{Communication protocol}\label{sec:channels}

First, we need to be able to establish a secure point-to-point communication channel between any two
of the user's devices (treating the mediator as one of the devices). The channel must provide
secrecy and integrity, so that an attacker cannot eavesdrop or modify communcation between devices.
It must also prevent spoofing, so that an attacker cannot impersonate a device or MITM a channel.

TLS~\cite{TLS} meets these requirements. Each device generates its own private key and self-signed
certificate, which are used only for inter-device communication and are independent from the mRSA
key fragments. One device then connects to another and performs a TLS handshake, using the devices'
certificates as client and server certificates for mutual authentication.

This approach does not require any PKI or certificate authorities: public key fingerprints are
exchanged out-of-band when devices are first discovered (see section~\ref{sec:newdevice}), so each
device can check whether its peer is the device that it claims to be. It is not important which
device acts as client or server in this channel, as the relationship between the devices is
symmetric.

However, there is a practical problem: it can be difficult to establish a peer-to-peer TCP
connection between two arbitrary devices, as they may be on different networks, behind firewalls,
and they may often be offline -- and rarely online at the same time. Thus, in addition to performing
TLS over TCP, we propose using an alternative transport layer in the form of a relay service.

The relay is a web service that can be reached from all devices (it may be co-located with the
mediator). It accepts messages from one device and relays them to another device: if the recipient
is online, the messages are forwarded immediately, otherwise they are stored until the recipient is
next online. The messages are records of the TLS Record Protocol~\cite{TLS}, so two devices can
perform a TLS handshake and establish a secure channel by sending each other messages via the relay.
The service needs to be highly available, but it does not need to be trusted for security purposes.

If devices are only occasionally online, the round-trip-times across this ``network connection'' may
be measured in days, so timeouts may need to be adjusted accordingly. However, once this channel has
been set up, it can remain open indefinitely (with occasional renegotiation), even across device
reboots.

Besides this relay, devices should support alternative transport mechanisms, such as direct TCP
connections on a local network, Bluetooth or NFC. These may only occasionally be used, but they are
good fallbacks so that the relay does not become a single point of failure. The same TLS record
protocol can be used over those transport layers.

\subsection{Data synchronization}\label{sec:devicesync}

These point-to-point channels are used for exchanging mRSA signing requests and responses between
devices. Moreover, we can use them for replicating information that we want stored on all devices:
each of the user's devices maintains a local database containing public key fingerprints of all of
the user's other devices (for mutual authentication of channels), configuration (e.g. URL of the
mediator), and information on which devices are paired (for key revocation purposes). In addition,
the user's physical devices maintain a database of usernames on all the user's services (for
username autofill), although this should not be replicated to the mediator, to protect the user's
privacy.

Each of the user's devices needs to be able to read and modify this database, and any changes should
be asynchronously propagated to the other devices. To this end, we can use the point-to-point
channels of section~\ref{sec:channels} to create a mesh of connections among the user's devices
(even between those which are not paired). It is not necessary to have a communication channel
between every pair of devices, because devices can forward incoming database changes to others.
However, a fairly densely connected graph of channels is desirable, so that any message from one
device reaches the other devices fairly quickly, and so that the communication is resilient to
device and channel failures.

Within this mesh network, we can use a \emph{gossip protocol} (also known as \emph{epidemic
protocol}) \cite{Demers89} for disseminating database changes. \emph{Version vectors}
\cite{ParkerJr83} can be used to efficiently detect data differences that need to be synchronized,
and \emph{CRDTs}~\cite{Shapiro11} can automatically resolve any conflicts between concurrent
database updates, without requiring user interaction. Every database change must be signed with the
device key of the device on which it originated, so that any changes made on a revoked device can be
rejected, and so that one device cannot make changes on another device's behalf.

\subsection{Bootstrapping}\label{sec:bootstrap}

When someone first starts using Octokey, they go through the following steps:

\begin{enumerate}
\item The user installs the Octokey application on a physical device, and selects ``create new
Octokey''.
\item The application generates the user's RSA keypair $(n, d, e)$, as well as the keypair and
self-signed certificate for that device.\footnote{It is possible, albeit rather slower, to perform
the key generation distributed across multiple devices, so that no device ever has the entire $d$.
\cite{Boneh01b} However, if we want to allow the user to print off the entire key as a backup, we
need to have the entire key in one place anyway. See section~\ref{sec:theftloss} for a discussion.}
\item The application contacts a mediator of the user's choice (a default may be provided) over TLS,
using its device certificate for client authentication, and verifying the server using well-known
certificate authorities (or a pinned public key fingerprint embedded in the application).
\item The application splits the private exponent $d = d_1 + d_2$ such that $d_1$ is a uniformly
distributed random integer with $0 < d_1 < d$. The device stores $d_1$ locally, and sends $d_2$ to
the mediator.
\item The user has the option of printing the private key on paper, as a last-resort backup in case
all of their devices are lost (see section~\ref{sec:theftloss}). Then the application deletes $d$.
\end{enumerate}

\subsection{Enrolling a new device}\label{sec:newdevice}

When the user has already set up Octokey on one or more devices, and wishes to enroll another
physical device (i.e. pair a new device with their existing devices), they proceed as follows:

\begin{enumerate}
\item The user installs the Octokey application on the new device, and selects ``set up this device
to use your existing Octokey''.
\item The application generates that device's keypair and self-signed certificate. It then registers
itself at the relay, using its device certificate for client authentication. The relay allocates a
URL at which this device can be reached, using the device's public key fingerprint as recipient
identifier.
\item The application on the device displays a 2D barcode on screen, containing an indicator that
this is an enrollment request, the relay URL at which it can be reached, the device's public key
fingerprint, and optionally information about other transport channels (e.g. Bluetooth) where it can
be reached. The barcode must not contain any sensitive data, since it may be visible to
eavesdroppers (see section~\ref{sec:barcode-intercept}).
\item The user scans this barcode using the camera of their existing Octokey device. The application
connects to the URL in the barcode, negotiates an end-to-end TLS connection and checks that the
remote key fingerprint matches the one in the barcode.
\item The existing Octokey device prompts the user whether they want to trust the device whose
barcode they just scanned. If the user is already using multiple devices, the applications on those
devices may additionally prompt the user.
\item When sufficient user approvals have been obtained, the device that scanned the barcode
registers the new device (including its public key fingerprint) in the database. This change is
replicated to other devices, which may connect to the new device (adding it to the mesh network).
\item The new device initiates the key re-pairing protocol (section~\ref{sec:pairing}) in order to
obtain key fragments.
\end{enumerate}

This interaction flow assumes that the first physical device on which the user sets up their private
key has a camera. This is plausible, since many laptops and smartphones come with built-in cameras
today. In the case where no camera is available, the user can type in the contents of the barcode in
textual form. Devices which are enrolled later do not require a camera, only a screen. For example,
a game console or an automobile on-board computer may not have a camera, but they can nevertheless
be enrolled.

If the user's existing device does not have internet access, and thus cannot reach the URL in the
barcode, the devices can set up the secure channel over a local wireless medium such as
Bluetooth, NFC or 802.11/WiFi instead. The necessary information to establish this connection can be
included in the barcode. We require that the enrollment is initiated using the barcode, and not
through the wireless medium, to ensure that an attacker cannot tamper with the public key
fingerprint (see also section~\ref{sec:barcode-phishing}).

\subsection{Key re-pairing}\label{sec:pairing}

Each mRSA pairing of devices requires the private exponent $d$ to be split in different ways:
$d = d_1 + d_2 = d_3 + d_4 = \dots$ with $d_1 \neq d_2 \neq d_3 \neq d_4$ (all values are drawn from
a uniform random distribution, so there is a very small probability that they might be equal).

Say there are two existing devices: device $a$ stores key fragment $d_1$ and device $b$ stores key
fragment $d_2$. When a new device $c$ is enrolled, and we want to pair it with device $a$, we need
to generate new key fragments $d_3$ (stored on device $c$) and $d_4$ (stored on device $a$). We can
do this as follows, without reassembling the entire private exponent $d$ on any single device (and
thus risking it being stolen):

\begin{enumerate}
\item Device $c$ chooses a random integer $d_3$ from the uniform range $0 < d_3 < n$ (where $n$ is
the RSA modulus).
\item Device $c$ chooses a second random integer $k$ from the uniform range $0 < k < d_3$.
\item Device $c$ sends $k$ to device $a$, and sends $(d_3 - k)$ to device $b$.
\item Device $b$, where $d_2$ is stored, calculates $(d_2 - (d_3 - k))$ and sends it to device $a$.
This number may be negative.
\item Device $a$, where $d_1$ is stored, has now received $k$ and $(d_2 - (d_3 - k))$. It uses these
to calculate $$d_4 = d_1 + (d_2 - (d_3 - k)) - k = d_1 + d_2 - d_3.$$
\item If $d_4$ is positive, the algorithm was successful: device $a$ stores $d_4$ and we're done. If
$d_4$ is negative, we start again from step 1, choosing new random numbers.
\end{enumerate}

The algorithm ensures that $d_3 + d_4 = d$, so $d_4$ is negative iff $d_3 > d$. However, since no
device knows $d$, we cannot simply generate $d_3$ to be less than $d$, but we must retry with new
random numbers until a suitable $d_3$ is found.

A malicious device could use this algorithm to determine $d$ by binary search. However, there are
also many other ways for a malicious device to reconstruct $d$, for example by pairing with a
colluding device. Thus, we must trust the software on the user's physical devices (see
section~\ref{sec:malware}), and use the fact that a device can only initiate the pairing process
after it has been authorized by the user. We also prohibit the mediator from initiating the
re-pairing process.

\subsection{Number of retries}\label{sec:retries}

The key re-pairing algorithm in section~\ref{sec:pairing} requires a potentially unbounded
number of retries. To avoid excessive waiting times for the user, we want to keep the number of
retries fairly small.

The number of retries $R$ is a geometrically distributed random variable with an expected value of
$\frac{n}{d}-1$. This means that if the private exponent $d$ is significantly smaller than the
modulus $n$, the number of retries can be high.

We therefore suggest that when the private key is first generated, it must be greater than some
threshold, e.g. $d \ge \frac{n}{16}$. With this restriction, a worst-case key would require 15
retries on average, and would have an approximately 0.1\% risk of requiring more than 100 retries.
Most keys would require significantly fewer retries.

Although this restriction would exclude 1/16th of the possible keyspace, with a 2048-bit key this is
equivalent to reducing the key size to $\log_2(2^{2048} - 2^{2044}) \approx 2047.9$ bits. It seems
unlikely that a loss of 0.1 bits would significantly harm the strength of the algorithm, but in any
case the cryptanalytic consequences of such a key restriction need to be studied carefully.

\subsection{Delegated login by mobile device}\label{sec:delegation}

A user should be able to authenticate with services on a semi-trusted device, for example a game
console at a friend's home, without having to generate key fragments for that device (and thus
giving the device full access to all of the user's accounts). To this end, we use the fact that a
signed Octokey mandate is like a one-time password: it can be given to another device in order to
authorize one-off access to one particular service.

By logging in, the semi-trusted device exchanges the mandate for a session identifier (e.g. a
cookie), which is destroyed when the user logs out, or after a timeout. In cases where the user
wants to give a device long-lived access to one service, the device can obtain an OAuth~\cite{OAuth}
token from the service. This is standard use of OAuth and does not require any special support from
Octokey.

The question is then how to get the mandate onto the device that needs it. We propose the following
flow:

\begin{enumerate}
\item Steps 1 to 4 are identical to the flow in section~\ref{sec:newdevice}, except that the user
selects ``log in with your Octokey on another device'' instead of ``set up this device to use your
existing Octokey''. The 2D barcode indicates that this is a signing request, not an enrollment
request.
\item The device requesting login requests a challenge $c$ from the service.
\item Using the secure channel that has been set up between the devices, the device requesting login
sends $c$ and the login URL $u$ to the device that scanned the barcode.
\item The latter device prompts the user whether they want to allow the device whose barcode they
just scanned to log in to their account at URL $u$. If yes, they select the username $r$ of the
desired account (which may be autofilled from their database of services).
\item If the user approves, the device goes through the usual mRSA signing flow to calculate a
signature $s$, and sends the mandate
$s \concat c \concat u \concat r \concat \mathit{ex} \concat n \concat e$
to the device requesting authentication, using the secure channel. That device sends the mandate to
the service to log in.
\end{enumerate}

It is important that the user checks whether the authentication request is for the URL they were
expecting, because a device could potentially request login for any service. As a usability
improvement, the user could be shown a screenshot of the website at the requested URL, not just the
URL itself. This would make it visually obvious if the user's banking login is being requested, when
they thought they were logging into an online game. (As before, creating a phishing website with
imitation design would not work --- in order to obtain a valid mandate for the user's bank, an
attacker must supply the real bank's URL.)

This process for delegated authentication is simple and fast, has fairly small risk of user error,
and can be implemented with any smartphone. It is significantly more pleasant to use than a password
manager, does not require typing anything, and is more secure.

\subsection{Key rotation}\label{sec:rotation}

A key that is considered strong today may be regarded as weak in future. Key fragments may be left
on old, rarely-used devices, so the user may forget to revoke them when they are lost or thrown
away. Security vulnerabilities may from time to time raise fears that private key material may have
been exposed. For all these reasons, we must assume that every key has a finite lifespan, and build
mechanisms for key rotation into the Octokey protocol from the start.

As described in section~\ref{sec:mandates}, every user keypair has an expiry date. We propose making
this four months after the key's creation date by default. Services must not accept public keys
without associated expiry date, or keys that have expired, or keys with an invalid signature on the
expiry date.

One month before the user's current key expires, one of the user's devices generates a new key
(similar to section~\ref{sec:bootstrap}), and distributes key fragments to the user's enrolled
devices (section~\ref{sec:pairing}). Each service should implement a HTTP API by which a device can
renew the user's public key: a mandate signed with the user's current key is sufficient
authorization to add a new public key to that user's account. The old and new key are both accepted
until the old key expires (see discussion in section~\ref{sec:recovery}).

Thus, a device that lies unused in a box in the user's attic will automatically lose access to the
user's accounts between one and four months after it was last used. A device whose enrollment has
lapsed can be re-enrolled. However, devices in active use will not require re-enrollment, because
they are updated with the latest key before the old key expires.

Our goal is to allow key rotation to happen automatically, with as little user interaction as
possible. The only main place where user interaction occurs is for printing a backup of the new key
on paper (see section~\ref{sec:theftloss}). It would be annoying for users to have to print
something, and take it to their safe storage place, every three months. Perhaps even this could be
automated: one could imagine banks or attorneys providing an API to which a user can securely submit
their private key; the bank promises to print it, to store it in a safe, and to return it to the
user if they authenticate themselves in person (by presenting a passport). Such an API would allow
key rotation to happen entirely without user interaction.

\section{Threat model and trade-offs}\label{sec:threat}

In this section, we discuss the assumptions we have made with regard to attackers' capabilities,
some aspects which require careful implementation, and some areas where different concerns need to
be traded off against each other.

\subsection{Malware and untrusted software}\label{sec:malware}

Most authentication schemes in a consumer context are based on the principle that once a user has
authenticated, they have unlimited access to their account: using different credentials for
different user actions is considered too inconvenient.\footnote{An exception are some online banking
websites, which require explicit authorization of each payment action, using a hardware token.}

Consequently, once a user has authenticated with a service on a device, the software on that device
can in principle do anything it wants to the user account. That is true no matter which
authentication method is used: if there is malware which can read files, log keystrokes and steal
session cookies, or if an attacker can obtain remote root access to a device, any services in use on
that device can be compromised.

With Octokey, there is the additional risk of the device's private key (for inter-device
communication) and mRSA key fragment being stolen. If the device has a built-in hardware security
module, the actual keys are protected, but malware can nevertheless sign arbitrary authentication
requests, because it is almost indistinguishable from legitimate user software. Thus, if malware is
detected on a device, the user must immediately revoke that device using another enrolled device.

In sandboxed environments (e.g. websites in web browsers, apps on some mobile OSes), mutually
untrusting applications are somewhat protected from each other. In this case, there needs to be an
explicit way of passing authentication requests between the Octokey application (which manages the
keys) and the websites and apps that require authentication. This mechanism must be implemented
carefully, ensuring that a website or application can only obtain a signed authentication request
for its own URL, and not some other one.

\subsection{Physical device theft and loss}\label{sec:theftloss}

We assume that the private key material on a device is protected by a human-to-machine
authentication step, e.g. encrypted with a password. This is not much use against malware, but it
does help against an attacker who physically steals a device (or a backup of a device's filesystem).
In section~\ref{sec:ratelimit} we discussed a technique for strengthening this step against brute
force and dictionary attacks.

If the user has enrolled another device, they can use that device to revoke the stolen device. If
the revocation happens before the human-to-machine authentication is broken, no harm is done. If the
device is not revoked in time, the attacker can sign arbitrary authentication requests, and steal
the key by enrolling another device and pairing it with the stolen device. The risk of key theft can
be mitigated by prompting the user on several devices before enrolling another device (provided that
the user has previously enrolled several devices).

If an attacker steals two devices that are paired with each other, the risk of key theft is much
greater, because the human-to-machine authentication is the only remaining protection: deleting key
fragments from the remote store is not sufficient to prevent the attacker from recovering the key.
The technique from section~\ref{sec:ratelimit} also no longer applies, because the encryption can be
brute-forced offline. This means there is a trade-off between security and reliability: pairing
physical devices with each other provides redundancy in case the remote key fragment store is
unavailable, but it increases the risk of the private key being stolen.

Another risk is that the user may lose access to their services: perhaps an attacker who gains
control of a device uses it to revoke the user's legitimate devices (thus preventing the user from
stopping the attacker); perhaps the user has only enrolled one or two physical devices, and they are
stolen simultaneously; perhaps the user's house burns down and all of their devices are lost. To
allow recovery from such situations, we recommend that users print off their private key on paper
and store it in a safe place.

An open question is whether the recovery key on paper should be the entire key, or a key fragment
paired with the fragment store. The advantage of making it a fragment is that the user can revoke it
in case the piece of paper is lost or stolen; the disadvantage is that an attacker who gains control
of a device can also revoke it, and thus lock out the legitimate user. On balance, it is probably
better to make it irrevocable (see also section~\ref{sec:rotation}).

\subsection{Network attacks}\label{sec:netattack}

A signed authentication request is tied to a particular server URL, so it cannot be reused with
another service (assuming that services verify the URL in authentication requests correctly). This
makes Octokey resilient to phishing attacks.

However, a signed authentication request is not tied to a particular client device, unless channel
binding (section~\ref{sec:channelbinding}) is used. This means that if an attacker obtains a signed
authentication request by eavesdropping, they can use it to authenticate as the user on that
particular service.

We rely on regular TLS to provide secrecy of the authentication request. The Octokey software
should, if possible, prevent users from accidentally sending an authentication request over an
unencrypted connection. Server certificate validation and HTTP Strict Transport Security (HSTS)
should be used to avoid simple MITM attacks with invalid certificates. We expect that Certificate
Transparency will help prevent more sophisticated MITM attacks with fraudulently issued
certificates. If the client is a native app, the service's public key fingerprint should be embedded
in the app and checked when connecting to the service (public key pinning).

For inter-device communication between enrolled Octokey devices, TLS with mutual authentication and
public key pinning is used, which makes it resistant to MITM without requiring CA certificates. The
public key fingerprints are distributed through a visual channel (using a 2D barcode), which is much
harder to tamper with than a network connection (see also section~\ref{sec:barcode-intercept}).

\subsection{2D barcode interception}\label{sec:barcode-intercept}

In the flow for enrolling a new device (section~\ref{sec:newdevice}), and with delegated login
(section~\ref{sec:delegation}), we proposed displaying a 2D barcode on the screen of one device, and
scanning it with the camera of another. We cannot assume confidentiality of these barcodes: an
attacker may snoop it electromagnetically~\cite{Kuhn05}, or simply point a camera at the victim's
screen.

An attacker can thus connect to the URL in the barcode, and enroll the new device to the attacker's
account, instead of the user's own account as intended. With delegated login, the attacker can get
the user to log in with the attacker's key rather than the user's own key. This does not compromise
the user's key, but if the user logs in to the wrong account and doesn't realize it, they may
inadvertently disclose sensitive information to the attacker while using the account.

It is not clear whether this would be a problem in practice. If it is, a mutual authentication step
could be added to the flows for enrolling a new device and for delegated authentication (for
example, verifying a 2D barcode in the other direction). However, this makes the process more
complicated for users, especially when trying to delegate authentication to a device which has no
camera, so the additional verification step should probably be optional.

\subsection{2D barcode phishing}\label{sec:barcode-phishing}

When enrolling a new device or performing delegated authentication, we assume that an attacker
cannot trick a user into scanning a different barcode from the one they intended --- i.e. we assume
that the visual channel between the barcode-displaying and the barcode-scanning device provides
integrity (but not confidentiality). We assume that malware (section~\ref{sec:malware}) would be the
only way for an attacker to manipulate what is displayed on screen, or manipulate the camera signal.

However, a real risk is that a malicious website or app displays a barcode that originates from an
attacker-controlled device, and tricks the user into scanning that barcode and granting the attacker
unwanted access. In this scenario we must rely on well-written warning messages and user education
to ensure the user really understands what they are doing.

For users who have already set up multiple physical devices, as an additional safeguard against
accidentally pairing with an attacker-controlled device, we can require that enrolling a new device
requires user approval on a quorum (e.g. majority) of existing devices. This does not require
additional cryptographic algorithms, but can simply be implemented as a policy in the Octokey
software.

\subsection{Online services}\label{sec:store-security}

The remote key fragment store and the store-and-forward service are not strictly required from a
cryptographic point of view, since the user could in principle use Octokey exclusively with physical
devices and direct device-to-device communication. However, they are very useful for providing a
good user experience, allowing a user to authenticate using only one physical device, and making
inter-device communication work ``out of the box''.

From a service owner's point of view, the key fragment store does not need to be trusted (unlike an
OpenID identity provider, for example), because a signed authentication request is simply an RSA
signature in any case. Thus, the choice of key fragment store and store-and-forward services is
entirely up to the user. We propose that a public service should be provided for free, operated by a
nonprofit foundation. However, technically sophisticated users or organizations may choose to
operate their own fragment stores, if they wish.

Operating a fragment store is not an easy task, since it is likely to become a popular attack
target: some attackers may want to steal the key fragments and other user data, whereas others may
try to disrupt the availability of the service by flooding it with traffic. The operators of the
fragment store will need expertise in dealing with such adversaries.

In the worst case, if attackers succeed in obtaining a copy of all the key fragments in the remote
store, the operators of the store will have to inform all users, and the users will have to re-pair
all of their devices with a new fragment store. The inter-device communication protocol should have
a mechanism by which a store operator can force all users to re-pair. Such a break-in would not
\emph{per se} expose private keys, but any stolen devices that are not re-paired with a new store
are much more likely to result in key theft.

The store-and-forward service is less critical for security, since it cannot read the contents of
messages it delivers, and it cannot tamper with them. It can only drop messages instead of
forwarding them, and thus interrupt the inter-device communication. To guard against this
possibility, physical devices should also try to connect through local communication channels when
possible, and thus exchange any messages that may have been lost.

\subsection{Key rotation}\label{sec:rotation}

Requires printing off a new recovery key every month?

\subsection{Multiple keypairs per user}

% How does Octokey know that a certain website URL is the same service as a certain iOS
% bundle ID, which is the same service as an Android app signed with a certain public key?
% If each URL/bundle ID/app ID has a separate keypair, there needs to be some way of tying them
% together.

\subsection{Multi-way key splitting}
\subsection{Security levels}

% Sync login history across devices, and compare to login history reported by server, to detect
% use of a stolen key


\subsection{Recovering from key theft}

Prevent denial of service due to rate limiting or due to an attacker revoking a user's active
devices.

In the worst case, if an attacker manages to get the entire private key (e.g. by stealing or
compromising two devices that are paired, and breaking the human-to-machine authentication step),
the user's last resort is to generate a new key and update their accounts on all services, adding
the new public key and removing the old one. Unfortunately, the same key-swapping can be performed
by the attacker to lock out the legitimate user.

Therefore, perhaps changing a user's public key on a service should require an additional hurdle,
e.g. using a recovery key that is only stored on paper but not electronically? Or a key that is
split 3 ways? But that would make key rotation difficult.

% \section{Related work}
%
% Certivox M-Pin~\cite{Scott14}
%
% Resources:
%
% https://www.schneier.com/blog/archives/2014/09/security_of_pas.html
% https://crypton.io/
% https://www.lightbluetouchpaper.org/2014/10/02/pico-part-iv-somethings-you-have/
% http://news.engineering.utoronto.ca/bionym-raises-14-million-wearable-password-replacing-tech/
% https://www.digits.com/
% http://www.cl.cam.ac.uk/techreports/UCAM-CL-TR-817.pdf (already in Papers)
% http://www.certivox.com/docs/m-pin-core
% Microsoft CardSpace http://msdn.microsoft.com/en-us/library/aa480189.aspx?ppud=4
% SQRL

\section{Conclusion}

In this paper, we have introduced the design of Octokey, an authentication system intended for
consumer internet use cases. We argue that it provides better security than all commonly-used
alternatives (passwords, password managers, and federated authentication methods), and provides a
better user experience and better flexibility than more secure alternatives (hardware-based
authenticators, smart cards).

It is quite simple for service owners to start accepting Octokey as an authentication mechanism,
since libraries for verifying PKCS\#1 signatures are available for all common programming languages.
Octokey can also be used alongside passwords or other authentication mechanisms, allowing it to be
adopted gradually. It does not require any special hardware or any changes to the way online
services are deployed.

Most of the complexity of Octokey lies in the implementation of the client applications and the
mediator service. We plan to develop these using an entirely open source model, and to make them
available as free software for anyone to run and modify. The protocols should be open standards, and
anybody should be free to create alternative implementations if they wish.

Octokey strikes a pragmatic compromise between security and convenience, and strives for a very
simple user experience with minimal risk of user error. We hope that it can be the viable
alternative to passwords that has eluded us for so long, and we invite others to join the project,
give feedback, spread the word, and start working towards a real implementation.


{\footnotesize
    \bibliographystyle{plain}
    \bibliography{references}{}
}

\end{document}
