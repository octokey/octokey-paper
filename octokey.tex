\documentclass{acm_proc_article-sp}
\usepackage{cite}
\usepackage{hyperref}
\sloppy

\begin{document}
\toappear{This is an unpublished draft. Please don't distribute without our permission.}
\title{Octokey: Public key authentication for the web}
\numberofauthors{2}
\author{
    \alignauthor Martin Kleppmann \\ \email{martin@kleppmann.com}
    \alignauthor Conrad Irwin \\ \email{conrad.irwin@gmail.com}}
\maketitle

\begin{abstract}
We propose Octokey, a cryptographic protocol for authenticating users, designed as a viable
replacement for password authentication in consumer internet use cases. When a user signs up to a
service, they provide a public key. Clients subsequently authenticate by presenting a challenge
signed with the user's private key, similar to SSH public key authentication. We also describe
protocols which allow users to log in to their accounts on multiple devices, provision new devices,
and revoke keys on devices that are lost or stolen.
\end{abstract}

\bigskip

\emph{Note: We, the authors, have a background in developing commercial web applications. In the
field of crypography, we are merely keen amateurs. We are aware that amateur cryptography is prone
to catastrophic flaws, and we fully expect such flaws to be found in the ideas we are proposing.
However, we have experience designing real-world systems with good user experiences, a perspective
that we feel is often lacking in security research. We hope that this paper can stimulate a
discussion that brings together the perspectives from research and industry, and we would like to
solicit constructive feedback from all sides.}

\section{Introduction}

Despite their many well-known flaws, passwords are still by far the most commonly used
authentication system on the web.~\cite{Bonneau12} However, as users sign up for ever more services,
their use is becoming increasingly unsustainable: demanding that users choose passwords with
sufficient entropy, use a different password for every site, not write them down and even rotate
passwords periodically is unrealistic for most internet services.

With phishing attacks and leaks of password databases unfortunately commonplace, we seek a better
solution. In this paper we focus on authentication in a consumer internet context, such as social
media or e-commerce websites and mobile apps. For a majority of such use cases, an authentication
system with the following properties is desirable:

\begin{enumerate}
\item The system should minimize exposure to human error (such as falling for a phishing attack).
Even with education, human error cannot be avoided completely, so the system should be designed such
that it minimizes the impact in case of an error.
\item In case a device is lost or stolen, the user should be able to easily revoke any credentials
stored on the device, to minimize the chances that an attacker can use a stolen device to gain
access to other systems.
\item It is not necessary, and often undesirable, for the system to authenticate the user as a
physical person. Many services allow pseudonymous signup, which is desirable for reasons of privacy
or freedom of speech. The purpose of the authentication system is thus only to verify that the
current user is the same person as the one who originally signed up under a particular username; no
authentication is performed at signup.
\item The system should not depend on any trusted third party, to avoid risks arising from failure
of the third party (security compromise, going out of business, change of policies, etc).
\item The system should be easy for end-users to install and use, and easy for service owners to
deploy. Since user signup and user activity rates are important business metrics for many online
services, the system should make signup and login \emph{easier} than using a password.
\item Users should have the freedom to use the system on a variety of devices, including shared or
public computers that are only partially trusted, whilst minimizing their exposure to attackers.
For example, a user may choose log in to `unimportant' services (e.g. casual online games) on a
shared or public computer, knowing that their account on that service may be compromised; however,
they limit their use of `important' accounts (e.g. online banking) to trusted devices. Since there
is no generally agreed boundary between `important' and `unimportant' services, the same
authentication system should be able to support both use cases.
\end{enumerate}

In the next section we discuss various authentication systems that are used on the web, and explain
why none of them meet these requirements. We then introduce Octokey, an authentication protocol
which addresses these issues. We believe that Octokey could be a viable solution to online
authentication in the near future, and we seek the information security community's feedback to
refine this proposal.

\section{Existing systems}
\subsection{Passwords}

Passwords have many of the desirable properties mentioned above: no dependence on any third party,
pseudonymity, simplicity of implementation, familiarity and cross-device compatibility. However,
remembering a large number of passwords is a big burden on users, and they are highly susceptible to
human error: use of weak passwords, reuse of the same password across different services, phishing
attacks and leaks of unhashed or weakly hashed password databases are sadly common.

Web browsers' built-in password managers, or external password manager products such as 1Password
and LastPass, make it feasible for users to maintain a strong, unique password for each service they
use. However, even when a unique password is used, an attacker who succeeds in stealing a password
(e.g. by phishing, malware, keylogging, man-in-the-middle attack, eavesdropping when it is
accidentally sent over an unencrypted connection, or exploiting a vulnerability in the password
manager~\cite{Li14, Silver14}) has access to that user account until the password is changed, which
is often a long time.

If an attacker steals the encrypted password database (perhaps by stealing user's device, or a
backup of the filesystem), they can mount an offline dictionary or brute-force attack on the
encryption password. If the user fears the password database may have been stolen, they must
manually change the password for each of their services, which is laborious if they use many
services. If an attacker succeeds in breaking the encryption before the passwords are changed, they
may lock the legitimate user out of their accounts.

Password managers also have grave exposure to human error. For example, consider a user who wishes
to log in to an `unimportant' website on an untrusted computer. It is very tempting for the user to
type the master passphrase of their password manager into the untrusted computer, in order to
decrypt the password database. This has the effect of exposing the user's entire password database
to an attacker running malware on the untrusted computer.

\subsection{One-time passwords and two-factor authentication}

One-time passwords (OTPs) are a big security improvement over regular passwords. The user must
download a list of one-time passwords on a trusted device, or register a cryptographic device that
generates a pseudorandom sequence of OTPs, or register an email address or phone number to which
OTPs can be sent on demand. Some services allow use of OTPs as sole authentication mechanism, while
others use them in conjunction with a regular password (two-factor authentication).

The advantage of OTPs is that the exposure to attacks is limited to a single service, and ends when
the user clicks the logout button (assuming the session is correctly invalidated on the server, and
assuming that the attacker does not have a means of extending their privileges, e.g. by granting
themselves OAuth access to the account).

However, OTPs are so inconvenient to use that very few services are willing to adopt them as their
only or primary authentication mechanism, and only the most security-conscious users are willing to
use them. Moreover, the common ``remember me on this device'' feature weakens two-factor
authentication (attackers just need to steal the ``remember me'' cookie as well as keylogging the
password).

OTPs sent on demand via email or SMS also have the effect of `outsourcing' the authentication to the
email provider or the mobile network provider, respectively: any attacker who can access the email
provider or SMS messages can also gain access to any service using on-demand OTPs. This essentially
makes them federated login schemes (see next section).

\subsection{Federated identity}

OpenID~\cite{OpenID} is probably the best-known attempt to remove the need for a separate password
on every service. Some identity providers, such as Google Federated Login~\cite{GoogleOpenID} and
Facebook Connect, combine the ideas of OpenID and OAuth~\cite{OAuth}. Mozilla Persona
(BrowserID)~\cite{Persona, BrowserID} uses a different protocol, but is similar in the way it
delegates authentication to the user's email provider (or a fallback service provided by Mozilla).

Such federated authentication protocols do not solve the fundamental problem of authentication: they
only delegate it to the identity provider, who must then use some other authentication method (most
commonly a password, possibly in conjunction with a hardware token). This means that all the
problems discussed above apply to the identity provider, with an additional privilege escalation
problem: if an attacker gains access to the user's account with the identity provider, they can
access any account associated with that identity. Like with a password manager, a single user error
can lead to all of a user's accounts being compromised.

Moreover, the relying party (the service where the user is trying to log in) needs to trust the
identity provider to correctly authenticate the user.  If the identity provider is compromised, the
relying party has no way of detecting unauthorized logins from that identity provider. If the
identity provider experiences an outage or goes out of business, users of that provider lose their
ability to log in (unless they had previously set up delegation of their identity URL, which is
unrealistic to expect of non-technical users). When different services are in competition with each
other, it is typically not in one service's interest to accept a competing service as identity
provider.

For these reasons, OpenID login is typically only accepted by small services; major online services
rarely act as a relying party in OpenID. The competitive dynamic between services makes it unlikely
that a user will ever be able to use a single identity provider across all services.

\subsection{Client certificates}\label{sec:clientcerts}

TLS~\cite{TLS} provides a mechanism for a client to authenticate itself to the server using a X.509
certificate. The server may specify the certificate authority from which it is willing to accept
certificates. By calculating a signature over the TLS key exchange messages, client certificate
authentication also provides protection against man-in-the-middle attacks (a signature generated on
one TLS connection cannot be reused on another TLS connection).

Client certificates are a good solution in situations where the physical identity of the user is
important: certificates might be issued by a government to its citizens (for filing taxes online),
or by a company to its employees (for accessing internal systems). For example, the government of
Estonia issues certificates on smart cards to its citizens through the national ID card scheme, and
the certificates are widely used for authentication by banks, utility companies and other
organisations in Estonia.~\cite{Parsovs14}

Client certificate authentication can be performed by hardware tokens, e.g. using the PKCS \#11
\emph{Cryptoki} API. This provides some protection against private keys being stolen by malware, but
it is inconvenient for users (see section~\ref{sec:hardware}).

Despite their advantages, client certificates are not widely used for consumer internet services.
Problems include:
\begin{itemize}
\item The user interface for installing and managing certificates is unfriendly in most web
browsers and operating systems. Services are not able to customize the look and feel of the signup
and authentication process.
\item It is not possible to be logged in to several accounts on one service at the same time. Even
logging out and switching to another account is awkward.
\item There is no good solution for authenticating on multiple devices, or even on multiple
sandboxed applications on the same device. Either private keys must be copied to each device
(increasing the risk of key theft), or each device must have its own keys (in which case, every time
the user acquires or disposes of a device, they must must manually update each of their services to
reflect the certificates for their current set of devices -- very laborious if they use hundreds of
services).
\item Certificates issued by an external CA typically contain personally identifying information,
such as a real name or email address. In order to enable pseudonymous usage, and to remove the
third-party trust dependency, a service provider must itself act as a CA.
\item Revocation of certificates (which requires CRLs or OCSP~\cite{OCSP}) is often not implemented
correctly. If OCSP servers are slow or unavailable, a service provider must either fail all logins
(unacceptable in practice), or skip revocation checking (allowing stolen certificates to be used).
\end{itemize}

\subsection{Private keys in secure hardware}\label{sec:hardware}

The \emph{Fast Identity Online} (FIDO) Alliance, an industry consortium, has drafted a specification
for new user authentication protocol.~\cite{FIDOOverview, FIDOSpec} Its basic mechanism is that the
user registers a public key when signing up to an online service. Whenever the user wants to log in,
the service provides a challenge, and the user signs the challenge using their private key to prove
their identity.

Unlike TLS client certificates, FIDO is an application-layer protocol. We believe that the basic
components of FIDO (public key authentication at the application layer) are promising. However, much
of the FIDO specification is dedicated to issues of using hardware modules (\emph{authenticators})
to store private keys and perform signing operations. On the other hand, the specification largely
ignores the issue of revoking a key that has been lost or stolen.

Another hardware-based authentication project is Pico~\cite{Stajano11}, which also proves ownership
of a secret key by signing a challenge. Pico addresses the risk of key theft by using $k$-out-of-$n$
secret sharing across several hardware modules carried by the user, such as glasses, belt, wallet,
and jewellery. This approach is theoretically interesting, but the possibility of a large-scale
deployment seems remote.

Whilst the security characteristics of hardware modules are appealing, the user experience is quite
inconvenient at present. For example, a USB-based module cannot be used on a mobile device that has
no USB port. If the cryptographic module is built into the device, the authentication capability is
not portable across devices.

A small number of high-value services (such as online banking) can afford to significantly
inconvenience users in the name of security, but for the majority of online services, any
authentication method that is less convenient than a password is unlikely to gain adoption, even if
it is significantly more secure. We seek a solution that can be deployed today, which does not
inconvenience users, and which does not require new hardware to be developed.

\section{Octokey protocol overview}

Octokey is a user authentication protocol designed to meet the goals stated in the introduction. It
aims to provide good security in a software-only configuration, with cryptographic hardware modules
being an optional extra. It strives to be a \emph{trust-free} protocol: it is completely
decentralised, there is no dependency on any identity provider, and the extent to which PKI
certificate authorities need to be trusted is minimised as far as possible.

The basic operation of Octokey is inspired by SSH public key authentication, which is widely used
for shell access to remote servers. When using Octokey for the first time, a user must generate a
RSA keypair.\footnote{We hope that the approach could be extended to support other public-key
cryptosystems such as ECC.} When the user signs up to a service, or migrates from another
authentication method to Octokey, the user submits their public key to the service. This is
analogous to creating a password when a user first signs up to a service.

Whenever the user wishes to log in, they must prove ownership of the private key. A client does this
by requesting a challenge from the server, signing it using the private key, and submitting the
signature to the server. The details of this protocol are given below.

We can think of Octokey as a machine-to-machine authentication protocol, and it needs to be preceded
by a human-to-machine authentication step: for example, a password or biometric information can be
used by the client device to unlock or decrypt the private key. However, since this password or
biometric information is not sent over the network, this concern is orthogonal to the
machine-to-machine authentication. Any improvements in biometric sensors, for example, do not
require any changes to the Octokey protocol or any support by service providers.

If a password is used to decrypt the user's private key, they need only remember a single password
per device (or perhaps per keypair), but not a separate password for every service where they have
an account. We believe that a single password is an acceptable user experience.

\subsection{Authentication requests}

Each user has a RSA keypair $(n, d, e)$ where $n$ is the modulus, $d$ is the private exponent and
$e$ is the public exponent. When signing up to a service, the user submits their public key
$(n, e)$ to the service. The service stores the public key as part of the user record.

To log in, the client first requests a challenge $c$ from the service via a HTTP endpoint. It then
calculates $m = H(c, u, r)$ where $u$ is the URL of the service endpoint and $r$ is the user's
registered username. $H$ encodes its arguments as a binary string and then applies the
\textsc{EMSA-PSS-Encode} operation (hashing and padding) as defined in PKCS\#1.~\cite{PKCS1} From
that, the RSA signature can be calculated as $s = m^d \mod n$.

The client then constructs the authentication request, which is an encoding of the tuple
$(s, c, u, r, n, e)$. The authentication request is sent to the server as part of a HTTP request
over TLS, and is handled at the application layer. The server can verify the authentication request
by checking that all of the following are true:
\begin{itemize}
\item $m^d \mod n$ is a valid PKCS\#1 signature of the values $(c, u, r)$, checked against the
public key $(n, e)$.
\item $c$ is a valid challenge issued by this service, has not been used for login before, and is
not older than some maximum age. This prevents replay attacks.
\item $u$ is a valid URL for this service. A login request for an unknown URL must be rejected.
This prevents phishing-like attacks, whereby an attacker creates an imitation website under a
similar-looking URL and tricks the user into logging in.
\item $r$ is a registered username for this service, and $(n, e)$ is a public key for that user.
\end{itemize}
If the authentication request is successfully verified, the user is logged in by the server in the
usual manner, for example by setting an appropriate session cookie.

\subsection{Protection against key theft}

Many users have multiple devices (e.g. laptop, smartphone, tablet, game console) on which they need
to be able to log in to their accounts with online services. As described above, the private key
$(n, d, e)$ would need to be copied to each of those devices. If any of those devices is lost or
compromised, and an attacker can break the human-to-machine authentication step (perhaps due to a
weak password on the private key), the attacker could gain access to all of a user's accounts.

To mitigate this risk, we ensure that the private exponent $d$ is never stored on any device.
Instead, we split it into key fragments that are distributed amongst the user's devices. We use the
\emph{mediated RSA} (mRSA) scheme proposed by Boneh et al.~\cite{Boneh01} which uses the fact that
$$s = m^d = m^{d_a + d_b} = m^{d_a} m^{d_b} \mod n$$ provided that $d = d_a + d_b \mod \phi(n)$.

If two devices $a$ and $b$ each have a key fragment $d_a$ and $d_b$ respectively, and those
fragments sum to the private exponent $d$, then we call those devices \emph{paired}. In order to
generate a valid signature, any two paired devices need to collaborate. If device $a$ wants to
generate an authentication request, it can send $(c, u, r)$ to device $b$, which may first ask the
user to confirm whether they wish to log in to URL $u$ using username $r$. If yes, $b$ calculates
$H(c, u, r)^{d_b} = m^{d_b}$ and returns that value to $a$. Then $a$ can calculate
$s = m^{d_a} m^{d_b}$ by using its own key fragment $d_a$ and the value returned by $b$, and thus
log in.

This scheme ensures that the user can revoke a device's login capability if it is lost, stolen or
compromised: every device that is paired with the lost device can be instructed to delete the key
fragment corresponding to the lost device. When the paired fragments are deleted, the key fragments
on the lost device become useless. Thus, even if the human-to-machine authentication is weak, not
all is lost: the user only needs to revoke the lost device's key fragments faster than an attacker
can break the human-to-machine authentication.

A key could also be split into more than two fragments, which would protect against the case where
an attacker manages to steal two or more devices. However, this comes at the cost of a more
cumbersome user experience, so we concentrate on the two-fragment case for now.

We expect that users have at least three devices (e.g.\ a laptop, a smartphone and a remote store --
see below), that each device is paired with each of the others, and that each pairing uses
different, randomly selected key fragments. If the user has $k$ devices, they can lose up to $k-2$
devices and still be able to log in to services without resorting to a backup.

\subsection{Remote key fragment store}

Authenticating by using paired physical devices (e.g.\ a laptop and a smartphone) yields a similar
user experience to current 2-factor authentication solutions, whereby the user must fetch the phone
from their pocket, launch the appropriate app, and perform some kind of handshake. This is possible,
but distinctly less convenient for users than typing a password, so it is not the simple user
experience we are looking for.

However, there is a simple solution within the mRSA framework: one of the user's `devices' may be a
remote service on the internet. This service stores key fragments that are paired with each of the
user's physical devices, and responds to signing requests by performing the modular exponentiation
using its key fragments. This allows a user to authenticate with services using only one physical
device -- the coordination with the remote key fragment store happens automatically behind the
scenes.

The key fragment store need only be partially trusted. It cannot authenticate as the user without
the cooperation of one of the user's physical devices. From the point of view of a service verifying
the signature on an authentication request, the key fragment store does not even exist (unlike a
federated login system, where the relying party must trust the identity provider). The user only
needs to trust the key fragment store to not collude with attackers who steal devices, and to delete
key fragments when the user requires key revocation.

TODO Prevent denial of service due to rate limiting or due to an attacker revoking a user's active
devices.

\section{Key distribution}

To provision a new device for a user (i.e. to pair it with their existing devices), we need to
generate a new key split $(d_{i+1,a}, d_{i+1,b})$ such that
$d_{i+1,a} + d_{i+1,b} = d_{i,a} + d_{i,b} \mod \phi(n)$, without reassembling the entire private
exponent $d$ on any single device (and thus risking it being stolen), and without knowing $\phi(n)$.

Say the user has two existing devices $a$ and $b$, storing $d_{i,a}$ and $d_{i,b}$ respectively. We
now want to introduce a new device $c$ and pair it with $a$. To this end, $c$ generates a uniformly
distributed random number $d_{i+1,c}$ with $0 < d_{i+1,c} < 2^n$, and another uniformly distributed
random number $g$ with $0 < g < d_{i+1,c}$.

By exchanging a sequence of messages, the devices can arrive in the desired state. The algorithm
will be described in the full paper.

\subsection{Key rotation}
\subsection{Multiple keypairs per user}
\subsection{Security levels}
\subsection{Device-to-device communication}
\subsection{Threats}
Undetected malware on a compromised machine can secretly log in to the user's services. If it's not
noticed, it can't be revoked. But that is true of any software-based solution; only dedicated
cryptographic hardware can prevent this, and we don't want to rely on hardware, as discussed in the
introduction.

\subsection{Channel binding}

Question: how to bind the Octokey signature to the TLS channel in a mobile app, where the TLS
connection will probably get torn town when switching to the Octokey app? Set up OBC first (before
login)?

Origin-Bound Certificates (OBC)~\cite{Dietz12} should be used to protect the session cookie from
MITM attackers.

Note that channel binding does not protect against all MITM attack scenarios. For example, in an
e-commerce setting, an attacker who has stolen the service's private key, or who has fraudulently
obtained a certificate from a CA trusted by the user's browser, can still impersonate the service.
If the user does not notice the impersonation, and enters their credit card number to make a
purchase, the attacker is able to steal the credit card number. In situations like these -- where
the user is providing confidential information to the service -- the PKI still needs to be trusted.

However, in situations where the flow of confidential information is from the service to the
authenticated user, channel binding is an effective additional protection against MITM attacker
trying to steal that information.

Another attack vector is via JavaScript injected into the user's browser by a MITM attacker. Any
code that executes within the context of a document in the browser has full access to the contents
of that document. A page that is served to an authenticated user may download JavaScript code on
separate TLS connections, for example from a content delivery network (CDN). If a MITM attacker is
able to impersonate the CDN, they can inject arbitrary JavaScript into the browser.

To prevent this, the page should not directly execute JavaScript that is downloaded. Instead, it
should fetch JavaScript from the server using XMLHttpRequest over an authenticated TLS connection. 

\section{Key distribution}

To provision a new device for a user (i.e. to pair it with their existing devices), we need to
generate a new key split $(d_{i+1,a}, d_{i+1,b})$ such that
$d_{i+1,a} + d_{i+1,b} = d_{i,a} + d_{i,b} \mod \phi(n)$, without reassembling the entire private
exponent $d$ on any single device (and thus risking it being stolen), and without knowing $\phi(n)$.

Say the user has two existing devices $a$ and $b$, storing $d_{i,a}$ and $d_{i,b}$ respectively. We
now want to introduce a new device $c$ and pair it with $a$. To this end, $c$ generates a uniformly
distributed random number $d_{i+1,c}$ with $0 < d_{i+1,c} < 2^n$, and another uniformly distributed
random number $g$ with $0 < g < d_{i+1,c}$.

By exchanging a sequence of messages, the devices can arrive in the desired state. The algorithm
will be described in the full paper.

\subsection{Key rotation}
\subsection{Multiple keypairs per user}
\subsection{Security levels}
\subsection{Device-to-device communication}
\subsection{Threats}
Undetected malware on a compromised machine can secretly log in to the user's services. If it's not
noticed, it can't be revoked. But that is true of any software-based solution; only dedicated
cryptographic hardware can prevent this, and we don't want to rely on hardware, as discussed in the
introduction.

\subsection{Channel binding}

Question: how to bind the Octokey signature to the TLS channel in a mobile app, where the TLS
connection will probably get torn town when switching to the Octokey app? Set up OBC first (before
login)?

Origin-Bound Certificates (OBC)~\cite{Dietz12} should be used to protect the session cookie from
MITM attackers.

Note that channel binding does not protect against all MITM attack scenarios. For example, in an
e-commerce setting, an attacker who has stolen the service's private key, or who has fraudulently
obtained a certificate from a CA trusted by the user's browser, can still impersonate the service.
If the user does not notice the impersonation, and enters their credit card number to make a
purchase, the attacker is able to steal the credit card number. In situations like these -- where
the user is providing confidential information to the service -- the PKI still needs to be trusted.

However, in situations where the flow of confidential information is from the service to the
authenticated user, channel binding is an effective additional protection against MITM attacker
trying to steal that information.

Another attack vector is via JavaScript injected into the user's browser by a MITM attacker. Any
code that executes within the context of a document in the browser has full access to the contents
of that document. A page that is served to an authenticated user may download JavaScript code on
separate TLS connections, for example from a content delivery network (CDN). If a MITM attacker is
able to impersonate the CDN, they can inject arbitrary JavaScript into the browser.

To prevent this, the page should not directly execute JavaScript that is downloaded. Instead, it
should fetch JavaScript from the server using XMLHttpRequest over an authenticated TLS connection. 

\section{Delegated login by mobile device}
Browser and phone meet at rendezvous point (URL in QR code) / communicate over rendezvous channel

We must assume that the contents of the barcode can be read by an attacker: it may be
electromagnetically snooped~\cite{Kuhn05}, or an attacker may simply point a camera at the victim's
screen. In order to establish a secure channel between the smartphone and the desktop browser, it is
therefore not appropriate to embed a symmetric private key in the barcode.

The browser has a separate keypair, with the private key identifying the browser itself (not the
user). The QR code contains the URL of the rendezvous point, and the fingerprint of the browser's
public key. When the rendezvous channel is established, the browser sends a message to the client
signed with the browser's private key; and the client checks the signature, and checks that the
fingerprint of the public key that made the signature matches the fingerprint in the QR code. Thus
man-in-the-middling would require an attacker to actually modify the on-screen QR code, rather than
just eavesdrop it. If the attacker has enough access to the device that they are able to manipulate
what is displayed on the screen, then they can impersonate the user after logging in anyway (so all
bets are off for that user account on that site). The user still needs to make sure that they are
not being tricked into logging in to a different site from the one they thought they were logging in
to; this can be done by displaying the URL of the login page on the mobile phone when the user is
prompted to approve the authentication request. If the URL is not what the user was expecting, they
must decline the authentication request.


% TODO: OAuth-like thing by adding a third party's public key to your account

\bibliography{references}{}
\bibliographystyle{plain}
\end{document}
