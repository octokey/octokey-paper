\documentclass{acm_proc_article-sp}
\usepackage{cite}
\usepackage{hyperref}
\sloppy

\begin{document}
\toappear{This is an unpublished draft. Please don't distribute without our permission.}
\title{Octokey: Public key authentication for the web}
\numberofauthors{2}
\author{
    \alignauthor Martin Kleppmann \\ \email{martin@kleppmann.com}
    \alignauthor Conrad Irwin \\ \email{conrad.irwin@gmail.com}}
\maketitle

\begin{abstract}
We propose Octokey, a cryptographic protocol for authenticating users, designed as a viable
replacement for password authentication in consumer internet use cases. When a user signs up to a
service, they provide a public key. Clients subsequently authenticate by presenting a challenge
signed with the user's private key, similar to SSH public key authentication. We also describe
protocols which allow users to log in to their accounts on multiple devices, provision new devices,
and revoke keys on devices that are lost or stolen.
\end{abstract}

\bigskip

\emph{Note: We, the authors, have a background in developing commercial web applications. In the
field of crypography, we are merely keen amateurs. We are aware that amateur cryptography is prone
to catastrophic flaws, and we fully expect such flaws to be found in the ideas we are proposing.
However, we have experience designing real-world systems with good user experiences, a perspective
that we feel is often lacking in security research. We hope that this paper can stimulate a
discussion that brings together the perspectives from research and industry, and we would like to
solicit constructive feedback from all sides.}

\section{Introduction}

Despite their many well-known flaws, passwords are still by far the most commonly used
authentication system on the web.~\cite{Bonneau12} However, as users sign up for ever more services,
their use is becoming increasingly unsustainable: demanding that users choose passwords with
sufficient entropy, use a different password for every site, not write them down and even rotate
passwords periodically is unrealistic for most internet services.

With phishing attacks and leaks of password databases unfortunately commonplace, we seek a better
solution. In this paper we focus on authentication in a consumer internet context, such as social
media or e-commerce websites and mobile apps. For a majority of such use cases, an authentication
system with the following properties is desirable:

\begin{enumerate}
\item The system should minimize exposure to human error (such as falling for a phishing attack).
Whilst human error is unavoidable, especially amongst non-technical users, the system should be
designed so as to minimize the impact in the case of error.
\item In case a device is lost or stolen, the user should be able to easily revoke any credentials
stored on the device, to minimise the chances that an attacker can use a stolen device to gain
access to other systems.
\item It is not necessary, and often undesirable, for the system to authenticate the user as a
physical person. Many websites allow pseudonymous signup, which is desirable for reasons of privacy
or freedom of speech. The purpose of the authentication system is thus only to verify that the
current user is the same person as the one who originally signed up under a particular username; no
authentication is performed at signup.
\item The system should not depend on any trusted third party, to avoid risks arising from failure
of the third party (security compromise, going out of business, change of policies, etc).
\item The system should be easy for end-users to install and use, and easy for website owners to
deploy. Since user signup and user activity rates are important business metrics for many online
services, the system should make signup and login extremely easy and enjoyable for users.
\item Users should have the freedom to use the system on a variety of devices, including shared or
public computers that are only partially trusted, whilst minimizing their exposure to attackers.
For example, a user may choose log in to `unimportant' services (e.g. casual online games) on a
shared or public computer, knowing that their account on that service may be compromised; however,
they limit their use of `important' accounts (e.g. online banking) to trusted devices. Since there
is no generally agreed boundary between `important' and `unimportant' websites, the same
authentication system should be able to support both use cases.
\end{enumerate}

In the next section we discuss various authentication systems that are used on the web, and explain
why none of them meet these requirements. We then introduce Octokey, an authentication protocol
which addresses these issues. We believe that Octokey could be a viable solution to online
authentication in the near future, and we seek the information security community's feedback to
refine this proposal.

\section{Existing web authentication systems}
\subsection{Passwords}

Passwords have many of the desirable properties mentioned above: no dependence on any third party,
pseudonymity, simplicity of implementation, familiarity and cross-device compatibility. However,
remembering a large number of passwords is a big burden on users, and they are highly susceptible to
human error: use of weak passwords, reuse of the same password across different services, phishing
attacks and leaks of unhashed or weakly hashed password databases are sadly common.

Password managers built into web browsers, or external password manager products such as 1Password
and LastPass, make it feasible for users to maintain a strong, unique password for each service they
use. However, even when a unique password is used, an attacker who succeeds in stealing a password
(e.g. by phishing, malware, keylogging, man-in-the-middle attack or eavesdropping when it is
accidentally sent over an unencrypted connection) has access to that user account until the password
is changed, which is often a long time.

Password managers also have grave exposure to human error. For example, consider a user who wishes
to log in to an `unimportant' website on an untrusted computer. It is very tempting for the user to
type the master passphrase of their password manager into the untrusted computer, in order to
decrypt the password database. This has the effect of exposing the user's entire password database
to an attacker running malware on the untrusted computer.

\subsection{One-time passwords and two-factor authentication}

One-time passwords (OTPs) offer a solution for users who want to log in to their accounts on
untrusted or partially trusted devices. The user must download a list of one-time passwords on a
trusted device, or register a cryptographic device that generates a pseudorandom sequence of OTPs,
or register an email address or phone number to which OTPs can be sent on demand. Some services
allow use of OTPs as sole authentication mechanism, while others use them in conjunction with a
regular password (two-factor authentication).

The advantage of OTPs is that the exposure to attacks is limited to a single service, and ends when
the user clicks the logout button (assuming the session is correctly invalidated on the server, and
assuming that the attacker does not have a means of extending their privileges, e.g. by granting
themselves OAuth access to the account).

However, OTPs are so inconvenient to use that very few websites are willing to adopt them as their
only or primary authentication mechanism, and only the most security-conscious users are willing to
use them. Moreover, the common ``remember me on this device'' feature weakens two-factor
authentication (attackers just need to steal the ``remember me'' cookie as well as keylogging the
password).

OTPs sent on demand via email or SMS also have the effect of `outsourcing' the authentication to the
email provider or the mobile network provider, respectively: any attacker who can access the email
provider or SMS messages can also gain access to any service using on-demand OTPs. This essentially
makes them federated login schemes (see next section).

\subsection{Federated identity}

OpenID~\cite{OpenID} is probably the best-known attempt to remove the need for a separate password
on every service. Some identity providers, such as Google Federated Login~\cite{GoogleOpenID} and
Facebook Connect, combine the ideas of OpenID and OAuth~\cite{OAuth}. Mozilla Persona
(BrowserID)~\cite{Persona, BrowserID} uses a different protocol, but is similar in the way it
delegates authentication to the user's email provider (or a fallback service provided by Mozilla).

Such federated authentication protocols do not solve the fundamental problem of authentication: they
only delegate it to the identity provider, who must then use some other authentication method (most
commonly a password, possibly in conjunction with a hardware token). This means that all the
problems discussed above apply to the identity provider, with an additional privilege escalation
problem: if an attacker gains access to the user's account with the identity provider, they can
access any account associated with that identity. Like with a password manager, a single user error
can lead to all of a user's accounts being compromised.

Moreover, the relying party needs to trust the identity provider to correctly authenticate the user.
If the identity provider is compromised, the relying party has no way of detecting unauthorized
logins from that identity provider. If the identity provider experiences an outage or goes out of
business, users of that provider lose their ability to log in (unless they had previously set up
delegation of their identity URL, which is unrealistic to expect of non-technical users). When
different services are in competition with each other, it is typically not in one service's interest
to accept a competing service as identity provider.

For these reasons, OpenID login is typically only accepted by small services; major online services
rarely act as a relying party in OpenID. The competitive dynamic between services makes it unlikely
that a user will ever be able to use a single identity provider across all services.

\subsection{Client certificates}

TLS~\cite{TLS} provides a mechanism for a client to authenticate itself to the server using a X.509
certificate. The server may specify the certificate authority from which it is willing to accept
certificates. By calculating a signature over the TLS key exchange messages, client certificate
authentication also provides protection against man-in-the-middle attacks (a signature generated on
one TLS connection cannot be reused on another TLS connection).

Client certificates are a good solution in situations where the physical identity of the user is
important: certificates might be issued by a government to its citizens (for filing taxes online),
or by a company to its employees (for accessing internal systems). For example, the government of
Estonia issues certificates on smart cards to its citizens through the national ID card scheme, and
the certificates are widely used for authentication by banks, utility companies and other
organisations in Estonia.~\cite{Parsovs14}

Client certificate authentication can be performed by hardware tokens, e.g. using the PKCS \#11
\emph{Cryptoki} API. This provides some protection against private keys being stolen by malware.

Despite their advantages, client certificates are not widely used for consumer internet services.
Problems include:

\begin{itemize}
\item The user experience of installing and managing certificates is unfriendly in most web
browsers.
\item Web browsers do not offer a consistent mechanism for logging out or switching to another
account after logging in with a client certificate.
\item A website provider must decide which CAs to trust, or it must itself act as a CA in order to
avoid a trust dependency on a third party.
\item Pseudonymous usage of services is difficult if the certificate is issued by an external CA.
\item The certificate (including identifying personal details such as the user's real name) is sent
unencrypted during the TLS handshake, making it visible to passive network attackers.\footnote{This
can be mitigated by performing the authentication handshake during a TLS renegotiation, but that has
a negative performance impact, and makes the server configuration more complicated.}
\item Certificate validation can be difficult to deploy on the server, as TLS termination is
typically performed on a different server from the application server running business logic.
\item Certificate validation can be computationally expensive, creating a risk of denial of
service.~\cite{Parsovs14}
\item Revocation of certificates (which requires CRLs or OCSP~\cite{OCSP}) is easy to get wrong, and
a slowdown or outage of OCSP servers can impact a service's availability (creating another
third-party dependency).
\end{itemize}

\subsection{Private keys in secure hardware}

The \emph{Fast Identity Online} (FIDO) Alliance, an industry consortium, has drafted a specification
for new user authentication protocol.~\cite{FIDOOverview, FIDOSpec} Its basic mechanism is that the
user registers a public key when signing up to an online service. Whenever the user wants to log in,
the service provides a challenge, and the user signs the challenge using their private key to prove
their identity.

Unlike TLS client certificates, FIDO is an application-layer protocol. We believe that the basic
components of FIDO (public key authentication at the application layer) are promising. However, much
of the FIDO specification is dedicated to issues of using hardware modules (\emph{authenticators})
to store private keys and perform signing operations. On the other hand, the specification largely
ignores the issue of revoking a key that has been lost or stolen.

Another hardware-based authentication project is Pico~\cite{Stajano11}, which also proves ownership
of a secret key by signing a challenge. Pico addresses the risk of key theft by using $k$-out-of-$n$
secret sharing across several hardware modules carried by the user, such as glasses, belt, wallet,
and jewellery. This approach is theoretically interesting, but the prospect of a practical
deployment seems remote.

Whilst the security characteristics of hardware modules are appealing, the user experience is quite
inconvenient at present. For example, a USB-based module cannot be used on a mobile device that has
no USB port. A small number of high-value services (such as online banking) can afford to
significantly inconvenience users in the name of security, but for the majority of online services,
any authentication method that is less convenient than a password is unlikely to gain adoption, even
if it is significantly more secure.

It is conceivable that in future, authentication modules will be ubiquitously built into devices, so
that users can to use them without the friction of a special-purpose authenticator device. However,
we seek a solution that can be deployed today, without waiting for a hardware upgrade cycle.

% https://www.schneier.com/blog/archives/2014/09/security_of_pas.html
% https://crypton.io/
% https://www.lightbluetouchpaper.org/2014/10/02/pico-part-iv-somethings-you-have/
% http://news.engineering.utoronto.ca/bionym-raises-14-million-wearable-password-replacing-tech/
% https://www.digits.com/
% http://www.cl.cam.ac.uk/techreports/UCAM-CL-TR-817.pdf (already in Papers)

\section{Octokey protocol overview}\label{sec:protocol}

Octokey is a user authentication protocol designed to meet the goals stated in the introduction. It
aims to provide good security in a software-only configuration, with cryptographic hardware modules
being an optional enhancement. It strives to be a \emph{trust-free} protocol: it is completely
decentralized, there is no dependency on any identity provider, and the extent to which PKI
certificate authorities need to be trusted is minimized as far as possible.

The basic operation of Octokey is inspired by SSH public key authentication, which is widely used
for shell access to remote servers. When using Octokey for the first time, a user generates a RSA
keypair.\footnote{We hope that the approach could be extended to support other public-key
cryptosystems such as ECC.} When the user signs up to a service, or migrates from another
authentication method to Octokey, the user submits their public key to the service. This is
analogous to creating a password when a user first signs up to a service.

Whenever the user wishes to log in, they must prove ownership of the private key. A client does this
by requesting a challenge from the server, signing it using the private key, and submitting the
signature to the server. The details of this protocol are given below.

We can think of Octokey as a machine-to-machine authentication protocol, and it needs to be preceded
by a human-to-machine authentication step: for example, a password or biometric information can be
used by the client device to unlock or decrypt the private key. However, since this password or
biometric information is not sent over the network, this concern is orthogonal to the
machine-to-machine authentication. Any improvements in biometric sensors, for example, do not
require any changes to the Octokey protocol or any support by service providers.

If a password is used to decrypt the user's private key, they need only remember a single password
per device, but not a separate password for every service where they have an account. We believe
that a single password (like in a password manager) is an acceptable user experience.

\subsection{Challenges}\label{sec:challenges}

A service that accepts Octokey login must generate challenges, which are then signed by clients (see
section~\ref{sec:mandates}). The client does not interpret challenges, but treats them as opaque
byte strings. We propose that a service constructs a challenge from a wall-clock timestamp $t$, a
nonce $x$, and a secret key $K$ that is known only to the service:
$$c = t \concat x \concat \mathrm{HMAC}(K, t \concat x)$$
The symbol $\concat$ denotes encoding and concatenating the values into a byte string. When a client
submits a signed challenge to the service, the service can verify its validity by checking that all
of the following are true:
\begin{itemize}
\item $\mathrm{HMAC}(K, t \concat x)$ is a valid HMAC using the secret key K. This ensures that the
challenge was issued by this service.
\item $t$ is a recent timestamp (e.g. no more than 5 minutes old). This is a guard against offline
brute-force attacks.
\item $x$ has not been used within the time interval $t$. This prevents replay attacks.
\end{itemize}

\subsection{Authentication mandates}\label{sec:mandates}

Each user has a RSA keypair $(n, d, e)$ where $n$ is the modulus, $d$ is the private exponent and
$e$ is the public exponent. A key also has an expiry date $\mathit{ex}$, which is chosen when the
key is generated, and which is used to rotate the user's keypair periodically (see
section~\ref{sec:rotation}). The service may have multiple public keys on record for a user, and
should allow any one of those keys to authenticate as the user.

To log in or sign up, the user's client first requests a challenge $c$ from the service via a HTTP
endpoint. It then calculates $m = H(c \concat u \concat r \concat \mathit{ex})$ where $u$ is the URL
of the service endpoint, $r$ is the user's registered username, and $\mathit{ex}$ is the expiry date
of the key. $H$ is shorthand for the \textsc{EMSA-PSS-Encode} operation (hashing and padding) as
defined in PKCS\#1.~\cite{PKCS1} The RSA signature can then be calculated as $s = m^d \mod n$.

The client then constructs the \emph{mandate}, which is an encoding of the RSA-signed message and
the user's public key: $$\mathit{mandate} = s \concat c \concat u \concat r \concat \mathit{ex} \concat n \concat e.$$
The mandate is sent to the server as part of a HTTP request over TLS, and is handled at the
application layer. The server can verify the mandate by checking that all of the following are true:
\begin{itemize}
\item $s$ is a valid PKCS\#1 signature of the message $c \concat u \concat r \concat \mathit{ex}$,
checked against the public key $(n, e)$.
\item $c$ is a valid challenge, as defined in section~\ref{sec:challenges}.
\item $u$ is a valid URL for this service. A mandate for an unknown URL must be rejected. This
prevents phishing-like attacks, whereby an attacker creates an imitation website under a
similar-looking URL and tricks the user into logging in.
\item For a signup request, the username $r$ is not yet taken. For a login request, $r$ is a
registered username for this service, and $(n, e, \mathit{ex})$ is a public key for that user.
\item $\mathit{ex}$ is in the future.
\end{itemize}

If a login mandate is successfully verified, the user is logged in by the server in the usual
manner, for example by setting an appropriate session cookie. If a signup mandate is successfully
verified, the signup flow continues as usual, e.g. asking for the user's name and verifying their
email address.

\subsection{Protection against key theft}\label{sec:revocation}

Many users have multiple devices (e.g. laptop, smartphone, tablet, game console) on which they need
to be able to log in to their accounts with online services. As described above, the private key
$(n, d, e)$ would need to be copied to each of those devices. If any of those devices is lost or
compromised, and an attacker can break the human-to-machine authentication step (perhaps due to a
weak password on the private key), the attacker could gain access to all of a user's accounts.

To mitigate this risk, we ensure that the private exponent $d$ is never stored on any one device.
Instead, we split it into key fragments that are distributed among the user's devices. We use the
\emph{mediated RSA} (mRSA) scheme~\cite{Boneh01, Kutyiowski12} which is based on the fact that
$$s = m^d = m^{d_a + d_b} = m^{d_a} m^{d_b} \mod n$$ provided that $d = d_a + d_b \mod \phi(n)$.

If two devices $a$ and $b$ each have a key fragment $d_a$ and $d_b$ respectively, and those
fragments sum to the private exponent $d$, then we call those devices \emph{paired}. In order to
generate a valid signature, any two paired devices need to collaborate. If device $a$ wants to
generate a mandate, it can send a signing request $\mathit{req}$ to device $b$:
$$\mathit{req} = H(c \concat u \concat r \concat \mathit{ex}) \concat n \concat e$$
where the public key $(n, e)$ indicates which key should be used, in case device $b$ stores multiple
keys. Device $b$ then uses its key fragment $d_b$ to calculate a response:
$$\mathit{resp} = H(c \concat u \concat r \concat \mathit{ex})^{d_b} = m^{d_b}$$
and returns $\mathit{resp}$ to $a$. Now, $a$ can calculate $s = m^{d_a} m^{d_b}$ by using its own
key fragment $d_a$ and $\mathit{resp}$, construct a mandate with a valid signature, and thus log in.

If a device is lost, stolen or compromised, this scheme allows the user to revoke that device's
login capability: every device that is paired with the lost device must be instructed to delete the
key fragment from the pairing with the lost device. When all the paired fragments have been deleted,
the key fragments on the lost device become useless. Thus, even if the human-to-machine
authentication is weak, not all is lost: the user only needs to revoke the lost device's key
fragments faster than an attacker can break the human-to-machine authentication.

A key could also be split into more than two fragments; we discuss this in section~\ref{sec:multiway}.

\subsection{The mediator service}\label{sec:mediator}

Authenticating by using paired physical devices (e.g.\ a laptop and a smartphone) yields a similar
user experience to current 2-factor authentication solutions, whereby the user must fetch the phone
from their pocket, launch the appropriate app, and perform some kind of handshake. This is
possible, but distinctly less convenient for users than typing a password, so it is not the simple
user experience we are looking for.

However, there is a simple solution within the mRSA framework: one of the user's `devices' may be a
remote service on the internet, which we call the \emph{mediator}. This service stores key fragments
that are paired with each of the user's physical devices, and responds to signing requests by
performing the modular exponentiation using its key fragments. This allows a user to authenticate
with services using only one physical device -- the coordination with the mediator happens
automatically behind the scenes.

The mediator need only be partially trusted. It cannot authenticate as the user without the
cooperation of one of the user's physical devices. The user only needs to trust the mediator to not
collude with attackers who steal devices, and to correctly delete key fragments when the user
requires key revocation. The user's privacy is protected by hashing the message
$c \concat u \concat r \concat \mathit{ex}$ before sending it to the mediator, so it does not learn
which services the user is logging in to, or which usernames they are using.

From the point of view of a service that accepts Octokey login, the mediator does not even exist: a
service simply verifies the RSA signature on a mandate, and does not care how that signature was
constructed. This is in contrast to federated login systems, where the relying party must trust the
identity provider.

\subsection{Rate limiting password guesses}\label{sec:ratelimit}

Besides enabling key revocation, mRSA can also be used to strengthen the human-to-machine
authentication step against offline attacks.

For example, say the key fragment on a device is encrypted with a symmetric key derived from a
password.\footnote{This discussion also applies to other human-to-machine authentication methods,
for example an encryption key that is derived from biometric measurements.} Consider an attacker who
has stolen this encrypted fragment. In order to brute-force the password, the attacker needs a way
of determining whether a password guess is correct. However, a key fragment is just a uniformly
distributed random number; by itself, the correctly decrypted key fragment is indistinguishable from
the garbage that results from trying to decrypt with the wrong password.\footnote{If the encryption
uses a block cipher, the key fragment must be padded with random bits up to the block size. Padding
with a predictable bit pattern would leak information on whether a password guess was correct.}

Assuming the attacker has no other key fragments, they can only determine whether the password guess
was correct by communicating with the mediator and testing whether they are able to construct a
valid PKCS\#1 signature. This gives us an opportunity to rate-limit password guessing attempts: if
the mediator receives too many requests based on an incorrect password, it can block further
attempts and advise the user to revoke the device pairing.

In order to achieve this, we must design the protocol such that an attacker must communicate with
the mediator for every password attempt, but without revealing the password or the decrypted key
fragment to the mediator. We can do this as follows:

Say the key fragment $d_a$ has been encrypted with password $\mathit{pass}$, and the attacker has
stolen the encrypted fragment
$$\mathit{efrag} = \mathrm{encrypt}(\mathrm{PBKDF2}(\mathit{pass}), d_a).$$
The attacker now guesses $\mathit{pass}^\prime$ and computes a guess $d_a^\prime$ of the plaintext:
$$d_a^\prime = \mathrm{decrypt}(\mathrm{PBKDF2}(\mathit{pass}^\prime), \mathit{efrag})$$
To check whether $d_a^\prime = d_a$ the attacker needs to contact the mediator where $d_b$ is held.

We modify the mediator's request processing as follows:
\begin{enumerate}
\item In addition to the signing request $\mathit{req}$, the client is required to submit a
signature $s_\mathit{req}$:
\begin{align*}
    \mathit{req} &= H(c \concat u \concat r \concat \mathit{ex}) \concat n \concat e \\
    s_\mathit{req} &= H(\mathit{req} \concat \mathit{cb})^{d_a^\prime}
\end{align*}
where $\mathit{cb}$ is the \texttt{tls-unique} channel binding~\cite{ChannelBinding}
of the TLS connection between the client and the mediator. Channel binding is further discussed in
section~\ref{sec:channelbinding}.
\item The mediator queries its database for a key fragment $d_b$ belonging to the user with public
key $(n, e)$. If the device uses a TLS client certificate when connecting to the mediator (see
section~\ref{sec:channels}), it can retrieve the key fragment for the authenticated client device.
\item Using the channel binding $\mathit{cb}^\prime$ of the TLS connection's server side, the
mediator computes
$$s_\mathit{req} H(\mathit{req} \concat \mathit{cb}^\prime)^{d_b} =
  H(\mathit{req} \concat \mathit{cb})^{d_a^\prime} H(\mathit{req} \concat \mathit{cb}^\prime)^{d_b}$$
and checks whether the result is a valid PKCS\#1 signature of
$\mathit{req} \concat \mathit{cb}^\prime$ for the user's public key $(n, e)$. This check succeeds if
$d_a^\prime = d_a$ (i.e. the user's password was correct), and if $\mathit{cb}^\prime = \mathit{cb}$
(preventing MITM and replay attacks).
\item If the signature is valid, the mediator computes
$\mathit{resp} = H(c \concat u \concat r \concat \mathit{ex})^{d_b}$ as before, and returns it to
the client. If the signature is not valid, the mediator returns ``bad signature''. A
password-guessing attacker learns that the password guess $\mathit{pass}^\prime$ was incorrect, but
otherwise nothing is revealed that would help them guess the password.
\end{enumerate}

Note that although the mediator computes an RSA signature using the user's private key, the value
being signed ($\mathit{req} \concat \mathit{cb}$) cannot be used to construct a mandate, so the
mediator cannot log in to services on the user's behalf.

This protection against password guessing only works if the attacker does not have any knowledge of
previous requests to the mediator. If the attacker knows $x^{d_a}$ (a request) or $x^{d_b}$ (a
response) for any $x$, they can brute-force the password without contacting the mediator, and thus
circumvent the rate-limiting.  It is therefore important that communication with the mediator is
protected from eavesdropping (using TLS) and is not logged on the device.

\subsection{Channel binding and preventing MITM}\label{sec:channelbinding}

TLS connections are susceptible to man-in-the-middle attacks -- using forged TLS certificates, due
to users ignoring warnings about invalid certificates, or due to malware. Such incidents have been
observed in practice.~\cite{Huang14, Adkins11} An attacker who succeeds in establishing a MITM
position can steal signed mandates, and use them to impersonate the user. Can we secure the
connection between the device and the service against MITM?

Some authentication methods such as SCRAM-SHA-1-PLUS~\cite{SCRAM} use \emph{channel binding} to
prevent MITM attacks. For example, the \texttt{tls-unique} channel binding type works by hashing the
handshake messages that established the TLS connection: in a direct connection, server and client
obtain the same hash value, but if the connection was terminated and restarted by a MITM, the server
and client's values differ. If the client incorporates this hash value into the mandate (such that
it cannot be changed by the MITM), and the server checks that it equals the server-side view of the
connection, then a mandate is rendered invalid by the presence of a MITM.

Origin-Bound Certificates~\cite{Dietz12} generalize this idea to the web, creating a channel
binding that is not tied to a particular TCP connection. When a client first connects to a
particular domain name, it automatically creates a self-signed TLS certificate (without any user
interaction) and presents it to the server. This certificate does not have any authentication
purpose, but the fingerprint of the certificate could be incorporated into a mandate as a channel
binding.

APIs for accessing TLS channel bindings and creating TLS certificates are currently not readily
available in web browsers and in HTTP server implementations. Thus, while channel binding for
Octokey mandates would yield a significant improvement in security, it would also make the protocol
much harder to deploy, both on the client and on the server side. We therefore propose making it an
optional extension.\footnote{For the communication between Octokey clients and mediator, as
described in section~\ref{sec:ratelimit}, we do require channel binding. As this protocol only needs
to be implemented in the Octokey software, not by every service that accepts Octokey as
authentication mechanism, the burden of deployment is much smaller.}

An alternative effort to prevent MITM is the Certificate Transparency project~\cite{CertTrans},
which aims to strengthen trust in the PKI by providing a public audit log of issued certificates,
and rejecting certificates that do not appear in the log. Certificate Transparency does not protect
against attackers who have stolen a website's private key (or governments, which can obtain the
private key with a court order), but it does prevent more casual kinds of MITM attacks, so it may be
sufficient.

\section{Key revocation}

What happens if one or more of the user's trusted devices are stolen? The passphrase with which the
key is encrypted can only delay an attacker from decrypting it, not reliably prevent decryption. So
it needs to be possible to revoke a device's ability to authenticate as a particular user.

The easiest way would be for each of a user's devices to have a different keypair and to register
all of a user's public keys with each of the services they want to log in to. Then revocation could
be implemented in two ways:

\begin{itemize}
\item On every authentication request, or periodically, the website checks a key revocation list at
a designated URL. This approach is problematic because it is not secure by default: a buggy or
missing implementation of revocation checking is likely to go unnoticed for a long time, and so an
attempted revocation may have no effect. It also raises questions about the server hosting the
revocation list: it must be trusted to always return the correct list of revoked keys, otherwise it
could be abused for denial of service (by marking too many keys as revoked) or unauthorized access
(by not omitting revoked keys from the list). Users also need some mechanism for securely adding
keys to the revocation list, which introduces an authentication bootstrapping problem.
\item Alternatively, when a user wishes to revoke a key, they use a different key on a different
device (or a backup key stored in a secure location) to authenticate with every service that it has
access to, and ask each service to remove the compromised public key from the list of authorized
keys for that user. This can be done programmatically if each website provides an API for
manipulating the authenticated user's public keys. However, it may still be slow (the user may have
accounts on hundreds of different websites) and error-prone (websites may be temporarily unavailable
or not implement the API correctly).
\end{itemize}

We propose a different approach to key revocation, which does not depend on the websites'
implementation of any protocol or API, and which allows revocation to take instant effect.

Instead of using a different keypair on each of a user's devices, we propose that a user should have
only one keypair. That keypair is never stored on any one device, even in encrypted form; instead,
the private key is split, each device only stores a portion of the private key, and the portions
from two devices must be combined in order to obtain the entire private key. One portion of the key
without the other portion is useless.

Boneh \emph{et al.}~\cite{Boneh01} describe the mRSA (mediated RSA) algorithm, which is well suited
in this application, because it allows an RSA signature to be incrementally generated on several
different devices, each holding a portion of the private key, without the complete private key ever
being present on any one of the devices. This reduces the risk of the private key being stolen.

A private key $p$ may be partitioned in several different ways, e.g. $p = a_1+a_2 = b_1+b_2 =
c_1+c_2$. If a device containing key portions $a_1$ and $b_1$ is lost or stolen, then to revoke that
device's access, the user must delete key portions $a_2$ and $b_2$ from any devices on which they
may be stored.

When two devices $D_1$ and $D_2$ store corresponding parts of the private key, e.g. $a_1$ and $a_2$
respectively, then we may say that $D_1$ and $D_2$ are \emph{paired}. Any two devices controlled by
the user may be paired. Of course, keys can be partitioned into three or more portions, but we
consider two portions to provide a good compromise between revokability and convenience.

For an authentication request to be signed, any two devices that have been paired must cooperate.
Typically, but not necessarily, one of the two devices is the device on which the user wants to log
in. The other device may be a device physically close to them (e.g. a smartphone), or a designated
remote server (that needs to be partially trusted, as discussed below).



\section{Provisioning new devices}

Question: how does Octokey know that a certain website URL is the same service as a certain iOS
bundle ID, which is the same service as an Android app signed with a certain public key?

\section{Delegated login by mobile device}
Browser and phone meet at rendezvous point (URL in QR code) / communicate over rendezvous channel

We must assume that the contents of the barcode can be read by an attacker: it may be
electromagnetically snooped~\cite{Kuhn05}, or an attacker may simply point a camera at the victim's
screen. In order to establish a secure channel between the smartphone and the desktop browser, it is
therefore not appropriate to embed a symmetric private key in the barcode.

The browser has a separate keypair, with the private key identifying the browser itself (not the
user). The QR code contains the URL of the rendezvous point, and the fingerprint of the browser's
public key. When the rendezvous channel is established, the browser sends a message to the client
signed with the browser's private key; and the client checks the signature, and checks that the
fingerprint of the public key that made the signature matches the fingerprint in the QR code. Thus
man-in-the-middling would require an attacker to actually modify the on-screen QR code, rather than
just eavesdrop it. If the attacker has enough access to the device that they are able to manipulate
what is displayed on the screen, then they can impersonate the user after logging in anyway (so all
bets are off for that user account on that site). The user still needs to make sure that they are
not being tricked into logging in to a different site from the one they thought they were logging in
to; this can be done by displaying the URL of the login page on the mobile phone when the user is
prompted to approve the authentication request. If the URL is not what the user was expecting, they
must decline the authentication request.


% TODO: OAuth-like thing by adding a third party's public key to your account

\bibliography{references}{}
\bibliographystyle{plain}
\end{document}
