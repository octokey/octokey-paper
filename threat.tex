\section{Threat model and trade-offs}\label{sec:threat}

In this section, we discuss the assumptions we have made with regard to attackers' capabilities,
some aspects which require careful implementation, and some areas where different concerns need to
be traded off against each other.

\subsection{Malware and untrusted software}\label{sec:malware}

Most authentication schemes in a consumer context are based on the principle that once a user has
authenticated, they have unlimited access to their account: using different credentials for
different user actions is considered too inconvenient.\footnote{An exception are some online banking
websites, which require explicit authorization of each payment action, using a hardware token.}

Consequently, once a user has authenticated with a service on a device, the software on that device
can in principle do anything it wants to the user account. That is true no matter which
authentication method is used: if there is malware which can read files, log keystrokes and steal
session cookies, or if an attacker can obtain remote root access to a device, any services in use on
that device can be compromised.

With Octokey, there is the additional risk of the device's private key (for inter-device
communication) and mRSA key fragment being stolen. If the device has a built-in hardware security
module, the actual keys are protected, but malware can nevertheless sign arbitrary authentication
requests, because it is almost indistinguishable from legitimate user software. Thus, if malware is
detected on a device, the user must immediately revoke that device using another enrolled device.

In sandboxed environments (e.g. websites in web browsers, apps on some mobile OSes), mutually
untrusting applications are somewhat protected from each other. In this case, there needs to be an
explicit way of passing authentication requests between the Octokey application (which manages the
keys) and the websites and apps that require authentication. This mechanism must be implemented
carefully, ensuring that a website or application can only obtain a signed authentication request
for its own URL, and not some other one.

\subsection{Physical device theft and loss}\label{sec:theftloss}

We assume that the private key material on a device is protected by a human-to-machine
authentication step, e.g. encrypted with a password. This is not much use against malware, but it
does help against an attacker who physically steals a device (or a backup of a device's filesystem).
In section~\ref{sec:ratelimit} we discussed a technique for strengthening this step against brute
force and dictionary attacks.

If the user has enrolled another device, they can use that device to revoke the stolen device. If
the revocation happens before the human-to-machine authentication is broken, no harm is done. If the
device is not revoked in time, the attacker can sign arbitrary authentication requests, and steal
the key by enrolling another device and pairing it with the stolen device. The risk of key theft can
be mitigated by prompting the user on several devices before enrolling another device (provided that
the user has previously enrolled several devices).

If an attacker steals two devices that are paired with each other, the risk of key theft is much
greater, because the human-to-machine authentication is the only remaining protection: deleting key
fragments from the remote store is not sufficient to prevent the attacker from recovering the key.
The technique from section~\ref{sec:ratelimit} also no longer applies, because the encryption can be
brute-forced offline. This means there is a trade-off between security and reliability: pairing
physical devices with each other provides redundancy in case the remote key fragment store is
unavailable, but it increases the risk of the private key being stolen.

Another risk is that the user may lose access to their services: perhaps an attacker who gains
control of a device uses it to revoke the user's legitimate devices (thus preventing the user from
stopping the attacker); perhaps the user has only enrolled one or two physical devices, and they are
stolen simultaneously; perhaps the user's house burns down and all of their devices are lost. To
allow recovery from such situations, we recommend that users print off their private key on paper
and store it in a safe place.

An open question is whether the recovery key on paper should be the entire key, or a key fragment
paired with the fragment store. The advantage of making it a fragment is that the user can revoke it
in case the piece of paper is lost or stolen; the disadvantage is that an attacker who gains control
of a device can also revoke it, and thus lock out the legitimate user. On balance, it is probably
better to make it irrevocable (see also section~\ref{sec:rotation}).

\subsection{Network attacks}\label{sec:netattack}

A signed authentication request is tied to a particular server URL, so it cannot be reused with
another service (assuming that services verify the URL in authentication requests correctly). This
makes Octokey resilient to phishing attacks.

However, a signed authentication request is not tied to a particular client device, unless channel
binding (section~\ref{sec:channelbinding}) is used. This means that if an attacker obtains a signed
authentication request by eavesdropping, they can use it to authenticate as the user on that
particular service.

We rely on regular TLS to provide secrecy of the authentication request. The Octokey software
should, if possible, prevent users from accidentally sending an authentication request over an
unencrypted connection. Server certificate validation and HTTP Strict Transport Security (HSTS)
should be used to avoid simple MITM attacks with invalid certificates. We expect that Certificate
Transparency will help prevent more sophisticated MITM attacks with fraudulently issued
certificates. If the client is a native app, the service's public key fingerprint should be embedded
in the app and checked when connecting to the service (public key pinning).

For inter-device communication between enrolled Octokey devices, TLS with mutual authentication and
public key pinning is used, which makes it resistant to MITM without requiring CA certificates. The
public key fingerprints are distributed through a visual channel (using a 2D barcode), which is much
harder to tamper with than a network connection (see also section~\ref{sec:barcode-intercept}).

\subsection{2D barcode interception}\label{sec:barcode-intercept}

In the flow for enrolling a new device (section~\ref{sec:newdevice}), and with delegated login
(section~\ref{sec:delegation}), we proposed displaying a 2D barcode on the screen of one device, and
scanning it with the camera of another. We cannot assume confidentiality of these barcodes: an
attacker may snoop it electromagnetically~\cite{Kuhn05}, or simply point a camera at the victim's
screen.

An attacker can thus connect to the URL in the barcode, and enroll the new device to the attacker's
account, instead of the user's own account as intended. With delegated login, the attacker can get
the user to log in with the attacker's key rather than the user's own key. This does not compromise
the user's key, but if the user logs in to the wrong account and doesn't realize it, they may
inadvertently disclose sensitive information to the attacker while using the account.

It is not clear whether this would be a problem in practice. If it is, a mutual authentication step
could be added to the flows for enrolling a new device and for delegated authentication (for
example, verifying a 2D barcode in the other direction). However, this makes the process more
complicated for users, especially when trying to delegate authentication to a device which has no
camera, so the additional verification step should probably be optional.

\subsection{2D barcode phishing}\label{sec:barcode-phishing}

When enrolling a new device or performing delegated authentication, we assume that an attacker
cannot trick a user into scanning a different barcode from the one they intended --- i.e. we assume
that the visual channel between the barcode-displaying and the barcode-scanning device provides
integrity (but not confidentiality). We assume that malware (section~\ref{sec:malware}) would be the
only way for an attacker to manipulate what is displayed on screen, or manipulate the camera signal.

However, a real risk is that a malicious website or app displays a barcode that originates from an
attacker-controlled device, and tricks the user into scanning that barcode and granting the attacker
unwanted access. In this scenario we must rely on well-written warning messages and user education
to ensure the user really understands what they are doing.

For users who have already set up multiple physical devices, as an additional safeguard against
accidentally pairing with an attacker-controlled device, we can require that enrolling a new device
requires user approval on a quorum (e.g. majority) of existing devices. This does not require
additional cryptographic algorithms, but can simply be implemented as a policy in the Octokey
software.

\subsection{Online services}\label{sec:store-security}

The remote key fragment store and the store-and-forward service are not strictly required from a
cryptographic point of view, since the user could in principle use Octokey exclusively with physical
devices and direct device-to-device communication. However, they are very useful for providing a
good user experience, allowing a user to authenticate using only one physical device, and making
inter-device communication work ``out of the box''.

From a service owner's point of view, the key fragment store does not need to be trusted (unlike an
OpenID identity provider, for example), because a signed authentication request is simply an RSA
signature in any case. Thus, the choice of key fragment store and store-and-forward services is
entirely up to the user. We propose that a public service should be provided for free, operated by a
nonprofit foundation. However, technically sophisticated users or organizations may choose to
operate their own fragment stores, if they wish.

Operating a fragment store is not an easy task, since it is likely to become a popular attack
target: some attackers may want to steal the key fragments and other user data, whereas others may
try to disrupt the availability of the service by flooding it with traffic. The operators of the
fragment store will need expertise in dealing with such adversaries.

In the worst case, if attackers succeed in obtaining a copy of all the key fragments in the remote
store, the operators of the store will have to inform all users, and the users will have to re-pair
all of their devices with a new fragment store. The inter-device communication protocol should have
a mechanism by which a store operator can force all users to re-pair. Such a break-in would not
\emph{per se} expose private keys, but any stolen devices that are not re-paired with a new store
are much more likely to result in key theft.

The store-and-forward service is less critical for security, since it cannot read the contents of
messages it delivers, and it cannot tamper with them. It can only drop messages instead of
forwarding them, and thus interrupt the inter-device communication. To guard against this
possibility, physical devices should also try to connect through local communication channels when
possible, and thus exchange any messages that may have been lost.

\subsection{Key rotation}\label{sec:rotation}

Requires printing off a new recovery key every month?

\subsection{Multiple keypairs per user}

% How does Octokey know that a certain website URL is the same service as a certain iOS
% bundle ID, which is the same service as an Android app signed with a certain public key?
% If each URL/bundle ID/app ID has a separate keypair, there needs to be some way of tying them
% together.

\subsection{Multi-way key splitting}
\subsection{Security levels}

% Sync login history across devices, and compare to login history reported by server, to detect
% use of a stolen key


\subsection{Recovering from key theft}

Prevent denial of service due to rate limiting or due to an attacker revoking a user's active
devices.

In the worst case, if an attacker manages to get the entire private key (e.g. by stealing or
compromising two devices that are paired, and breaking the human-to-machine authentication step),
the user's last resort is to generate a new key and update their accounts on all services, adding
the new public key and removing the old one. Unfortunately, the same key-swapping can be performed
by the attacker to lock out the legitimate user.

Therefore, perhaps changing a user's public key on a service should require an additional hurdle,
e.g. using a recovery key that is only stored on paper but not electronically? Or a key that is
split 3 ways? But that would make key rotation difficult.
