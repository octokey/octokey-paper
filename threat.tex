\section{Threat model and trade-offs}\label{sec:threat}

Undetected malware on a compromised machine can secretly log in to the user's services. If it's not
noticed, it can't be revoked. But that is true of any software-based solution; only dedicated
cryptographic hardware can prevent this, and we don't want to rely on hardware, as discussed in the
introduction.

For software-only implementations of Octokey, we assume that there is no malware running on devices
that store key fragments. If an attacker can read the key fragment on a device and also break the
human-to-machine authentication step (e.g. by keylogging the password with which the key fragment is
encrypted), then the attacker is almost indistinguishable from a legitimate user.

The key fragment store is likely to be an attack target. If it is compromised, all users need to
re-pair with a new key fragment store (otherwise the ability to revoke keys is lost). For this
reason, the protocol should allow forced re-pairing within a deadline.

Denial of service attacks on the key fragment store are also likely.

Pair all physical devices with each other? Advantage: redundancy in case remote store has an outage.
Disadvantage: if an attacker steals two paired devices (e.g. taking both your laptop and your
smartphone in a robbery), mRSA revocation is not possible, so the human-to-machine authentication
step is the only remaining protection of the private key.

Does revocation still work if the user has only one device (besides the remote service)?
Need to assume everyone has two physical devices?

Does the remote key fragment store need to authenticate signing requests?

The remote key fragment store sees plaintext URLs and challenges, so it could track a user's
activity. Is this ok? Advantage: allows more granular limiting of abuse, e.g. rate limiting login
attempts on one website without affecting other websites.

The user should print off their entire private key, and store it in a safe place (e.g. bank vault).
This gives them a recovery method in case all of their devices are simultaneously destroyed (house
burns down).

Question: how does Octokey know that a certain website URL is the same service as a certain iOS
bundle ID, which is the same service as an Android app signed with a certain public key?

\subsection{Denial of service}

Prevent denial of service due to rate limiting or due to an attacker revoking a user's active
devices.

In the worst case, if an attacker manages to get the entire private key (e.g. by stealing or
compromising two devices that are paired, and breaking the human-to-machine authentication step),
the user's last resort is to generate a new key and update their accounts on all services, adding
the new public key and removing the old one. Unfortunately, the same key-swapping can be performed
by the attacker to lock out the legitimate user.

Therefore, perhaps changing a user's public key on a service should require an additional hurdle,
e.g. using a recovery key that is only stored on paper but not electronically? Or a key that is
split 3 ways? But that would make key rotation difficult.
