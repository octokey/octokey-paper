\section{Threat model and trade-offs}\label{sec:threat}

Undetected malware on a compromised machine can secretly log in to the user's services. If it's not
noticed, it can't be revoked. But that is true of any software-based solution; only dedicated
cryptographic hardware can prevent this, and we don't want to rely on hardware, as discussed in the
introduction.

For software-only implementations of Octokey, we assume that there is no malware running on devices
that store key fragments. If an attacker can read the key fragment on a device and also break the
human-to-machine authentication step (e.g. by keylogging the password with which the key fragment is
encrypted), then the attacker is almost indistinguishable from a legitimate user.

The key fragment store is likely to be an attack target. If it is compromised, all users need to
re-pair with a new key fragment store (otherwise the ability to revoke keys is lost). For this
reason, the protocol should allow forced re-pairing within a deadline.

Denial of service attacks on the key fragment store are also likely.

Pair all physical devices with each other? Advantage: redundancy in case remote store has an outage.
Disadvantage: if an attacker steals two paired devices (e.g. taking both your laptop and your
smartphone in a robbery), mRSA revocation is not possible, so the human-to-machine authentication
step is the only remaining protection of the private key.

Does revocation still work if the user has only one device (besides the remote service)?
Need to assume everyone has two physical devices?

Does the remote key fragment store need to authenticate signing requests?

The remote key fragment store sees plaintext URLs and challenges, so it could track a user's
activity. Is this ok? Advantage: allows more granular limiting of abuse, e.g. rate limiting login
attempts on one website without affecting other websites.

The user should print off their entire private key, and store it in a safe place (e.g. bank vault).
This gives them a recovery method in case all of their devices are simultaneously destroyed (house
burns down).

Question: how does Octokey know that a certain website URL is the same service as a certain iOS
bundle ID, which is the same service as an Android app signed with a certain public key?

\subsection{2D barcode interception}\label{sec:barcode-intercept}

In the flow for enrolling a new device (section~\ref{sec:newdevice}), and with delegated login
(section~\ref{sec:delegation}), we proposed displaying a 2D barcode on the screen of one device, and
scanning it with the camera of another. We cannot assume confidentiality of these barcodes: an
attacker may snoop it electromagnetically~\cite{Kuhn05}, or simply point a camera at the victim's
screen.

An attacker can thus connect to the URL in the barcode, and enroll the new device to the attacker's
account, instead of the user's own account as intended. With delegated login, the attacker can get
the user to log in with the attacker's key rather than the user's own key. This does not compromise
the user's key, but if the user logs in to the wrong account and doesn't realize it, they may
inadvertently disclose sensitive information to the attacker.

It is not clear whether this would be a problem in practice. If it is, a mutual authentication step
could be added to the flows for enrolling a new device and for delegated authentication (for
example, verifying a 2D barcode in the other direction). However, this makes the process more
complicated for users, especially when trying to delegate authentication to a device which has no
camera, so the additional verification step should probably be optional.

\subsection{2D barcode phishing}\label{sec:barcode-phishing}

When enrolling a new device or performing delegated authentication, we assume that an attacker
cannot trick a user into scanning a different barcode from the one they intended --- i.e. we assume
that the visual channel between the barcode-displaying and the barcode-scanning device provides
integrity (but not confidentiality). If the attacker is able to manipulate what is displayed on
screen, or somehow insert themselves into that channel, the user has bigger problems.

However, a real risk is that a malicious website or app displays a barcode that originates from an
attacker-controlled device, and tricks the user into scanning that barcode and granting the attacker
unwanted access. In this scenario we must rely on well-written warning messages and user education
to ensure the user really understands what they are doing.

For users who have already set up multiple physical devices, as an additional safeguard against
accidentally pairing with an attacker-controlled device, we can require that enrolling a new device
requires user approval on a quorum (e.g. majority) of existing devices. This does not require
additional cryptographic algorithms, but can simply be implemented as a policy in the Octokey
software.

\subsection{Key rotation}
\subsection{Multiple keypairs per user}
\subsection{Security levels}

% Sync login history across devices, and compare to login history reported by server, to detect
% use of a stolen key


\subsection{Denial of service}

Prevent denial of service due to rate limiting or due to an attacker revoking a user's active
devices.

In the worst case, if an attacker manages to get the entire private key (e.g. by stealing or
compromising two devices that are paired, and breaking the human-to-machine authentication step),
the user's last resort is to generate a new key and update their accounts on all services, adding
the new public key and removing the old one. Unfortunately, the same key-swapping can be performed
by the attacker to lock out the legitimate user.

Therefore, perhaps changing a user's public key on a service should require an additional hurdle,
e.g. using a recovery key that is only stored on paper but not electronically? Or a key that is
split 3 ways? But that would make key rotation difficult.
