\documentclass{llncs}
\usepackage{amsmath} % for \mod
\usepackage[hyphens]{url}
\usepackage{hyperref}

\newcommand*{\concat}{\mathbin{\|}}

\begin{document}
\title{Strengthening Public Key Authentication against Key Theft}
\subtitle{Short Paper}
\author{Martin Kleppmann\inst{1} \and Conrad Irwin\inst{2}}
\institute{
    \email{martin@kleppmann.com} \and \email{conrad.irwin@gmail.com}
}
\maketitle

\begin{abstract}
Authentication protocols based on an asymmetric keypair (e.g.\ SSH public key authentication, TLS
client certificates, FIDO UAF and U2F) can provide strong authentication provided that the private
key is adequately protected. Use of dedicated cryptographic hardware helps, but does not remove all
risks of key theft. In this paper we discuss algorithms for further protecting private key material
against theft, based on mediated RSA (mRSA) signatures. We show how users can revoke lost or stolen
devices without relying on a trusted authority. When private key material is encrypted with a
password, we show how a zero-knowledge proof can prevent offline brute-force attacks.
\end{abstract}

\section{Introduction}\label{sec:intro}

Although passwords are the prevalent authentication mechanism on the internet today, there are some
niches in which public key authentication systems have been successfully adopted. For example, SSH
public key authentication~\cite{SSH} is widely used for remote login to servers, TLS client
certificates~\cite{TLS} are used in some countries for access to public services~\cite{Parsovs14},
and FIDO U2F~\cite{FIDOOverview} provides 2-factor authentication for web applications.

These protocols require the user's private key to be stored on a device -- either the client device
or dedicated cryptographic hardware -- making them primarily machine-to-machine authentication
protocols (a ``something you have'' factor). To make the key less vulnerable to theft, the protocol
must be preceded by a human-to-machine authentication step: for example, a password or biometric
information can be used to unlock or decrypt the private key.

However, passwords and biometric identifiers are typically low in entropy, making them susceptible
to offline attacks if a device is stolen. In this paper we propose a scheme for storing an RSA
private key in a way that makes it harder for an attacker to make use of stolen key
material.\footnote{In this work we limit our attention to RSA, but we hope to extend our approach to
support other public-key cryptosystems such as ECC in future.}

In Sect.~\ref{sec:revocation} we discuss an algorithm for revoking key material on a device that is
lost, stolen or compromised, and in Sect.~\ref{sec:ratelimit} we show how to prevent offline attacks
(giving the user more time to revoke the compromised device's key before its human-to-machine
authentication step is broken).

\section{Revocable Public Key Authentication}\label{sec:revocation}

A service that needs to authenticate users may store a set of known public keys for a given username
$r$, or it may rely on a certificate authority (CA) to associate usernames with public keys. When
RSA~\cite{RSA} is used, a user's public key $(n, e)$ consists of the modulus $n$ and the public
exponent $e$.

\subsection{Constructing a Signature}\label{sec:signature}

Whenever a user wishes to log in, they must prove ownership of their private key $(n, d)$, where $d$
is the private exponent. For example, an RSA signature $s$ may be constructed as follows:
\begin{equation}
s = m^d = H(c \concat u \concat r)^d \mod n
\end{equation}
where $c$ is a challenge or session ID obtained from the service,\footnote{When TLS is used, a
channel binding~\cite{ChannelBinding} or Origin-Bound Certificate~\cite{Dietz12} may be incorporated
into the challenge $c$.} $u$ is the URL of the service, $r$ is the username, and $(d, n)$ is the
private key. The symbol $\concat$ denotes encoding and concatenating the values into a byte string.
$H$ is shorthand for the \textsc{EMSA-PSS-Encode} operation (hashing and padding) defined in
PKCS\#1~\cite{PKCS1}.

The client then sends the authentication request $(s, c, u, r, n, e)$ to the service, where it is
validated by checking that $s$ is a valid PKCS\#1 signature of $c \concat u \concat r$ using the
public key $(n, e)$, that $c$ and $u$ are valid for this service, and that $(n, e)$ is an acceptable
public key for user $r$.

\subsection{Key Splitting}\label{sec:splitting}

If the private exponent $d$ is stored on any device, even in encrypted form, it is susceptible to
theft and offline attacks. To prevent this, we split it into key fragments that are distributed
among several devices, using the \emph{mediated RSA} (mRSA) scheme~\cite{Boneh01,Kutyiowski12}.
mRSA is based on the identity:
\begin{equation}
s = m^d = m^{d_a + d_b} = m^{d_a} m^{d_b} \mod n
\end{equation}
provided that $d = d_a + d_b \mod \phi(n)$.

If two devices $a$ and $b$ each store a key fragment $d_a$ and $d_b$ respectively, whose sum is the
private exponent $d$, then we call those devices \emph{paired}. In order to generate a valid
signature, any two paired devices need to collaborate.

If device $a$ wants to generate a signature, it can send a signing request $\mathit{req}$ to device $b$:
\begin{equation}
\mathit{req} = H(c \concat u \concat r) \concat n \concat e
\end{equation}
where the public key $(n, e)$ indicates which key should be used, in case device $b$ stores multiple
keys. Device $b$ then uses its key fragment $d_b$ to calculate a response:
\begin{equation}
\mathit{resp} = H(c \concat u \concat r)^{d_b} = m^{d_b} \mod n
\end{equation}
and returns $\mathit{resp}$ to $a$. Now, $a$ can calculate the signature $s$:
\begin{equation}
s = H(c \concat u \concat r)^{d_a} \cdot \mathit{resp} = m^{d_a} m^{d_b} \mod n \enspace,
\end{equation}
construct a valid signature, and thus log in.

To revoke a device's key fragment, every device that is paired with the lost device must be
instructed to delete the key fragment from the pairing with the lost device. When all the paired
fragments have been deleted, the key fragments on the lost device become useless.

\subsection{The Mediator Service}\label{sec:mediator}

Splitting a key across two physical devices provides limited benefit: a user must carry both devices
with them, and if both are stolen at the same time, the revocation capability is lost. However,
there is a simple solution: one of the user's `devices' may be a remote service on the internet,
which we call the \emph{mediator}. This service stores key fragments that are paired with each of
the user's physical devices, and responds to signing requests by performing the modular
exponentiation using its key fragments. This allows a user to authenticate with services using only
one physical device, assuming that an internet connection is available.

When the user requires a device to be revoked, they must authenticate the revocation request from
one of their other devices (see Sect.~\ref{sec:ratelimit} for an algorithm). This implies that a
user must pair at least two physical devices with the mediator, so that the remaining device can
revoke a lost device. A paper print-out of the key can serve as last resort in case all devices are
lost or destroyed.

The mediator need only be partially trusted. It cannot authenticate as the user without the
cooperation of one of the user's physical devices. The user only needs to trust the mediator to not
collude with attackers who steal devices, and to correctly delete key fragments when the user
requires key revocation. The user's privacy is protected by hashing the message
$c \concat u \concat r$ before sending it to the mediator, so the mediator does not learn which
services the user is logging in to, or which usernames they are using.

From the point of view of a service that uses public key authentication, the mediator does not even
exist: a service simply verifies the RSA signature, and does not care how that
signature was constructed. This is in contrast to federated login systems such as OpenID, where the
relying party must trust the identity provider.

\section{Rate Limiting Password Guesses}\label{sec:ratelimit}

Besides enabling key revocation, mRSA can also be used to strengthen the human-to-machine
authentication step against offline attacks.

For example, say the key fragment on a device is encrypted with a symmetric key derived from a
password. Consider an attacker who has stolen this encrypted fragment. In order to brute-force the
password, the attacker needs a way of determining whether a password guess is correct. However, a
key fragment is just a uniformly distributed random number; by itself, the correctly decrypted key
fragment is almost indistinguishable from the garbage that results from attempting to decrypt with
the wrong password (see Sect.~\ref{sec:fragment-encryption}).

Assuming the attacker has no other key fragments, they can only determine whether the password guess
is correct by communicating with the mediator and testing whether they are able to construct a valid
signature. This gives us an opportunity to rate-limit password guessing attempts: if the mediator
receives too many requests based on an incorrect password, it can block further attempts and advise
the user to revoke the device pairing. Similar ideas have been used to strengthen key exchange
protocols against weak passwords~\cite{Bellovin92}.

In order to achieve this, we must design the protocol as a zero-knowledge proof, such that an
attacker must communicate with the mediator for every password guess, but without revealing the
password or the decrypted key fragment to the mediator. An algorithm is described in
Sect.~\ref{sec:mediator-auth}.

\subsection{Key Fragment Encryption}\label{sec:fragment-encryption}

Let $k$ be the minimum number of bits required to encode the RSA modulus $n$ (i.e.\ the RSA key
length). We can encode the key fragment $d_a$ as a $k$-bit string, using zero padding for the most
significant bits.

To encrypt the key fragment, we derive an encryption key from the password $\mathit{pass}$ using a
slow key derivation function (KDF) such as scrypt~\cite{Percival09}, and generate $k$ bits of key
stream, e.g.\ using AES-128 in CTR mode~\cite{Lipmaa00}:
\begin{equation}
\mathit{efrag} =
    \mathrm{AESCTR}(\mathit{ctr}, \mathrm{scrypt}(\mathit{pass}))_{\{0 \dots k-1\}} \oplus d_a
    \enspace,
\end{equation}
where $\mathit{ctr}$ is a random nonce that is stored in plaintext and incremented by AESCTR for
each block of key stream.

An attacker who has stolen $\mathit{efrag}$ may try to recover the key fragment by guessing a
password $\mathit{pass}^\prime$ (from a dictionary or by brute force). Decrypting with
$\mathit{pass}^\prime$ yields a guess $d_a^\prime$ of the key fragment:
\begin{equation}
d_a^\prime =
    \mathrm{AESCTR}(\mathit{ctr}, \mathrm{scrypt}(\mathit{pass}^\prime))_{\{0 \dots k-1\}} \oplus \mathit{efrag}
    \enspace.
\end{equation}

If an incorrect password is guessed, $d_a^\prime$ is a uniformly distributed pseudo-random number
between 0 and $2^k$. To what extent is it distinguishable from a correct key fragment? Key fragments
are chosen to be uniformly distributed between 0 and $d$. Since $d < 2^k$, a password guess that
results in a larger decrypted value is less likely to be correct than a password guess that results
in a smaller decrypted value. A password-guessing attacker can use this knowledge to prioritize
guesses.

To quantify the bias, we generated 50,000 RSA keys ($k=2048$ bits) using OpenSSL, and generated a
random $d_a$ with $0 \le d_a < d$ for each private exponent $d$. Table~\ref{tab:bias} shows the bias
in the most significant bits of $d_a$ when encoded in $k$ bits. The key fragments had an entropy of
2047.05 bits, implying that 0.95 bits of information is leaked by the bias.

\renewcommand{\arraystretch}{1.5}
\setlength\tabcolsep{4pt}
\begin{table}
\centering
\caption{Probability that bit $i$ of $d_a$ is 1, when encoded in $k=2048$ bits}\label{tab:bias}
\begin{tabular}{r|rrrrrrrrr}
$i$ & 2047 & 2046 & 2045 & 2044 & 2043 & 2042 & 2041 & 2040 & 2039 \\ \hline
Probability &
0.070 & 0.240 & 0.340 & 0.406 & 0.446 & 0.469 & 0.482 & 0.493 & 0.499
\end{tabular}
\end{table}

\subsection{Authenticating Requests to the Mediator}\label{sec:mediator-auth}

The algorithm of Sect.~\ref{sec:splitting} does not prevent offline attacks: an attacker could make
a single request to calculate $m^{d_b} \mod n$ for some $m$, and this would enable them to test
whether a password guess $\mathit{pass}^\prime$ is correct by checking whether
$m^{d_a^\prime} m^{d_b} \mod n$ is a valid signature. However, we can change the protocol so that
the mediator refuses to compute $m^{d_b}$ unless the device proves that it knows the correct $d_a$:

\begin{enumerate}
\item In addition to the signing request $\mathit{req}$, the client is required to submit a
signature $s_\mathit{req}$:
\begin{align}
    \mathit{req} &= H(c \concat u \concat r) \concat n \concat e \\
    s_\mathit{req} &= H(\mathit{req} \concat \mathit{cb})^{d_a^\prime} \mod n
\end{align}
where $\mathit{cb}$ is the \texttt{tls-unique} channel binding~\cite{ChannelBinding}
of the TLS connection between the client and the mediator.
\item Using the channel binding $\mathit{cb}^\prime$ of the TLS connection's server side, the
mediator computes
\begin{equation}
s_\mathit{req} \cdot H(\mathit{req} \concat \mathit{cb}^\prime)^{d_b} =
  H(\mathit{req} \concat \mathit{cb})^{d_a^\prime} \cdot
  H(\mathit{req} \concat \mathit{cb}^\prime)^{d_b} \mod n
\end{equation}
and checks whether the result is a valid signature of $\mathit{req} \concat \mathit{cb}^\prime$ for
the user's public key $(n, e)$. This check succeeds if $d_a^\prime = d_a$ (i.e.\ the user's password
was correct), and if $\mathit{cb}^\prime = \mathit{cb}$ (preventing MITM and replay attacks).
\item If the signature is valid, the mediator computes
\begin{equation}
\mathit{resp} = H(c \concat u \concat r)^{d_b} \mod n
\end{equation}
as before, and returns it to the client. If the signature is not valid, the mediator returns ``bad
signature''. A password-guessing attacker merely learns that the password guess
$\mathit{pass}^\prime$ was incorrect.
\end{enumerate}

This protection against password guessing only works if the attacker does not have any knowledge of
previous requests to the mediator. If the attacker knows $x^{d_a} \mod n$ (a request) or
$x^{d_b} \mod n$ (a response) for any $x$, they can brute-force the password without contacting the
mediator, and thus circumvent the rate-limiting.  It is therefore important that communication with
the mediator is protected from eavesdropping (using TLS) and is not logged on the device.

The same request authentication mechanism can be used to protect key revocation against abuse: the
mediator will only process a revocation request for a device if it is authenticated by another
device of the same user.

\section{Conclusion}

The security of key-based authentication is only as good as the protection of the private key
material. Encrypting the private key using a slow KDF is not sufficient to prevent offline attacks.

In this paper we discussed two related techniques for strengthening private key material in the
situation where a user's device is stolen or compromised: a revocation process that a user can
initiate on another device (without requiring a trusted authority or a backup key), and a
rate-limiting process for password guesses (giving the user more time to revoke the lost device's
key).

\bibliographystyle{splncs03}
\bibliography{references}{}

\end{document}
